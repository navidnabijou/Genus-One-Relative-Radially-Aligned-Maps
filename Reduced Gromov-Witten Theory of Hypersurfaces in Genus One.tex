\documentclass[11pt]{amsart}
%\linespread{1.25}
\usepackage[margin=3.5cm]{geometry}

\usepackage[noend,boxruled]{algorithm2e}
%Customize appearance using line below:
\SetKwFor{For}{\normalfont{for}}{:}{endfor}

\usepackage[english]{babel}
\usepackage{appendix}
\usepackage{amsmath}
\usepackage{amsfonts}
\usepackage{amssymb}
%\usepackage{showlabels}
\usepackage{hyperref}
\usepackage{amsthm}
\usepackage{marginnote}
\usepackage{stmaryrd}
\usepackage{enumitem}
\usepackage[english]{babel}
\usepackage{yfonts}
\usepackage[T1]{fontenc}
\usepackage[utf8x]{inputenc}

\usepackage{calrsfs}
\DeclareMathAlphabet{\pazocal}{OMS}{zplm}{m}{n}

\usepackage{verbatim}
\usepackage{graphicx}
\usepackage{verbatim}
\usepackage{faktor}
\usepackage{xcolor}
\usepackage{xfrac}
\usepackage{tikz,tikz-cd}
\usetikzlibrary{decorations.pathmorphing,decorations.pathreplacing,patterns,arrows.meta}

%%%%%%%%%%%%%%%%%%%
\begin{comment}
\usetikzlibrary{ipe} % ipe compatibility library
\tikzstyle{ipe stylesheet} = [
  ipe import,
  even odd rule,
  line join=round,
  line cap=butt,
  ipe pen normal/.style={line width=0.4},
  ipe pen heavier/.style={line width=0.8},
  ipe pen fat/.style={line width=1.2},
  ipe pen ultrafat/.style={line width=2},
  ipe pen normal,
  ipe mark normal/.style={ipe mark scale=3},
  ipe mark large/.style={ipe mark scale=5},
  ipe mark small/.style={ipe mark scale=2},
  ipe mark tiny/.style={ipe mark scale=1.1},
  ipe mark normal,
  /pgf/arrow keys/.cd,
  ipe arrow normal/.style={scale=7},
  ipe arrow large/.style={scale=10},
  ipe arrow small/.style={scale=5},
  ipe arrow tiny/.style={scale=3},
  ipe arrow normal,
  /tikz/.cd,
  ipe arrows, % update arrows
  <->/.tip = ipe normal,
  ipe dash normal/.style={dash pattern=},
  ipe dash dashed/.style={dash pattern=on 4bp off 4bp},
  ipe dash dotted/.style={dash pattern=on 1bp off 3bp},
  ipe dash dash dotted/.style={dash pattern=on 4bp off 2bp on 1bp off 2bp},
  ipe dash dash dot dotted/.style={dash pattern=on 4bp off 2bp on 1bp off 2bp on 1bp off 2bp},
  ipe dash normal,
  ipe node/.append style={font=\normalsize},
  ipe stretch normal/.style={ipe node stretch=1},
  ipe stretch normal,
  ipe opacity 10/.style={opacity=0.1},
  ipe opacity 30/.style={opacity=0.3},
  ipe opacity 50/.style={opacity=0.5},
  ipe opacity 75/.style={opacity=0.75},
  ipe opacity opaque/.style={opacity=1},
  ipe opacity opaque,
]
\end{comment}

\definecolor{red}{rgb}{1,0,0}
\usepackage[all]{xy}
\usepackage{bbm}
\usepackage{tabularx}
\usepackage{longtable}
\usepackage{tabu}
\usepackage{booktabs}
\usepackage{mathtools}

\usepackage[]{textcomp}
\usepackage[sups]{Baskervaldx}
\usepackage{cabin}
\usepackage[varqu,varl]{inconsolata}
\usepackage[baskervaldx,bigdelims,vvarbb]{newtxmath}
\usepackage[cal=cm]{mathalfa}


\newcommand{\plC}{\scalebox{0.8}[1.3]{$\sqsubset$}}
\newcommand{\sidenote}[1]{\marginpar{\textbf{\color{red}#1}}}
\newcommand{\lcm}{\operatorname{lcm}}

% FIGURES FOR USE LATER
%%%%%%%%%%%%%%%%%%%%%%%%%%%%%%%%%%%%%%%%%%%%%%%%%%%%%%%%%%%%
\def\Yagraph{\tikz[baseline=-3pt,scale=.8]{
\draw (2,0) -- (0,1) (2,0) -- (0,.5) (2,0) -- (0,-1);
\draw (2,0) circle(2pt)[fill=black];
\draw [->] (2,0) -- (2.7,0);
\draw (2.6,0) node[right]{\tiny{$x_k$}};
\draw (0,1) circle(2pt)[fill=white];
\draw (0,.5) circle(2pt)[fill=black];
\draw (0,-.25) node{$\vdots$};
\draw (0,-1) circle(2pt)[fill=black];
\draw [blue,fill=blue] (0,-2) circle[radius=2pt];
\draw [->,blue,thick] (0,-2) -- (3,-2);
\draw [blue,fill=blue] (2,-2) circle[radius=2pt];
\draw [->,blue] (1,-0.9) -- (1,-1.6);

\draw [blue] (2,-2) node[above]{\tiny$\diamond$};
\draw [blue] (0,-2) node[above]{\tiny$0$};

\draw (2.05,0) node[above]{\tiny{$\sqC_0$}};
\draw (0,1) node[left]{\tiny{$\sqC_1$}};
\draw (0,.5) node[left]{\tiny{$\sqC_2$}};
\draw (0,-1) node[left]{\tiny{$\sqC_r$}};
}}

\def\Ybgraph{\tikz[baseline=-3pt,scale=.8]{
\draw (2,0) to[out=120,in=0] (0,1) (2,0) -- (0,1) (2,0) -- (0,.5) (2,0) -- (0,-1);
\draw (2,0) circle(2pt)[fill=black];
\draw [->] (2,0) -- (2.7,0);
\draw (2.6,0) node[right]{\tiny{$x_k$}};
\draw (0,1) circle(2pt)[fill=black];
\draw (0,.5) circle(2pt)[fill=black];
\draw (0,-.25) node{$\vdots$};
\draw (0,-1) circle(2pt)[fill=black];
\draw [blue,fill=blue] (0,-2) circle[radius=2pt];
\draw [->,blue,thick] (0,-2) -- (3,-2);
\draw [blue,fill=blue] (2,-2) circle[radius=2pt];
\draw [->,blue] (1,-0.9) -- (1,-1.6);
\draw [blue] (2,-2) node[above]{\tiny$\diamond$};
\draw [blue] (0,-2) node[above]{\tiny$0$};

\draw (2.05,0) node[above]{\tiny{$\sqC_0$}};
\draw (0,1) node[left]{\tiny{$\sqC_1$}};
\draw (0,.5) node[left]{\tiny{$\sqC_3$}};
\draw (0,-1) node[left]{\tiny{$\sqC_r$}};
}}

\def\Ycgraph{\tikz[baseline=-3pt,scale=.8]{
\draw (2,0) -- (0,1) (2,0) -- (0,.5) (2,0) -- (0,-1);
\draw [->] (2,0) -- (2.7,0);
\draw (2.6,0) node[right]{\tiny{$x_k$}};
\draw (2,0) circle(2pt)[fill=white];
\draw (0,1) circle(2pt)[fill=black];
\draw (0,.5) circle(2pt)[fill=black];
\draw (0,-.25) node{$\vdots$};
\draw (0,-1) circle(2pt)[fill=black];
\draw [blue,fill=blue] (0,-2) circle[radius=2pt];
\draw [->,blue,thick] (0,-2) -- (3,-2);
\draw [blue,fill=blue] (2,-2) circle[radius=2pt];
\draw [blue] (2,-2) node[above]{\tiny$\diamond$};
\draw [blue] (0,-2) node[above]{\tiny$0$};
\draw [->,blue] (1,-0.9) -- (1,-1.6);

\draw (2.05,0) node[above]{\tiny{$\sqC_0$}};
\draw (0,1) node[left]{\tiny{$\sqC_1$}};
\draw (0,.5) node[left]{\tiny{$\sqC_2$}};
\draw (0,-1) node[left]{\tiny{$\sqC_r$}};
}}
%%%%%%%%%%%%%%%%%%%%%%%%%%%%%%%%%%%%%%%%%%%%%%%%%%%%%%%%%%%%%%

\newcommand{\mathsq}[1]{#1}

\newcommand{\sqC}{\scalebox{0.8}[1.3]{$\sqsubset$}}
\newcommand{\Kim}{\operatorname{Kim}}
\newcommand{\ACGS}{\operatorname{ACGS}}

\newcommand{\pM}{\pazocal{M}}
\newcommand{\TT}{\operatorname{T}}
\newcommand{\oM}{\overline{\mathcal{M}}}
\newcommand{\M}[4]{\overline{\mathcal{M}}_{#1,#2}(#3,#4)}

\newcommand{\Mlog}[4]{\overline{\mathcal{M}}^{\operatorname{log}}_{#1,#2}(#3,#4)}
\newcommand{\MLog}{\overline{\mathcal{M}}^{\operatorname{log}}}
\newcommand{\Mpunct}[4]{\overline{\mathcal{M}}^{\operatorname{punct}}_{#1,#2}(#3,#4)}
\newcommand{\MG}[4]{\overline{\mathcal{M}}^{\rm G}_{#1,#2}(#3,#4)}
\newcommand{\Q}[4]{\mathcal{Q}_{#1,#2}(#3,#4)}
\newcommand{\Qe}[4]{\mathcal{Q}^{\epsilon}_{#1,#2}(#3,#4)}
\newcommand{\Qt}[4]{\widetilde{\mathcal Q}_{#1,#2}(#3,#4)}
\newcommand{\QG}[4]{\mathcal{Q}G_{#1,#2}(#3,#4)}
\newcommand{\QGe}[4]{\mathcal{Q}G^{\epsilon}_{#1,#2}(#3,#4)}
\newcommand{\D}[3]{\mathcal{D^Q}(#1,#2,#3)}
\newcommand{\E}[3]{\mathcal{E^Q}(#1,#2,#3)}
\newcommand{\PP}{\mathbb P}
\newcommand{\Z}{\mathbb{Z}}
\newcommand{\VZ}{\pazocal{V\!Z}}
\newcommand{\tVZc}[4]{\widetilde{\mathcal{V\!Z}}^{\rm{ctr}}_{#1,#2}(#3,#4)}
\newcommand{\VZc}[4]{\mathcal{V\!Z}_{#1,#2}(#3,#4)}
\newcommand{\VZcLi}[4]{\mathcal{V\!Z}^{\rm{ctr, Li}}_{#1,#2}(#3,#4)}
\newcommand{\VZrel}[4]{\mathcal{V\!Z}^{\rm{rel}}_{#1,#2}(#3,#4)}
\newcommand{\stab}{\rm{stab}}
\newcommand{\st}{\star}
\newcommand{\stC}{C'}
\newcommand{\stpi}{\pi'}
\newcommand{\sts}{s'}
\newcommand{\N}{\mathbb{N}}
\newcommand{\OO}{\mathcal{O}}
\renewcommand{\to}{\rightarrow}
\newcommand{\A}{\mathcal A}
\newcommand{\B}{\mathcal B}
\newcommand{\C}{\mathfrak C}
\newcommand{\cC}{\mathcal C}
\newcommand{\EE}{\mathbf{E}}
\renewcommand{\L}{\mathcal L}
\newcommand{\LL}{\mathbf{L}}
\newcommand{\MM}{\mathfrak M}
\newcommand{\Aaff}{\mathbb{A}}
\newcommand{\kfield}{\Bbbk}
\newcommand{\comp}{\chi}
\newcommand{\sst}{\sigma^{\operatorname{ss}}}
\newcommand{\Pic}{\operatorname{Pic}}
\newcommand{\Def}{\operatorname{Def}}
\newcommand{\Spec}{\operatorname{Spec}}
\newcommand{\Proj}{\operatorname{Proj}}
\newcommand{\Hom}{\operatorname{Hom}}
\newcommand{\Ext}{\operatorname{Ext}}
\newcommand{\Gm}{\mathbb{G}_{\text{m}}}
\newcommand{\virt}[1]{[#1]^{\operatorname{virt}}}
\newcommand{\vip}[1]{[#1]^{\operatorname{prod}}}
\newcommand{\Id}{\operatorname{Id}}
\newcommand{\CC}{\mathbb{C}}
\newcommand{\QQ}{\mathbb{Q}}
\newcommand{\HH}{\operatorname{H}}
\newcommand{\Achow}{\operatorname{A}}
\newcommand{\pt}{\operatorname{pt}}
\newcommand{\bq}{\begin{equation}}
\newcommand{\eq}{\end{equation}}
\newcommand{\ba}{\begin{aligned}}
\newcommand{\ea}{\end{aligned}}
\newcommand{\be}{\begin{enumerate}}
\newcommand{\ee}{\end{enumerate}}
\newcommand{\bsm}{\left(\begin{smallmatrix}}
\newcommand{\esm}{\end{smallmatrix}\right)}                   
\newcommand{\bpm}{\begin{pmatrix}}
\newcommand{\epm}{\end{pmatrix}}
\newcommand{\barr}{\begin{displaymath}\begin{array}{cccc}}
\newcommand{\earr}{\end{array}\end{displaymath}}
\newcommand{\barrl}{\begin{displaymath}\begin{array}{lcl}}
\newcommand{\earrl}{\end{array}\end{displaymath}}
\newcommand{\barl}{\begin{displaymath}\begin{array}{l}}
\newcommand{\earl}{\end{array}\end{displaymath}}
\newcommand{\bxym}{ \begin{displaymath}\xymatrix }
\newcommand{\exym}{\end{displaymath}}
\newcommand{\bcd}{\begin{center}\begin{tikzcd}}
\newcommand{\ecd}{\end{tikzcd}\end{center}}
\newcommand{\R}{\operatorname{R}^{\bullet}}
\newcommand{\dvr}{\Delta}
%\newcommand{\sslash}{\mathbin{/\mkern-6mu/}}
\newcommand{\tr}{{\rm tr}}
\newcommand{\Isom}{\text{Isom}}
\newcommand{\pr}{\operatorname{pr}}
\newcommand{\ev}{\operatorname{ev}}
\newcommand{\fgt}{\operatorname{fgt}}
\newcommand{\codim}{\operatorname{codim}}
\newcommand{\vdim}{\operatorname{vdim}}
\newcommand{\ildef}[1]{\emph{#1}}
\newcommand{\om}[1]{\mathcal{#1}}
\newcommand{\h}{\operatorname{h}}
\newcommand{\Aut}{\operatorname{Aut}}
\newcommand{\Acal}{\mathcal{A}}
\newcommand{\Scal}{\mathcal{S}}
\newcommand{\Mcal}{\mathcal{M}}
\newcommand{\Dcal}{\mathcal{D}}
\newcommand{\Ecal}{\mathcal{E}}
\newcommand{\Ncal}{\mathcal{N}}
\newcommand{\Fcal}{\mathcal{F}}
\newcommand{\Lcal}{\mathcal{L}}
\newcommand{\Ccal}{\mathcal{C}}
\newcommand{\Pcal}{\mathcal{P}}
\newcommand{\Ycal}{\mathcal{Y}}
\newcommand{\cchern}{\mathrm{c}}
\newcommand{\ol}[1]{\overline{#1}}
\newcommand{\ul}[1]{\underline{#1}}
\newcommand{\op}[1]{\operatorname{#1}}
\newcommand{\gp}{\operatorname{gp}}
\newcommand{\RR}{\mathbb{R}}
\newcommand{\NN}{\mathbb{N}}
\newcommand{\ovm}[1]{\overline{\mathcal{#1}}}
\newcommand{\ovt}[1]{\widetilde{\mathcal{#1}}}
\newcommand{\ov}[1]{\overline{#1}}

\theoremstyle{definition}
\newtheorem{thm}{Theorem}[section]
\newtheorem{lem}[thm]{Lemma}
\newtheorem{lemma}[thm]{Lemma}
\newtheorem{prop}[thm]{Proposition}
\newtheorem{cor}[thm]{Corollary}
\newtheorem*{teo*}{Theorem}
\newtheorem{ipotesi}{ipotesi}
\newtheorem*{nota}{Nota}
\newtheorem{claim}{Claim}
\newtheorem{question}[thm]{Question}
\newtheorem{conj}[thm]{Conjecture}
\newtheorem{notation}[thm]{Notation}

\newtheorem{innercustomthm}{Theorem}
\newenvironment{customthm}[1]
  {\renewcommand\theinnercustomthm{#1}\innercustomthm}
  {\endinnercustomthm}

\theoremstyle{definition}
\newtheorem{example}[thm]{Example}
\newtheorem{ex}[thm]{Example}
\newtheorem{dfn}[thm]{Definition}
\newtheorem{definition}[thm]{Definition}
\newtheorem{aside}[thm]{Aside}
\newtheorem{remark}[thm]{Remark}
\newtheorem{com}[thm]{Comment}
\newtheorem{num}{Number}
\newtheorem*{sketch}{Sketch}
\newtheorem*{rem}{Remark}
\newtheorem*{aside*}{Aside}
\newtheorem*{acknowledgements}{Acknowledgements}

\newlist{steps}{enumerate}{1}
\setlist[steps, 1]{label = Step (\arabic*):}

\newcommand{\ilemph}[1]{\emph{#1}}

\setcounter{tocdepth}{1}

\newcommand{\todo}[1]{\vspace{5mm}\par \noindent
\framebox{\begin{minipage}[c]{0.95 \textwidth} \tt #1\end{minipage}} \vspace{5mm} \par}

\def\ti{-\allowhyphens}
\newcommand{\thismonth}{\ifcase\month % case 0 --- impossible!
  \or January\or February\or March\or April\or May\or June%
  \or July\or August\or September\or October\or November%
  \or December\fi}
\newcommand{\thismonthyear}{{\thismonth} {\number\year}}
\newcommand{\thisdaymonthyear}{{\number\day} {\thismonth} {\number\year}}

\title{Curve counting in genus one: elliptic singularities and relative geometry}
\author{Luca Battistella, Navid Nabijou and Dhruv Ranganathan}
\date{\thismonthyear}

\begin{document}


\begin{abstract}
\end{abstract}

\maketitle

\appendixtitletocoff
\tableofcontents


\section*{Introduction}

\noindent Gromov--Witten theory in genus zero has been studied intensely using a large assortment of techniques, many of which rely crucially on the remarkable geometry of the Kontsevich space of genus zero maps to projective space. On the other hand, the higher genus spaces have uncontrollable singularities, and an entirely different set of techniques must be appealed to. Indeed, much less is known in higher genus, though the last few years have seen remarkable progress. 

\subsection{The transition point in genus one} The singularities of the space of maps in genus one are delicate, but understandable. Vakil and Zinger exhibited this by giving an explicit blowup algorithmic that desingularized the space of maps to projective space, giving rise to ``reduced’’ Gromov—Witten invariants, which are closer to enumerative counts~\cite{VZ}. Certain tools, such as localization and quantum Lefschetz, can thus be extended to genus one. However, there is as yet no degeneration formula for the Vakil—Zinger invariants. 

Recently, a more conceptual and geometric understanding of Vakil and Zinger’s construction has come out of tropical geometry and logarithmic Gromov—Witten theory. Specifically, Ranganathan, Santos-Parker, and Wise give a modular interpretation of the construction and a deformation theoretic proof its smoothness by using the geometry of elliptic singularities~\cite{RSPW,RSPW2}. In this paper we seek an understanding of Vakil—Zinger invariants in relative geometries and the degeneration formula.

\subsection{Results} Our first contribution is a definition and construction of relative reduced Gromov--Witten theory of projective space relative to a hyperplane. The geometric content is the following.

\begin{customthm}{A}
There exists a logarithmically smooth and proper Deligne--Mumford stack giving a modular compactification for the space of genus one maps to projective space with fixed contact order with a hyperplane. 
\end{customthm}

With regards to the degeneration formula, we pursue the strategy laid out by Vakil and Gathmann, building on work of Caporaso and Harris. We reinterpret and extend their methods using logarithmic geometry and tropical techniques. Our main result is the following.

\begin{customthm}{B}
Given a smooth pair $(X,Y)$ with $Y$ very ample, there is an explicit recursive algorithm to calculate
\begin{enumerate}
\item the reduced genus one Gromov--Witten invariants of $Y$;
\item the reduced genus one relative Gromov--Witten invariants of $(X,Y)$; 
\item the reduced genus one rubber invariants of $P=\PP_Y(\operatorname{N}_{Y|X} \oplus \OO_Y)$
\end{enumerate}
from the invariants of the ambient variety $X$.
\end{customthm}

\subsection{Broader contributions} Beyond genus one, we pass through a number of seemingly useful general techniques. First, we explain how to describe degenerate moduli spaces as fiber products. This is substantially more delicate than the standard situation because of the need to work with elliptic singularities and aligned logarithmic structures. This interaction of the degeneration formula with the geometry of curve singularities is a new phenomenon. The results of~\cite{RSPW} suggest the existence of a reduced higher genus Gromov--Witten theory formed by replacing contracted elliptic components with singularities\footnote{We have been informed by Jonathan Wise that his Ph.D. candidate Sebastian Bozlee has constructed these higher genus reduced Gromov--Witten invariants. A manuscript is in preparation.}. The discussion here is likely to carry over to this setting.

Second, a key step in our recursion is a description of the locus of maps with higher than prescribed tangency, which was identified by Gathmann. We realize the locus as the vanishing locus of a section of a line bundle that comes from tropical geometry — from a piecewise linear function on the fan. The systematic understanding of logarithmic line bundles arising in this fashion is likely to play an important in logarithmic enumerative geometry.

Our analysis leads naturally to a study of a ``main component’’ double ramification cycle for target manifolds in genus one. The virtual geometry of this was studied in recent work of Janda, Pandharipande, Pixton, and Zvonkine~\cite{DRCBundle}. We do not know how to calculate an expression for this main component contribution, but we repurpose E.~Katz's topological recursion relations \cite{EKatz} to complete the algorithm. We note that even in genus one, an understanding of the main component contributions of the double ramification cycle for target manifolds would be interesting.

Finally, we note that our results require us to encounter several variants of the space of relative/logarithmic stable maps, with both fixed and expanded target. The conceptual features of one space are often computational bugs, and vice versa, and we believe that the techniques developed in the genus one case here will be used repeatedly in logarithmic Gromov--Witten theory calculations. In particular, we use frequently use Kato's perspective on logarithmic blowups as subfunctors that was applied in~\cite{RSPW,RSPW2}.

\subsection*{Acknowledgements} We have benefited from fruitful conversations with friends and colleagues, including Davesh Maulik, Jonathan Wise, and [other people]. Important  aspects of this work were completed while L.B. and N.N. were visiting MIT in Spring 2018 and the University of Cambridge in Winter 2019, and we thank these institutions for their hospitality.

\part{The moduli space and its deformation theory}

\section{Construction and logarithmic smoothness}

\subsection{The absolute story} Let $\mathfrak M_{1,n}$ be the logarithmic algebraic stack of genus $1$ $n$-pointed prestable curves. Let $\plC$ be a $n$-pointed genus $1$ tropical curve. Let $C$ be a logarithmic curve over $S$, let $\plC$ denote its tropicalization, and let $\lambda$ be the section of $\overline{M}_{C/S}$ giving the distance from the core. 

We begin by recalling the logarithmic moduli spaces of genus $1$ curves constructed in~\cite[Sections 2 \& 4]{RSPW}. Given a family of tropical curves over a base $\sigma$, a central alignment is a piecewise-linearly varying choice of radius $\delta_s$ for $s\in\sigma$, together with a consistent ordering of the vertices that lie inside the circle of radius $s$ around the core components. Precisely, let $T$ be a geometric point with logarithmic structure.

\begin{definition}
A \textbf{central alignment} of $C/T$ is an element $\delta\in\overline{M}_S$ such that
\begin{enumerate}
    \item the section $\delta$ is comparable to $\lambda(v)$ for all vertices $v$ of $\plC$,
    \item for any pair of vertices $v$ and $w$ at distance less than $\delta$, the sections $\lambda(v)$ and $\lambda(w)$ are comparable.
\end{enumerate}
\end{definition}

The moduli stack $\mathfrak M_{1,n}^{\mathrm{cen}}$ is a logarithmic algebraic stack in the smooth topology. The main construction of~\cite{RSPW} canonically associates to any radially aligned family of curves $\mathcal C_S$ a partial destabilization $\widetilde{\mathcal C_S}$ and a contraction
\[
\widetilde{\mathcal C_S}\to \overline{\mathcal C_S},
\]
where $\overline{\mathcal C_S}$ is a Gorenstein elliptic singular curve. The number of branches is equal to the number of excident edges at the circle of radius $\delta$. This data uniquely determines the singularity.

The space of stable maps $\overline{\mathcal M}^{\mathrm{cen}}_{1,n}(\mathbb P^N,d)$ from the universal nodal curve $\mathcal C$ over the stack $\mathfrak M_{1,n}^{\mathrm{cen}}$ is a proper and algebraic, with projective coarse moduli. The \textbf{factorization condition} that the map $\mathcal C\to \mathbb P^N$ factors through the contraction $\widetilde{\mathcal C}\to \overline{\mathcal C}$ is a closed condition. 
\begin{thm}[{\cite[Theorem B]{RSPW}}]
The substack $\mathcal{VZ}_{1,n}(\mathbb P^N,d)$ of $\overline{\mathcal M}^{\mathrm{cen}}_{1,n}(\mathbb P^N,d)$ parameterizing maps to $\mathbb P^N$ that satisfy the factorization property is smooth and proper of the expected dimension.
\end{thm}

\subsection{Relative geometry: compactification} Fix $H\subset \mathbb P^N$ be a hyperplane. Let $\alpha$ be a partition of the degree $d$. Consider the moduli space $\mathcal M_{1,\alpha}^\circ(\mathbb P^N|H)$ of maps from smooth elliptic curves $C\to \mathbb P^N$ that meet $H$ at finitely many marked points with vanishing orders given by the partition $\alpha$. This is a smooth non-compact Deligne--Mumford stack. 

We will first compactify the space described above, and then desingularize it. For the compactification, we begin with Abramovich--Chen--Gross--Siebert's space of logarithmic stable maps, though we will typically work with various subcategories and variants. 

The moduli space $\overline{\mathcal{M}}^{\operatorname{log}}_{1,\alpha}(\mathbb P^N|H)$ is a fibred category over logarithmic schemes, whose fiber over $(S,M_S)$ is the groupoid of logarithmic curves of genus $1$ over $(S,M_S)$ equipped with a map to $\mathbb P^N$ of degree $d$ and contact order $\alpha$. It is a fundamental fact in the subject that this category is representable by a proper algebraic stack with logarithmic structure. 

There is a representable finite logarithmic morphism to the Kontsevich space, forgetting the logarithmic structure on the target:
$$
\overline{\mathcal M}^{\mathrm{log}}_{1,\alpha}(\mathbb P^N|H) \to \overline{\mathcal M}_{1,n}(\mathbb P^N,d).
$$
The space of centrally aligned maps is a logarithmic modification
$$
\overline{\mathcal M}^{\mathrm{cen}}_{1,n}(\mathbb P^N,d)\to\overline{ \mathcal M}_{1,n}(\mathbb P^N,d)
$$ 
The fiber product leads to a fourth moduli space of \textbf{centrally aligned logarithmic maps to $(\mathbb P^n,H)$}, which we denote $\widetilde{\mathcal{VZ}}_{1,\alpha}(\mathbb P^N|H)$.


\subsection{Relative geometry: expansions} Our next task is to pick out a non-singular principal component in $\widetilde{\mathcal{VZ}}_{1,\alpha}(\mathbb P^N|H)$. The principal component of this space, consisting of the closure of the space of maps from nonsingular curves, maps into the principal component of $\overline{\mathcal M}^{\mathrm{cen}}_{1,n}(\mathbb P^N,d)$. Indeed, smoothable logarithmic maps are, in particular, smoothable as ordinary maps. However, an additional condition is required to isolate the principal component of the space of logarithmic maps. 

In order make the obstruction theory more geometric, we expand the target. To elucidate the connection with the static target, consider a logarithmic stable map $[C\to (\mathbb P|H)]$ over $\mathrm{Spec} \ (\mathbb N\to \mathbb C)$. At the level of tropicalizations, we have a map of fans
\[
\plC\to \mathbb R_{\geq 0}.
\]
Choose a subdivision of $\mathbb R_{\geq 0}$ whose vertices consist of the images of vertices of $\plC$. Pull this subdivision back to $\plC$ by marking all preimages of the vertices of $\mathbb R_{\geq 0}$. Denote the resulting map $\widetilde \plC \to \widetilde{\mathbb R}_{\geq 0}$. 

These subdivisions induce logarithmic modifications
\[
\widetilde C\to \mathbb P^N[s],
\]
see~\cite{AW}. Here the latter is the $s$-times iterated deformation to the normal cone of $H$ in $\mathbb P^N$. The number of components is equal to the number of vertices in $\widetilde{\mathbb R}_{\geq 0}$. The curve is modified by adding rational components corresponding to the newly introduced vertices.

The result is a logarithmic stable map to an expanded target $\mathbb P^N[s]$, together with a contraction to the main target component $\mathbb P^N[s]\to\mathbb P^N$. More generally, this construction is easily modified for any family of logarithmic maps, such that the vertices of $\plC$ are totally ordered in the family, see~\cite{} for details.

Globally, Kim constructs a moduli space of logarithmic stable maps to expanded degenerations, which on logarithmic points, reduces to the above construction~\cite{KimLog}. Indeed, by above description, Kim's space is identified with a subcategory of the Abramovich--Chen--Gross--Siebert space, and its minimal objects are identified with a logarithmic modification of the unexpanded space.

\subsection{Relative geometry: factorization}\label{subsection factorisation} Let $[C\to \mathbb P^N[s]\to \mathbb P^N]$ be a logarithmic map from a centrally aligned curve to an expansion. Recall that $\mathbb P^N[s]$ consists of a union of $\mathbb P^N$ with the projective bundle $\mathbb P(\mathcal O\oplus \mathcal O(1))$. We refer to these latter components as \textbf{the higher levels}. Thus, we will say that subcurve $D\subset C$ maps to \textbf{higher level} if the collapsed map
\[
D\to \mathbb P^N
\]
is contained in $H\subset \mathbb P^N$. 

Let $D_1\subset C$ be the maximal genus $1$ subcurve that is mapped to higher level contracted by the map $C\to \mathbb P^N[s]$ in the fiber direction. Let $\delta_1$ be the associated radius from the core to the nearest non-contract component. Let $D_2$ be the maximal genus $1$ subcurve that is contracted by the collapsed map to $\mathbb P^N$, and let $\delta_2$ be the associated radius. This coincides with the radius of the underlying map to $\mathbb P^N$. Of course, $\delta_1\leq \delta_2$.

The datum $(\delta_1,\delta_2)$ determines a destabilization $\widetilde C$ of $C$ together with successive contractions $\widetilde C\to \overline C_1\to\overline C_2$. Keeping this definition, we come to the key definition.

\begin{definition}
The map $[C\to \mathbb P^N[s]\to \mathbb P^N]$ \textbf{factorizes completely} if
\begin{itemize}
\item the map $C\to \mathbb P^N[s]$ factors through $\overline C_1$ such that at least one branch of $\overline C_1$ has positive degree in the fiber direction. 
\item the collapsed map to $\mathbb P^N$ factorizes through $\overline C_2$ such that at least one branch of $\overline C_2$ has positive degree.
\end{itemize}
\end{definition}

In particular, if $[C\to \mathbb P^N[s]\to \mathbb P^N]$ is a family of centrally aligned maps over $S$ that factorizes completely, there is a forgetful moduli map $S\to \mathcal{VZ}_{1,n}(\mathbb P^N,d)$, to the principal component of the space of absolute maps.

Let $\mathcal{VZ}_{1,\alpha}(\mathbb P^N|H)$ be stack of maps from centrally aligned curves to expansions of $\mathbb P^N$ that factorize completely.

\begin{thm}\label{thm: log-smoothness}
The stack $\mathcal{VZ}_{1,\alpha}(\mathbb P^N|H)$ is proper and logarithmically non-singular over $\mathrm{Spec} \ \mathbb C$. 
\end{thm}

\begin{proof}
We prove the results via the forgetful morphism
$$
\nu: \mathcal{VZ}_{1,\alpha}(\mathbb P^N|H)\to \mathcal{VZ}_{1,n}(\mathbb P^N,d),
$$
which remembers only the stabilization of the collapsed map. The map is certainly centrally aligned, and we have already argued above that it satisfies the factorization property for $\mathbb P^N$ after composition. The morphism $\nu$ is in fact proper: the verification for the valuative criterion is the essential content of Vakil's Lemma~\cite[Lemma~5.9]{Vak}, and a detailed proof may be found in~\cite[Theorem~4.3]{RSPW}.

We come to logarithmic smoothness. Assume first that the elliptic curve maps into higher level. Note that the relative logarithmic tangent bundle of the expansion, over the base $\mathbb P^N$ is a trivial line bundle of rank $1$, as the fibers are toric. Consider an $S$-family of maps. After replacing the source $C$ by a destabilization, we have maps
\bcd
C \ar[r] & \overline{C}_1 \ar[r] \ar[d] & \overline{C}_2 \ar[d] \\
& \mathbb P^N[s] \ar[r] & \mathbb P^N
\ecd
factorizing completely. Let $f_B$ and $f_T$ be the maps from $C$ to the base and total space respectively. Examining the morphism $\nu$, we see that there is a map
\[
\mathrm{Def}(C,f_B)\to \mathrm{Obs}(f_T/(C,f_B)) = H^1(\overline{C}_1,\mathcal O_{\overline{C}_1}),
\]
where the former is the space of deformations of the curve and map to the base, as a factorized centrally aligned map. The latter is the space of obstructions to lifting a map to the base into the total space. The cokernel of this map are the absolute obstructions, which we will show vanishes, proving logarithmic smoothness.

We recall how the group $H^1(\overline{C}_1,\mathcal O_{\overline{C}_1})$ functions as the obstruction group for the lifting. It suffices to work near the minimal grnus $1$ subcurve $D_1$ of $C_1$, since the rest of the curve is rational. This lifting is given by a rational function with prescribed orders of poles given by the slopes of the tropicalization map. That is, if $\alpha$ be the piecewise linear function giving this tropical map, then the lifting is described by a section of the associated bundle $\mathcal O_{C_1}(-\alpha)$.

Given a strict square-zero extension $S'$ of $S$, the piecewise linear function $\alpha$ extends uniquely by strictness to any deformation of the curve. The resulting deformation of $\mathcal O_{C_1}(-\alpha)$ produces an infinitesimal deformation of the trivial bundle in the Picard group, whose class in $H^1(\overline{C}_1,\mathcal O_{\overline{C}_1})$, is the obstruction to deformation. We will show that it is possible to choose a deformation of $(C,f_B)$ that cancels out the obstruction to lifting to $f_T$, and thus the map to the obstruction space above is surjective. 

The line bundle $\mathcal O_{C_1}(-\alpha)$ is equivalent to the divisor $\sum a_i x_i$ where $x_i$ are the points connecting $D_1$ to the rest of the curve, and $a_i$ are the slopes of $\alpha$ along the edges corresponding to the $x_i$. Infinitesimally moving the point $x_i$ is a deformation of the curve that is unobstructed by the map $f_B$, since the map is constant on the interiors of the circuits. By deforming the $x_i$, at least one of which is nonzero, we produce a one-dimensional space of obstructions. The absolute obstructions therefore vanish, since they are the cokernel of a surjective map, so the result follows.  
%Consider an $S$-family of expanded maps $[F: C\to \mathbb P^N[s]$. Choose $N$ generic hyperplanes in and let $\Delta = \{H\}\cup\{H_1,\ldots,H_N}$ be this set of hyperplanes. There is a morphism of logarithmic schemes $(\mathbb P^N,\Delta)\to (\mathbb P^N,H)$. This induces a new expansion $\mathbb P^N[\Delta,s]$. Moreover, by the genericity of the $H_i$, they each intersect the image of $C$ in finitely many reduced points. Pulling back the logarithmic structure, we obtain a new logarithmic map 
%\[
%F': C'\to \mathbb P^N[\Delta,s].
%\]
%We make two observations. First, since the logarithmic structure on $C'$ is strict away from $H$, unobstructedness of deformations for $F'$ is equivalent to unobstructedness for $F$. Second, the target degeneration is now a toric degeneration, and the logarithmic tangent bundle is trivial.
\end{proof}

The above theorem guarantees that the singularities of the space of totally factorized maps to expansions is logarithmically smooth. In fact, one can say more in this case.

\begin{cor}
The logarithmically smooth stack $\mathcal{VZ}_{1,\alpha}(\mathbb P^N|H)$ has at worst orbifold singularities, i.e. admits a non-representable cover by a smooth Deligne--Mumford stack.
\end{cor}

\begin{proof}
Since $\mathcal{VZ}_{1,\alpha}(\mathbb P^N|H)$ is logarithmically smooth, it will suffice to show that the cones of its tropicalization are simplicial. To see this, consider a logarithmic stable map to an expansion $C\to \mathbb P^N[s]$ without a central alignment. The tropical moduli cone obtained as the dual of the minimal base monoid can be identified with $\mathbb R_{\geq 0}^{s}$, see for instance~\cite[Section~2.2]{ChenDegeneration}. The alignment procedure is an iterated barycentric subdivision at the level of tropical moduli spaces, as explained in~\cite[Section 4.6]{RSPW}, and such subdivisions preserve the property of being simplicial. We conclude from this that the blowup $\widetilde{\mathcal{VZ}}^{\mathrm{exp}}_{1,\alpha}(\mathbb P^N|H)$ obtained by centrally aligning Kim's spaces has simplicial cones. Finally, the morphism $\mathcal{VZ}_{1,\alpha}(\mathbb P^N|H)\to \widetilde{\mathcal{VZ}}^{\mathrm{exp}}_{1,\alpha}(\mathbb P^N|H)$ is strict, so the cones of $\mathcal{VZ}_{1,\alpha}(\mathbb P^N|H)$ are simplicial, as claimed.
\end{proof}

\section{Stratification}

\noindent The logarithmic smoothness established in Theorem~\ref{thm: log-smoothness} implies that $\mathcal{VZ}_{1,\alpha}(\mathbb P^N|H,d)$ is an orbifold toroidal embeddeding. Consequently, the irreducible components of its boundary and their intersections form a stratification of the space. This stratification will be important in the sequel. 

The tropicalizations of logarithmic maps that factorize completely satisfy a natural combinatorial condition known as well-spacedness. The version we use here is essentially identical to~
\cite{RSPW2}. 

\begin{definition}
Let $\plC$ be a tropical curve of genus $1$ and let $\plC_0$ be its minimal subcurve of genus $1$. A tropical map $F: \plC\to \mathbb R_{\geq 0}$ is said to be \textbf{well-spaced} if one of the following two conditions are satisfied:
\begin{enumerate}
    \item no open neighborhood of $\plC_0$ is contracted to a point in $\mathbb R_{>0}$, or
    \item if an open neighborhood of $\plC_0$ is contracted and $t_1,\ldots,t_k$ are the vertex-edge direction flags whose vertex is mapped to $F(\plC_0)$, but along which $F$ has nonzero slope; then, the minimum of the distances from $\plC_0$ to $t_i$ occurs for at least two indices $i$.
\end{enumerate}
\end{definition}

\begin{prop}\label{prop: well-spaced}
Let $[C\to \mathbb P^N]$ be a logarithmic stable map from a centrally aligned curve to an expansion, that factorizes completely. Then the tropicalization $\plC\to \mathbb R_{\geq 0}$ is well-spaced.
\end{prop}

\begin{proof}
The required result is essentially contained in~\cite[Section 4]{RSPW2}, so we explain how to deduce the requisite result from the one in loc. cit. We may focus on a single component of the expansion $\mathbb P^N[s]$ that contains the image of a contracted genus $1$ subcurve, as this is the only relevant case. We may thus replace the target with the projective bundle $\mathbb P(\mathcal O(1)\oplus \mathcal O)$ over $\mathbb P^{N-1}$ equipped with the divisorial logarithmic structure from the $0$ and $\infty$ sections. Let $p$ be the point to which the genus $1$ subcurve is contracted. Passing to an open neighborhood of $p$, the map to the bundle is given by a rational function $f$ on an open curve $C^\circ$, determined by the the subgraph formed by $\plC_0$ and the flags $t_i$ described in the definition above. To describe the tropical map to $\mathbb R_{>0}$, we observe that $\plC_0$ is contracted to a fixed point $q\in\mathbb R_{>0}$. The flags at the vertex bases at $t_i$ correspond to nodes or markings of $C$. The pole orders of $f$ at these distinguished points determine the slopes of the tropical map. We are now exactly in the situation considered in~\cite[Second Paragraph of Section~4.6]{RSPW2}, and Lemma~4.6.1 of loc. cit. guarantees the well-spacedness as required.
\end{proof}

\subsection{The cone complex} To understand the stratification via combinatorial data, we will build the straification from known objects. \\

\noindent
\textbf{Step 1}. Let $\Sigma^{\mathrm{log}}$ be the tropical moduli space of genus $1$ tropical stable maps to $\mathbb R_{\geq 0}$. This is naturally identified with the tropicalization (in the logarithmic sense) of the Abramovich--Chen--Gross--Siebert space of logarithmic stable maps to the pair $(\mathbb P^N,H)$. \\

\noindent
\textbf{Step 2}. Given such a tropical stable map, we may subdivide $\mathbb R_{\geq 0}$ such that the image of every vertex of $\plC$ is a vertex of $\mathbb R_{\geq 0}$. Call this subdivision $\widetilde{\mathbb R}_{\geq 0}$. The preimages of vertices of the subdivision form a subdivision of $\plC$. After this procedure, the images of vertices of $\plC$ which lie in $\RR_{>0}$ are totally ordered, in a manner extending the partial order obtained from the map to $\mathbb R_{\geq 0}$. The combinatorial types of such \textbf{image-ordered} maps produce the cones of a subdivision of $\Sigma^{\mathrm{log}}$ which we denote $\Sigma^{\mathrm{Kim}}$.\\

\noindent
\textbf{Step 3}. Given a tropical map $F: \plC\to \mathbb R_{\geq 0}$ parameterized by $\Sigma^{\mathrm{Kim}}$, there is a largest radius $\delta$ (possible equal to $0$) such that every vertex strictly contained in the circle of radius $\delta$ around the core has degree-marking $d=0$. Let $
\Sigma^{\mathrm{cen}}$ be the subdivision obtained by requiring that the vertices contained with the circle of radius $\delta$ around the core of $\plC$ are totally ordered. This involves introducing cones along which certain $\Z$-linear combinations of edge lengths are identified.\\

\noindent
\textbf{Step 4}. Let $
\Sigma_{1,\alpha}(\mathbb P^N|H,d)$ be the subcomplex of $\Sigma^{\mathrm{cen}}$ obtained by passing to the subcomplex of $\Sigma^{\mathrm{cen}}$ that consists of well-spaced tropical maps.

\begin{prop}
The cone complex $
\Sigma_{1,\alpha}(\mathbb P^N|H,d)$ is the fan of the toroidal embedding $
\mathcal{VZ}_{1,\alpha}(\mathbb P^N|H)$. In particular, the codimension $k$ strata of $
\mathcal{VZ}_{1,\alpha}(\mathbb P^N|H)$ are in inclusion reversing bijection with the dimension $k$ cones in $
\Sigma_{1,\alpha}(\mathbb P^N|H,d)$.
\end{prop}

\begin{proof}
The construction above has been given to mimic the construction of the space $
\mathcal{VZ}_{1,\alpha}(\mathbb P^N|H)$. Specifically, the fact that cone complex $\Sigma^{\mathrm{cen}}$ is the cone complex attached to the logarithmic stack $\widetilde{\mathcal{VZ}}_{1,\alpha}(\mathbb P^N|H)$ of centrally aligned maps to expansions, follows immediately from its description as a subcategory of the fibered category (over logarithmic schemes) of $\mathcal M^{\mathrm{log}}_{1,\alpha}(\mathbb P^N|H)$. To complete the result, we note that $
\mathcal{VZ}_{1,\alpha}(\mathbb P^N|H)$ has a strict map to $
\mathcal{VZ}_{1,\alpha}(\mathbb P^N|H)$, so its cone complex is a subcomplex of $\Sigma^{\mathrm{cen}}$. The fact that it must be contained in the subcomplex of well-spaced curves is immediate from Proposition~\ref{prop: well-spaced}.l
\end{proof}

\subsection{Indexing the strata} \label{subsection indexing strata} The dimension-$k$ cones in $\Sigma_{1,\alpha}(\PP^N|H,d)$ can be enumerated as follows. First, the cones in $\Sigma^{\operatorname{log}}$ are indexed by \textbf{combinatorial types} of tropical maps to $\RR_{\geq 0}$. Here a combinatorial type encodes all of the data of a tropical map, except for the edge lengths and precise vertex positions. To be more precise, a combinatorial type $\Delta$ consists of:
\begin{enumerate}
\item a finite graph;
\item genus, degree and marking assignments on the vertices
\item the data of which stratum of $\RR_{\geq 0}$ each vertex is mapped to;
\item integral slopes along the edges (both finite and infinite).
\end{enumerate}
The corresponding cone $\sigma \in \Sigma^{\operatorname{log}}$ is then given by the resulting moduli space of tropical maps, given by choices of edge lengths and vertex positions which produce a continuous tropical map. Given $\sigma\in \Sigma^{\operatorname{log}}$ we then produce all the cones in $\Sigma_{1,\alpha}(\PP^N|H,d)$ mapping to $\sigma$ by performing Steps 2--4 outlined above. This amounts to taking a particular subdivision of $\sigma$. Note that in this process, new cones are created which map into larger-dimensional cones of $\Sigma^{\operatorname{log}}$. The process for enumerating the codimension $k$ strata of $\VZ_{1,\alpha}(\PP^N|H,d)$ is therefore:
\begin{enumerate}
\item fix a combinatorial type $\Delta$;
\item perform the subdvidision of the resulting cone $\sigma \in \Sigma^{\operatorname{log}}$;
\item identify the dimension $k$ cones of that subdivision.
\end{enumerate}
In \S \ref{} we will perform this analysis in the case $k=1$.

\section{Rubber variants and virtual construction}

\subsection{Rubber variants} In our implementation of the relative theory, we will need the naturally ``rubber'' variant of the space of maps $
\mathcal{VZ}_{1,\alpha}(\mathbb P^N|H)$, where the entire curve is mapped into higher levels. 

Specifically, let $\mathbb P$ denote the projective bundle $\mathbb P(\mathcal O(1)\oplus\mathcal O)$ on $\mathbb P^{N-1}$. Equip this space with the logarithmic structure coming from the $0$ and $\infty$ sections of the bundle. Consider the moduli space $\mathcal{VZ}_{1,\alpha}(\mathbb P/\mathbb G_m)$ of logarithmic maps
\[
C\to \mathbb P[s]\to \mathbb P^{N-1},
\]
which factorize completely, and where automorphisms are considered over the space of maps to $\mathbb P^{N-1}$. Specifically, two maps that differ by a $\mathbb G_m$ translate in the fiber direction are considered equivalent. Total factorization, similar to the previous case, means that the map to the bundle and the map to the base $\mathbb P^{N-1}$ both factorize through possibly different singularities. 

\begin{thm}
The stack $\mathcal{VZ}_{1,\alpha}(\mathbb P/\mathbb G_m)$ is logarithmically smooth.
\end{thm}

It is convenient to work with the logarithmic multiplicative group and its torsors. Recall that the \textit{logarithmic multiplicative group} $\mathbb G_{\mathrm{log}}$ is the functor on logarithmic schemes whose value on a logarithmic scheme $S$ is the group of global sections $H^0(S,M_S^{\mathrm{gp}})$. This functor is representable only after a logarithmically \'etale cover. 

The bundle $\mathbb P\to \mathbb P^{N-1}$ gives rise to a $\mathbb G_m$-torsors by deleting the zero and infinity sections. Replacing these fibers by their $\mathbb G_{\mathrm{log}}$ compactifications, we obtain a non-representable functor on logarithmic schemes $\mathbb P_{\mathrm{log}}$ and a logarithmically \'etale modification
\[
\mathbb P\to \mathbb P_{\mathrm{log}}.
\]

{\bf Add reference to Wise--R '19 and Marcus--Wise '17 for the multiplicative group.}

\begin{proof}
Consider the stack over logarithmic schemes $\mathcal{VZ}_{1,\alpha}(\mathbb P_{\mathrm{log}})$ of stable logarithmic maps that factorize completely. Here, stability for the map coincides with stability for the projection to $\mathbb P^{N-1}$. The arguments in the previous section show that maps to $\mathbb P$ that factorize completely are unobstructed. Since this space is a logarithmically \'etale cover of $\mathcal{VZ}_{1,\alpha}(\mathbb P_{\mathrm{log}})$, logarithmic smoothness of the latter follows.

The logarithmic multiplicative group $\mathbb G_{\mathrm{log}}$ acts on $\mathcal{VZ}_{1,\alpha}(\mathbb P_{\mathrm{log}})$ by translation. Tautologically, $\mathcal{VZ}_{1,\alpha}(\mathbb P_{\mathrm{log}})$ is a $\mathbb G_{\mathrm{log}}$-torsor over the moduli problem of maps up to this $\mathbb G_{\mathrm{log}}$ translation. It is representable after a logarithmically \'etale modification by a family of nodal curves. It follows that the space  $\mathcal{VZ}_{1,\alpha}(\mathbb P_{\mathrm{log}}/\mathbb G_{\mathrm{log}})$ of maps up to translation that factorize completely is logarithmically smooth. 

To conclude the main result, we compare the stacks $\mathcal{VZ}_{1,\alpha}(\mathbb P_{\mathrm{log}}/\mathbb G_{\mathrm{log}})$ and  $\mathcal{VZ}_{1,\alpha}(\mathbb P/\mathbb G_m)$. Since a map to $\mathbb P$ gives rise immediately to a map to $\mathbb P_{\mathrm{log}}$, there is a morphism
\[
\mathcal{VZ}_{1,\alpha}(\mathbb P/\mathbb G_m)\to \mathcal{VZ}_{1,\alpha}(\mathbb P_{\mathrm{log}}/\mathbb G_{\mathrm{log}}).
\]
A straightforward application of the infinitesimal lifting criterion shows that this map is logarithmically \'etale, so the result follows.
\end{proof}

The main point in the proof above is that on the space of non-degenerate maps with smooth domains, the map from $\mathcal{VZ}^\circ_{1,\alpha}(\mathbb P)\to \mathcal{VZ}^\circ_{1,\alpha}(\mathbb P/\mathbb G_m)$ is a $\mathbb G_m$ torsor, and correspondingly on the compact spaces it is a $\mathbb G_{\mathrm{log}}$-torsor. The logarithmically \'etale modification guarantee representability, and turn this into a nodal curve fibration.

\subsection{Virtual construction for very ample pairs}
Now let $(X,Y)$ be a smooth pair with $Y$ very ample. The definition given in \S \ref{subsection factorisation} applies in this setting as well, producing a moduli space $\VZ_{1,\alpha}(X|Y,\beta)$ together with a morphism
\begin{equation*} \VZ_{1,\alpha}(X|Y,\beta) \to \ol\Mcal^{\operatorname{log}}_{1,\alpha}(X|Y,\beta)\end{equation*}
obtained by performing a logarithmic modification and then passing to a closed substack. This moduli space will not in general be logarithmically smooth, but we may equip it with a virtual class as follows. The complete linear system $|\OO_X(Y)|$ defines an embedding $X \hookrightarrow \PP^N$ with $Y=X\cap H$ for $H$ some hyperplane.

\begin{lemma} The following square is cartesian (in the category of ordinary stacks):
\bcd
\VZ_{1,\alpha}(X|Y,\beta) \ar[r] \ar[d] \ar[rd,phantom,"\square"] & \VZ_{1,\alpha}(\PP^N|H,d) \ar[d] \\
\VZ_{1,n}(X,\beta) \ar[r,"i"] & \VZ_{1,n}(\PP^N,d).
\ecd
\end{lemma}
\begin{proof} It is clear from the modular description that this square is cartesian in the category of fs log stacks. But the morphism $i$ is strict, which implies that the square is also cartesian in the category of coherent log stacks. Since fibre products in the latter category are compatible with the functor forgetting log structures, the claim follows.\end{proof}

Since $\VZ_{1,n}(\PP^N,d)$ is smooth and $\VZ_{1,n}(X,\beta)$ carries a natural virtual class \cite[Theorem 4.4.1]{RSPW} there is a diagonal pull-back morphism \cite{Ga} \cite[Appendix C]{BattistellaNabijou}, which we use to define the virtual class on the space of maps to $(X,Y)$:
\begin{equation*} \virt{\VZ_{1,\alpha}(X|Y,\beta)} := i_\Delta^! [\VZ_{1,\alpha}(\PP^N|H,d)]. \end{equation*}
Since $\VZ_{1,\alpha}(X|Y,\beta)$ comes equipped with evaluation maps and cotangent line classes, we immediately arrive at a definition of reduced Gromov--Witten invariants for the pair $(X,Y)$.


\part{The recursion formula and applications}\footnote{(Navid) From here on, when I write a relative space without any decoration, I mean a Kim space. We should establish this convention from the start} \noindent In this part we prove recursion relations on the moduli space of reduced stable maps (Theorem \ref{theorem recursion}). These are inspired by the constructions of Vakil and Gathmann, but go beyond these in several key respects, giving a full treatment of descendant invariants for very ample pairs. We apply the recursions to produce a quantum Lefschetz algorithm in genus one, reconstructing the absolute theory of $Y$ and the relative theory of $(X,Y)$ from the absolute theory of $X$. Along the way we prove a splitting axiom for the reduced relative theory; the form of this splitting axiom is nontrivial, insofar as it involves the introduction of additional tautological classes.

In \S\S \ref{Section Gathmann line bundle}--\ref{section reduced splitting axiom} we prove the recursion formula for $(\PP^N,H)$. In \S \ref{section recursion for general pair} we indicate how to obtain from this the recursion formula for an arbitrary very ample pair $(X,Y)$. Finally in \S \ref{section recursion algorithm} we describe the quantum Lefschetz algorithm.

\section{Gathmann's line bundle via tropical geometry}\label{Section Gathmann line bundle} Consider a reduced relative space $\VZ_{1,\alpha}(\PP^N|H,d)$. For each marking $x_k$ we consider the locus $\Dcal_{1,\alpha,k}(\PP^N|H,d) \subseteq \VZ_{1,\alpha}(\PP^N|H,d)$ where $x_k$ belongs to an internal component of the collapsed map. In this section, we use the log structure on $\VZ_{1,\alpha}(\PP^N|H,d)$ to construct a line bundle $\Lcal_k$ together with a section $s_k$ which vanishes precisely along $\Dcal_{1,\alpha,k}(\PP^N|H,d)$. We use the correspondence with tropical geometry to identify $\cchern_1(\Lcal_k)$ in terms of tautological classes on $\VZ_{1,\alpha}(\PP^N|H,d)$, and to compute the vanishing order of $s_k$ along the components of $\Dcal_{1,\alpha,k}(\PP^N|H,d)$. Combined with the relative splitting axioms established in the next section, we obtain a recursion relation inside $\VZ_{1,\alpha}(\PP^N|H,d)$

The pair $(\Lcal_k,s_k)$ is most naturally constructed on the moduli space $\MLog_{1,\alpha}(\PP^N|H,d)$ of non-expanded log maps; the corresponding pair on $\VZ_{1,\alpha}(\PP^N|H,d)$ will be obtained via pull-back. Consider therefore the tropicalisation $\Sigma^{\log}$ of $\MLog_{1,\alpha}(\PP^N|H,d)$. By tropicalising the universal family, we obtain a family of tropical maps
\bcd
\sqC \ar[d,"\pi"] \ar[r,"f"] & \RR_{\geq 0} \\
\Sigma^{\log} \ar[u, bend left=40pt, "x_k" left]
\ecd
where $x_k$ is the section which for every point $\lambda \in \Sigma^{\log}$ picks out the vertex of $\sqC_\lambda$ containing the leg $x_k$. The composition $f \circ x_k \colon \Sigma^{\log} \to \RR_{\geq 0}$ defines a piecewise-linear function on $\Sigma^{\log}$ whose preimage over the open cone $\RR_{>0}$ consists of those tropical maps where $x_k$ belongs to an internal component. This produces a section of the ghost sheaf on $\MLog_{1,\alpha}(\PP^N|H,d)$, which in the usual way induces a line bundle and section $(\Lcal_k,s_k)$ on the moduli space, and the tropical description above shows that the zero locus of $s_k$ is (set-theoretically) the locus where $x_k$ belongs to an internal component.

We now calculate $\cchern_1(\Lcal_k)$. Choose a family of log stable maps over $S$ and let $\mu \in \Gamma(S,\ol{M}_S)$ be the global section of the ghost sheaf constructed in the previous paragraph. This pulls back along $\pi$ to give a global section $\pi^\flat(\mu) \in \Gamma(C,\ol{M}_C)$.  Interpreted as a piecewise-linear function on the tropicalisation $\sqC$ with values in $\ol{M}_S$ \cite[Remark 7.3]{CavalieriChanUlirschWise}, this assigns $\mu$ to every vertex and has slope zero along every edge. By construction, the line bundle associated to this section is $\pi^\st \Lcal_k$. Consider on the other hand the generator $1 \in \N = \Gamma(\PP^N,\ol{M}_{\PP^N})$ with associated line bundle $\OO(H)$. The section $f^\flat(1) \in \Gamma(C,\ol{M}_C)$ has associated line bundle $f^\st\OO(H)$. If we let $v$ denote the vertex containing $x_k$, then by construction $f^\flat(1)$ assigns $\mu$ to $v$ and has slope $\alpha_k$ along the leg $x_k$. Thus if we consider the difference $f^\flat(1) - \pi^\flat(\mu)$ then this assigns $0$ to $v$ and still has slope $\alpha_k$ along $x_k$. Thus by \cite[Proposition 2.4.1]{RSPW} the corresponding line bundle restricted to $C_v$ is given by
\begin{equation*} \OO_{C_v} \left(\alpha_k x_k + \sum_e \mu_e x_e \right) \end{equation*}
where the sum is over the edges $e$ adjacent to $v$ and distinct from $x_k$. Thus we see that:
\begin{equation*} \left( f^\st\OO(H) \otimes \pi^\st \Lcal_k^{-1} \right) \big|_{C_v} = \OO_{C_v} \left(\alpha_k x_k + \sum_e \mu_e x_e \right).\end{equation*}
Since $x_k$ factors through $C_v$ we may pull back along $x_k$ to obtain
\begin{equation*} \Lcal_k = x_k^\st\pi^\st \Lcal_k = x_k^\st \OO_{C_k}(-\alpha_k x_k) \otimes x_k^\st f^\st\OO(H) = x_k^\st \OO_{C_k}(-\alpha_k x_k) \otimes \ev_k^\st \OO(H) \end{equation*}
and taking Chern classes gives:
\begin{equation*} \cchern_1(\Lcal_k) = \alpha_k \psi_k + \ev_k^\st H.\end{equation*}
This gives the construction of $(\Lcal_k,s_k)$ on $\MLog_{1,\alpha}(\PP^N|H,d)$; the construction on $\VZ_{1,\alpha}(\PP^N|H,d)$ is obtained by pull-back. The relevant piecewise-linear function is simply the composition:
\begin{equation*} \Sigma_{1,\alpha}(\PP^N|H,d) \to \Sigma^{\log} \to \RR_{\geq 0}.\end{equation*}
Notice in particular that $\psi_k$ should be interpreted as a \textbf{collapsed psi class} on $\VZ_{1,\alpha}(\PP^N|H,d)$, i.e. a psi class coming from the stabilised curve of the collapsed map. In this paper, all psi classes are collapsed unless stated otherwise. We immediately arrive at:

\begin{thm} \label{theorem recursion} We have the following relation in the Chow ring of $\VZ_{1,\alpha}(\PP^N|H,d)$:
\begin{equation}\label{equation recursion} (\alpha_k \psi_k + \ev_k^\st H) \cap [\VZ_{1,\alpha}(\PP^N|H,d)] = \sum_{\Dcal} \lambda_\Dcal [ \Dcal ].\end{equation}
The sum is over irreducible components $\Dcal$ of the divisor $\Dcal_{1,\alpha,k}(\PP^N|H,d) \subseteq \VZ_{1,\alpha}(\PP^N|H,d)$, and $\lambda_\Dcal$ is the vanishing order of $s_k$ along this component. \end{thm}

\begin{remark} This construction gives the natural logarithmic analogue of Gathmann's line bundle and section \cite[Construction 2.1]{Ga}. A benefit of the logarithmic approach is to make the computation of vanishing orders entirely combinatorial (see \S \ref{} below), circumventing the difficult technical calculation given by Gathmann. \end{remark}

\section{Recursive description of boundary strata: the relative splitting axiom}\label{section reduced splitting axiom}
In this section we provide an explicit description of the terms appearing on the right-hand side of \eqref{equation recursion} above. For each such term, we provide a combinatorial method to calculate the vanishing order $\lambda_\Dcal$, and a recursive description of the stratum $\Dcal$ in terms of fibre products of moduli spaces with smaller numerical data. The form which this recursive description takes is non-obvious, in the sense that additional tautological classes appear, which must be calculated. This constitutes the most technically difficult part of the paper. 


\subsection{Identifying the irreducible components of $\Dcal_{1,\alpha,k}(\PP^N|H,d)$} Here we explain the strategy for identifying the irreducible components of $\Dcal_{1,\alpha,k}(\PP^N|H,d)$ via tropical geometry.

\begin{lemma} \label{Lemma components are log divisors} Every irreducible component of $\Dcal_{1,\alpha,k}(\PP^N|H,d)$ is a codimension $1$ log stratum.\end{lemma}
\begin{proof} This is not hard to see: the locus where the log structure is trivial coincides precisely with the  locus where the source curve is smooth and not mapped inside $H$, and since $\VZ_{1,\alpha}(\PP^N|H,d)$ is log smooth this locus is open and dense \cite{Niziol}. By definition $\Dcal_{1,\alpha,k}(\PP^N|H,d)$ is contained in the  complement of this locus, which (by the very definition of logarithmic strata) can be written as a union of log strata of positive codimension. Since $\Dcal_{1,\alpha,k}(\PP^N|H,d)$ has codmension $1$, it must therefore be equal to a union of log divisors.\end{proof}
In \S \ref{subsection indexing strata} we discussed a procedure for enumerating the log strata of $\VZ_{1,\alpha}(\PP^N|H,d)$, using the cones of the tropicalisation $\Sigma_{1,\alpha}(\PP^N|H,d)$. To recap: every divisorial logarithmic stratum is obtained via a three-step process:
\begin{enumerate}
\item choose a combinatorial type $\Delta$ of a tropical map;
\item subdivide the corresponding tropical moduli space $\sigma$;
\item choose a ray in this subdivision.
\end{enumerate}
Notice that this process contains some redundancies: when we choose a ray in step (3) above, some of the edge lengths or vertex positions may get set to $0$. This induces a generisation of the intial combinatorial type $\Delta$, given by contracting appropriate edges of the dual graph and/or moving appropriate vertices from $\RR_{>0}$ to $0$. By the combinatorial type of a stratum in $\VZ_{1,\alpha}(\PP^N|H,d)$ we will always mean this generisation. This is independent of the choice of $\Delta$, and in fact we can (and will) always choose $\Delta$ to coincide with the generisation.

The log divisors contained in $\Dcal_{1,\alpha,k}(\PP^N|H,d)$ are precisely those whose  combinatorial types map the vertex of the dual graph containing $x_k$  into the interior $\RR_{>0} \subseteq \RR_{\geq 0}$. Thus via the above procedure, we are able to enumerate the irreducible components of $\Dcal_{1,\alpha,k}(\PP^N|H,d)$ in a combinatorial manner, and the combinatorial type $\Delta$ allows us to describe the general element of such a component. What we are still lacking is a recursive description of the components, which we need in order to compute integrals over them. This is the subject of the remainder of this section.

\subsection{Recursive description of the divisors: types $A,B$ and $C^+$} Choose an irreducible component $\Dcal \subseteq \Dcal_{1,\alpha,k}(\PP^N|H,d)$. By Lemma \ref{Lemma components are log divisors} this is a log divisor, and hence may be written as:
\begin{equation*} \Dcal = \widetilde\Dcal \cap \VZ_{1,\alpha}(\PP^N|H,d) \end{equation*}
for a unique log stratum $\widetilde\Dcal \subseteq \widetilde{\VZ}_{1,\alpha}(\PP^N|H,d)$. Since
\begin{equation*}\widetilde{\VZ}_{1,\alpha}(\PP^N|H,d) \to \ol\Mcal_{1,\alpha}(\PP^N|H,d)\end{equation*}
is a log modification, $\widetilde{\Dcal}$ is either exceptional or the proper transform of a log divisor. These cases correspond, respectively, to when the core is assigned zero degree or nonzero degree by the combinatorial type. To begin with we will focus on the latter case. Suppose therefore that $\widetilde{\Dcal}$ is the proper transform of a log divisor:
\begin{equation*} \Ecal \subseteq \ol\Mcal_{1,\alpha}(\PP^N|H,d). \end{equation*}
The birational map $\widetilde{\Dcal} \to \Ecal$ induces a morphism $\Dcal \to \Ecal$. In the remainder of this subsection, we will show how to interpret this morphism as a ``desingularisation of the main component''. Since $\Ecal$ admits a recursive description in terms of relative and rubber moduli spaces, this will allow us to compute integrals over $\Dcal$.

\begin{lemma}\label{lem:combs} Let $\Dcal \subseteq \Dcal_{1,\alpha,k}(\PP^N|H,d)$ be an irreducible component which contributes nontrivially to the recursion, and let $\Delta$ be the corresponding combinatorial type. Suppose that $\Delta$ assigns positive degree to the core. Then $\Delta$ takes one of the following forms:
\begin{figure}[h]
    \centering
    \begin{minipage}{0.3\textwidth}
        \centering
        \Yagraph
        \caption{$A$}
    \end{minipage}\hfill
    \begin{minipage}{0.3\textwidth}
        \centering
        \Ybgraph
        \caption{$B$}
    \end{minipage}\hfill
    \begin{minipage}{0.3\textwidth}
        \centering
        \Ycgraph
        \caption{$C^+$}
    \end{minipage}
\end{figure}

\noindent The terminology is due to Vakil, and the pictures in this case are very similar to \cite{Vre} (as we will see later, when the core is assigned zero degree the picture becomes \emph{very} different). Note that in these pictures we have deliberately omitted  the degree of each vertex, the expansion factors of the edges and the marked points (except for $x_k$). The omitted combinatorial data can be distributed arbitrarily, as long as:
\begin{enumerate}
\item the vertices $\sqC_1,\ldots,\sqC_r$ have positive degree;
\item the core has positive degree;
\item every vertex is stable;
\item the balancing condition is satisfied.
\end{enumerate}\end{lemma}

\begin{proof} Let $\sigma \in \Sigma$ be the cone corresponding to the stratum $\Ecal \subseteq \ol\Mcal_{1,\alpha}(\PP^N|H,d)$. By the discussion above $\Dcal$ corresponds to a ray in the subdivision of $\sigma$ given by imposing the central alignment condition. But since the core is assigned positive degree, both radii are equal to $0$ and the subdivision is trivial. We conclude that $\sigma=\RR_{\geq 0}$ (recall we are assuming that the initial and generised combinatorial types coincide). Since there must be at least one vertex mapped to higher level, we conclude that the tropical target $\widetilde\RR_{\geq 0}$ is obtained from $\RR_{\geq 0}$ by subdividing at a single point $\diamond \in \RR_{> 0}$.

In order to have $\sigma=\RR_{\geq 0}$, the dual graph $\Gamma$ must be a bipartite graph with vertices over $0$ and $\diamond$. The cases $A,B$ and $C^+$ enumerated above cover situations where there is a single vertex mapped to $\diamond$. We claim that if there is more than one such vertex, then the contribution to the recursion vanishes. To see this, we consider the stratum
\begin{equation*} \Ecal^{\text{\tiny{log}}} \subseteq \ol\Mcal^{\text{\tiny{log}}}_{1,\alpha}(\PP^N|H,d) \end{equation*}
to which $\Ecal$ maps under the collapsing morphism, and examine the composition $\Dcal \to \Ecal \to \Ecal^{\text{\tiny{log}}}$. If we let $\Fcal^{\text{\tiny{log}}}$ denote the intersection of $\Ecal^{\text{\tiny{log}}}$ with the main component of the moduli space, then we obtain a factorisation $\Dcal \to \Fcal^{\text{\tiny{log}}} \hookrightarrow \Ecal^{\text{\tiny{log}}}$. Since the moduli space is generically unobstructed along $\Fcal^{\text{\tiny{log}}}$, the codimension of $\Fcal^{\text{\tiny{log}}}$ is given by the dimension of the associated cone in the tropicalisation $\Sigma^{\text{\tiny{log}}}$. If there is more than one vertex mapped to higher level, then this cone has dimension $\geq 2$. Therefore the map $\Dcal \to \Fcal^{\text{\tiny{log}}}$ has positive-dimensional fibres, and since all insertions are pulled back from the latter space the contribution vanishes by the projection formula.\end{proof}

\begin{remark} The difference in dimensions between $\Dcal$ and $\Fcal^{\text{\tiny{log}}}$ (or equivalently, the difference in virtual dimensions between $\Ecal$ and $\Ecal^{\text{\tiny{log}}}$) may be interpreted as the difference in dimensions between moduli spaces of \emph{disconnected} rubber and their images under the collapsing morphisms (see \ref{} below). \end{remark}

We now investigate the three types $A,B,C^+$ separately, giving a recursive description of the boundary divisor in each case. For the remainder of this subsection, therefore, we fix a one-dimensional cone $\tau \in \Sigma_{1,\alpha}(\PP^N|H,d)$, let $\Dcal \subseteq \VZ_{1,\alpha}(\PP^N|H,d)$ be the associated log divisor, and assume that $\Dcal$ is contained in $\Dcal_{1,\alpha,k}(\PP^N|H,d)$ and is of type $A,B$ or $C^+$. We let $\Ecal \subseteq \ol\Mcal_{1,\alpha}(\PP^N|H,d)$ be the log stratum into which $\Dcal$ is mapped; this is indexed by a cone $\sigma \in \Sigma$ corresponding to a combinatorial type $\Delta$, and $\sigma$ is one-dimensional since we are restricting to the type $A,B,C^+$ cases.


\subsubsection{Type $A$}\label{subsubsection type A} Suppose $\Delta$ is of type $A$. Then $\Ecal$ admits a finite splitting morphism onto the fibre product:
\begin{equation*} \Ecal \to \left( \ol\Mcal_{1,\alpha^{(1)}\cup(m_1)}(\PP^N|H,d_1) \times \prod_{i=2}^r \ol\Mcal_{0,\alpha^{(i)}\cup(m_i)}(\PP^N|H,d_i) \right) \times_{H^r} \ol\Mcal^{\leftrightarrow}_{0,\alpha^{(0)}\cup (-m_1,\ldots,-m_r)}(P|H_0+H_\infty,d_0).\end{equation*}

\begin{lemma} \label{Lemma type A gluing} $\Dcal$ admits a natural splitting morphism
\begin{equation*}\Dcal \xrightarrow{\rho} \left(\VZ_{1,\alpha^{(1)}\cup(m_1)}(\PP^N|H,d_1)\times\prod_{i=2}^r \ol\Mcal_{0,\alpha^{(i)}\cup(m_i)}(\PP^N|H,d_i)\right) \times_{H^r} \ol\Mcal^{\leftrightarrow}_{0,\alpha^{(0)}\cup (-m_1,\ldots,-m_r)}(P|H_0+H_\infty,d_0)\end{equation*}
such that the map $\Dcal \to \Ecal$ covers the map on fibre products obtained by desingularising the main component of the factor corresponding to $\sqC_1$.\end{lemma}

\begin{proof}
The choices involved in gluing log maps to expansions (which account for the degrees of the splitting morphisms above) is not affected by the introduction of alignments, so for the purposes of this proof we may act as if these splitting morphisms are the identity maps.

Note that the map $\Dcal\to\Ecal$ is an isomorphism away from the blown-up locus, which is contained inside the locus where the core is contracted. Given an element of $\Dcal$ we can split it along the nodes $q_1,\ldots,q_r$ as in \cite{}. It is then clear that $\sqC_1$ is aligned. We claim that $\sqC$ satisfies the factorisation property if and only if $\sqC_1$ does. This immediately implies the lemma.

On $\sqC$ there are associated contraction radii $\delta_1,\delta_2$. Let $\delta \in \{\delta_1,\delta_2\}$. By Lemma \ref{type A radius lemma} below, we have $\lambda(\sqC^\prime) > \delta$ for any component $\sqC^\prime$ of $\sqC_0$. This means that $\sqC_0,\sqC_2,\sqC_3,\ldots,\sqC_r$ lie outside both contraction radii, and so the aligned curve $\sqC$ satisfies the factorisation condition if and only if $\sqC_1$ does. \end{proof}

\begin{lemma}\label{type A radius lemma} $\lambda(\sqC^\prime) > \delta$ for any component $\sqC^\prime$ of $\sqC_0$ (with notation as in the proof above).\end{lemma}
\begin{proof} The basic point is that $\Dcal$ is obtained as the disjoint union of the locally closed logarithmic strata contained in the closure of the locally closed stratum on which all of the $\sqC_i$ are irreducible. If we look at one of these boundary strata, then by definition the associated cone $\sigma \in \Sigma_{1,\alpha}(\PP^N|H,d)$ contains a ray $\tau$ corresponding to the stratum where all the $\sqC_i$ are irreducible; this amounts to setting all edge lengths other than $e_1,\ldots,e_r$ to zero. If $f_1,\ldots,f_l$ are some collection of additional edge lengths in $\sigma$ (corresponding to internal nodes in degenerations of the $\sqC_i$) then since $\sigma$ is adjacent to $\tau$ we must have (by construction of the subdivision) $f_1+\ldots+f_l < e_j$ for all $j\in\{1,\ldots,r\}$, since $f_1=\ldots=f_l=0$ and $e_j \neq 0$ on $\tau$. In particular, if $\sqC_1$ is degenerate and if $\delta$ denotes the minimal distance from the core to a non-contracted vertex of $\sqC_1$ (which certainly exists since $\sqC_1$ has positive degree) then $\delta < e_1 \leq \lambda(\sqC^\prime)$, as claimed.\end{proof}

Lemma \ref{Lemma type A gluing} provides a means to calculate integrals over the class $\lambda_\Dcal[\Dcal]$ appearing on the right-hand side of Theorem \ref{theorem recursion}, provided that we can calculate $\lambda_\Dcal$ and the degree of the splitting morphism. Simple closed formulae for these are given in \S \ref{subsubsection splitting degree} below.

\subsubsection{Type $B$}
Now suppose $\Delta$ is of type $B$. Note that in this case, it is impossible for the core to be contracted. Hence $\Ecal$ is disjoint from the blown-up locus, and the map $\Dcal \to \Ecal$ is an isomorphism. We thus obtain the following description, entirely in terms of genus zero data:
\begin{lemma} $\Dcal$ admits a finite splitting morphism:
\begin{equation*} \Dcal \xrightarrow{\rho} \left(\ol\Mcal_{0,\alpha^{(1)}\cup(m_1,m_2)}(\PP^N|H,d_1)\times\prod_{i=3}^r \ol\Mcal_{0,\alpha^{(i)}\cup(m_i)}(\PP^N|H,d_i)\right) \times_{H^r} \ol\Mcal^{\leftrightarrow}_{0,\alpha^{(0)}\cup(-m_1,\ldots,-m_r)}(P|H_0+H_\infty,d_0).\end{equation*}\end{lemma}

\subsubsection{Type $C^+$} \label{subsubsection type C+} Finally suppose that $\Delta$ is of type $C^+$. As before we have a finite morphism:
\begin{equation*} \Ecal \to  \left( \prod_{i=1}^r \ol\Mcal_{0,\alpha^{(i)}\cup(m_i)}(\PP^N|H,d_i) \right) \times_{H^r} \ol\Mcal^{\leftrightarrow}_{1,\alpha^{(0)}\cup(-m_1,\ldots,-m_r)}(P|H_0+H_\infty,d_0). \end{equation*}
The same arguments as in \S \ref{subsubsection type A} then apply to give:
\begin{lemma} $\Dcal$ admits a finite splitting morphism
\begin{equation*}\Dcal \xrightarrow{\rho} \left( \prod_{i=1}^r \ol\Mcal_{0,\alpha^{(i)}\cup(m_i)}(\PP^N|H,d_i) \right) \times_{H^r} \VZ^{\leftrightarrow}_{1,\alpha^{(0)}\cup(-m_1,\ldots,-m_r)}(P|H_0+H_\infty,d_0)\end{equation*}
such that $\Dcal \to \Ecal$ corresponds to the obvious map on fibre products.\end{lemma}
Here the first factor in the fibre product is the logarithmic blow-up of the moduli space of rubber maps constructed in \S \ref{}. We will compute integrals over such spaces as part of the recursion (see \S \ref{section recursion algorithm}).\footnote{(Navid) Say something about reduced rubber for target spaces here?}

\subsubsection{Splitting degree and vanishing order} \label{subsubsection splitting degree} In each of the subsections \ref{subsubsection type A}--\ref{subsubsection type C+} above, we obtained a finite splitting morphism $\rho$ from $\Dcal$ to a fibre product of moduli spaces with smaller numerical data. Here we describe the degree of $\rho$ and calculate the vanishing order $\lambda_\Dcal$ of the section $s_k$ constructed in \S \ref{Section Gathmann line bundle}. This gives a complete recursive description of the terms appearing on the right-hand side of Theorem \ref{theorem recursion} which are of type $A,B$ or $C^+$.

\begin{lemma}\label{lem:saturation} The degree of $\rho$ is given by:
\begin{equation*} \label{degree of gluing} \dfrac{\prod_{i=1}^r m_i}{\lcm(m_1,\ldots,m_r)}. \end{equation*}\end{lemma}
\begin{proof} This calculation is by now standard, and is given by examining the base order of the corresponding family of targets. See, for instance \cite{ChenDegeneration}, \cite{ACGSDecomposition}.
\end{proof}

\begin{lemma}\label{lemma vanishing order} $\lambda_\Dcal = \lcm(m_1,\ldots,m_r).$ \end{lemma}
\begin{proof} Let $\tau \in \Sigma_{1,\alpha}(\PP^N|H,d)$ be the cone corresponding to $\Dcal$ and let
\begin{equation*} \varphi \colon \Sigma_{1,\alpha}(\PP^N|H,d) \to \Sigma^{\log} \to \RR_{\geq 0} \end{equation*}
be the piecewise-linear function constructed in \S \ref{Section Gathmann line bundle}. It follows from the tropical description that $\lambda_\Dcal$ is equal to the index of the map of integral cones $\tau \to \RR_{\geq 0}$ obtained by restricting $\varphi$. By tropical continuity we have $\tau \subseteq \RR_{\geq 0}^r$ with integral generator:
\begin{equation*} w = \left( \dfrac{\lcm(m_1,\ldots,m_r)}{m_1},\ldots,\dfrac{\lcm(m_1,\ldots,m_r)}{m_r} \right).\end{equation*}
The map $\tau \to \RR_{\geq 0}$ is given by projecting onto the $i$th factor and then multiplying by $m_i$ (this is independent of $i$ by continuity of the tropical map). Thus we conclude that the index is equal to $\lcm(m_1,\ldots,m_r)$ as claimed.\end{proof}

\begin{remark} A similar line bundle and section is considered in \cite{Ga} in the genus zero setting. The computation of $\lambda_{\Dcal}$ is the most difficult technical part of that paper. One of our contributions here is to observe that the tropical description reduces this computation to simple combinatorics.\end{remark}

In summary, then: the contribution of each term $\lambda_\Dcal [\Dcal]$ is given by integrating over the appropriate fibre product and then multiplying the result by $\lambda_\Dcal \cdot \deg\rho = \prod_{i=1}^r m_i$.

\subsection{Recursive description of the divisors: type IIIb} The treatment of the type I, II and IIIa cases above is a fairly straightforward extension of the ideas of Vakil and Gathmann (albeit updated with logarithmic and tropical technology, which streamlines the presentation and simplifies the proofs). The remaining type IIIb case, however, is entirely new. In this section we provide a recursive description of these boundary divisors. This constitutes the technical heart of the paper.

\subsubsection{Possible combinatorial types} Recall that a boundary divisor $\Dcal \subseteq \Dcal_{1,\alpha,k}(\PP^N|H,d)$ is said to be of type IIIb if the core is assigned degree zero by the combinatorial type (this is the case which was ignored in Lemma \ref{lem:combs} and the subsequent discussion). In this case, the combinatorial type is not as tightly constrained as before; the target may expand multiple times, and as such the combinatorial type need not be a bipartite graph with a single interior vertex. Nevertheless, we have the following lemma:
\begin{lemma} 
Let $\Dcal$ be a locus of type IIIb and let $\Delta$ be the corresponding combinatorial type. Then for every nonzero vertex $\diamond$ of the expanded tropical target, the fibre of the tropical map over $\diamond$ contains exactly one stable vertex, which lies on the radius.
\end{lemma}
\begin{proof}
By stability, we must always have at least one stable vertex lying over each  vertex of the target. If there was more than one, then (since both the insertions and the factorisation condition are pulled back from the moduli space of maps without expansions) we can show, as in the proof of Lemma \ref{lem:combs}, that the contribution vanishes.

Furthermore, for each vertex of the target there must always be at least one vertex in the preimage which lies on the radius, since otherwise there would be at least two independent tropical parameters, and so $\Dcal$ would not be a divisor.

A priori it could be that the stable vertex is not amongst those vertices lying on the radius. But in this case the corresponding locus $\Fcal$ in the moduli space of unexpanded maps has larger codimension (for the same reason as before), and so the contribution vanishes by the projection formula. 
\end{proof}




\subsubsection{Recursive description of $\Ecal$}
Just as before, we let $\tau \in \Sigma$ be the ray corresponding to $\Dcal \subseteq \VZ_{1,\alpha}(\PP^N|H,d)$, $\sigma \in \Sigma^{\Kim}$ be the minimal cone containing $\tau$ and $\Ecal \subseteq \ol\Mcal_{1,\alpha}(\PP^N|H,d)$ be the stratum corresponding to $\sigma$, so that we have a map $\Dcal \to \Ecal$ (and $\Ecal$ is the minimal stratum through which the map from $\Dcal$ factors). Recall that we also have a unique stratum $\widetilde\Dcal \subseteq \widetilde\VZ_{1,\alpha}(\PP^N|H,d)$ such that $\Dcal = \widetilde\Dcal \cap \VZ_{1,\alpha}(\PP^N|H,d)$. Note that the cone $\sigma$ has dimension equal to the number of levels of the target expansion in the combinatorial type, which in this case could possibly be larger than $1$. 

We can describe the locus $\Ecal$ in a simple recursive manner. We write $\Delta_0$ for the collection of stable vertices of the combinatorial type which are mapped to $0$ by the tropical map, and $\Delta_{>0}$ for the stable vertices which are mapped to $\RR_{>0}$. For each vertex $v$ there is associated discrete data $\Gamma_v$, which defines a moduli space of relative stable maps if $v \in \Delta_0$ and a moduli space of rubber maps if $v \in \Delta_{>0}$. Then $\Ecal$ admits a finite splitting morphism onto the fibre product
\begin{equation}\label{IIIa fibre product} \Ecal \xrightarrow{\rho} \left( \prod_{v \in \Delta_0} \ol\Mcal_{\Gamma_v}(\PP^N|H) \times \prod_{v \in \Delta_{>0}} \ol\Mcal^{\leftrightarrow}_{\Gamma_v}(P|H_0+H_\infty) \right) \times_{H^{2\epsilon}} H^{\epsilon} \end{equation}
where $\epsilon$ is the number of edges of the (stabilised) dual graph, and the fibre product is taken over the appropriate evaluation maps. Thus, tautological integrals over $\Ecal$ are completely determined in terms of integrals over moduli spaces with ``smaller'' combinatorial data.

Notice that the actual dimension of $\Ecal$ will typically be larger than that of $\Dcal$, since $\Ecal$ is contained inside the excess components of the moduli space. We will explain how to compute integrals over $\Dcal$ in two steps:
\begin{enumerate}
\item we explain how to construct $\widetilde\Dcal$ from $\Ecal$ in \S \ref{};
\item we explain how to express the class of $\Dcal \subseteq \widetilde\Dcal$ in terms of tautological classes in \S \ref{}.
\end{enumerate}

\begin{remark}[Splitting degree and vanishing order, revisited] Similarly to \S \ref{subsubsection splitting degree} above, we can calculate the degree of $\rho$ and the vanishing order $\lambda_{\Dcal}$ of the section $s_k$ along $\Dcal$. The degree of $\rho$ is given exactly as in Lemma \ref{lem:saturation}, where now we obtain one such factor for each bounded edge of the tropical target. The vanishing order $\lambda_{\Dcal}$ is slightly more delicate, but may be computed as the index of an explicit morphism of one-dimensional lattices, as explained in the proof of Lemma \ref{lemma vanishing order}.\end{remark}


\subsubsection{$\widetilde\Dcal$ from $\Ecal$} Recall that the map $\widetilde\Dcal \to \Ecal$ is obtained by restricting the logarithmic blow-up
\begin{equation*} \widetilde\VZ_{1,\alpha}(\PP^N|H,d) \to \ol\Mcal_{1,\alpha}(\PP^N|H,d)\end{equation*}
to the locus $\widetilde\Dcal$. In toric geometry, the method for studying this restriction is as follows. The fan of $\widetilde\Dcal$ (respectively, $\Ecal$) is obtained by projecting the star of the ray $\tau$ (respectively, the cone $\sigma$) onto the quotient lattice. The map $\widetilde\Dcal \to \Ecal$ is then a toric morphism, induced by the obvious map of fans. In order to describe this morphism, we first subdivide the fan of $\Ecal$ by taking the images of cones in the fan of $\widetilde\Dcal$. This induces a toric blow-up of $\Ecal$ which we denote by $\widetilde\Ecal$. The morphism $\widetilde\Dcal \to \Ecal$ naturally factors through a morphism $\widetilde\Dcal \to \widetilde\Ecal$ which is now flat (in contrast to $\widetilde\Dcal \to \Ecal$), and now $\Dcal$ can be described as a compactification of a torus bundle over $\widetilde\Ecal$, with strata determined by the combinatorics of the fan.

Returning to the logarithmic setting, the first step is to define appropriate logarithmic structures on $\widetilde\Dcal$ and $\Ecal$. We equip $\widetilde\Dcal$ with the logarithmic structure obtained by first restricting the logarithmic structure on $\widetilde\VZ_{1,\alpha}(\PP^N|H,d)$ and then quotienting by the normal direction to $\widetilde\Dcal$. By the same process we may also define a logarithmic structure on $\Ecal$, and this coincides with the pull-back along $\rho$ of the product logarithmic structure on the fibre product in \eqref{IIIa fibre product}. The map $\widetilde\Dcal \to \Ecal$ is then naturally a logarithmic map.

The next step is to construct $\widetilde\Ecal$ as a logarithmic modification of $\Ecal$, and to do this we need to understand the tropicalisation of the map $\widetilde\Dcal \to \Ecal$. In fact, it is more convenient to consider the \emph{extended tropicalisation} \cite{}. We begin by fixing notation. Associated to the combinatorial type $\Delta$, we have stable vertices $\sqC_0,\sqC_1,\ldots,\sqC_r$ with $\sqC_0$ corresponding to the core and $\sqC_1,\ldots,\sqC_r$ corresponding to stable vertices which lie on the raidus (see Lemma \ref{} above). For $i \in \{1,\ldots,r\}$ let $e_i$ denote the sum of the edge lengths separating $\sqC_i$ from $\sqC_0$. These form co-ordinates on the cone $\sigma$, and in these co-ordinates $\tau$ is the ray given by the equations $e_1=\ldots=e_r$. Moreover, the $e_i$ define functions on any cone in the star of $\sigma$.

Now consider the extended tropicalisation of $\widetilde\Dcal$. This is obtained inside the extended tropicalisation of $\widetilde\VZ_{1,\alpha}(\PP^N|H,d)$ by intersecting the star of $\tau$ with the locus $e_1=\ldots=e_r=\infty$. On the other hand, the extended tropicalisation of $\Ecal$ is obtained inside the extended tropicalisation of $\ol\Mcal_{1,\alpha}(\PP^N|H,d)$ by intersecting the star of $\sigma$ with the same locus. It is therefore clear that the extended tropicalisation of $\widetilde\Dcal$ is a subdivision of that of $\Ecal$. Moreover we claim that we can describe this subdivision explicitly.

Consider any cone $\rho$ in the star of $\tau$. This is defined by a number of equalities and inequalities involving edge lengths coming from the combinatorial type on $\rho$. Amongst these edge lengths are $e_1,\ldots,e_r$ which corresponding to the separating nodes which persist on all of $\widetilde\Dcal$; there are also other edge lengths, corresponding to nodes over $\rho$ which are smoothed out on $\tau$. We first deal with the (in)equalities relating distances of vertices obtained by degenerating $\sqC_1,\ldots,\sqC_r$. By the definition of a well-spaced tropical map, we see that these are all of the form
\begin{equation*} f + e_i = g + e_j \end{equation*}
where $f$ and $g$ are positive linear combinations of additional edge length parameters which are set to zero in $\tau$. We may now pass to the subcone of $\rho$ obtained by imposing the additional equations $e_1=\ldots=e_r$. Note that this is \emph{not} a subcone of $\tau$ in general, since there are additional edge lengths which must be set to zero if one wishes to recover $\tau$. The additional equations mean that we can rewrite our previous (in)equalities in the form
\begin{equation*} f = g \end{equation*}
and this is now well-defined in the limit $e_1=\ldots=e_r=\infty$. We thus obtain a cone in the subdivision of the extended tropicalisation of $\Ecal$, and all such cones are obtained in this manner.

Taking this subdivision amounts to the following modular blow-up of the fibre product of the moduli spaces attached to the vertices $\sqC_1,\ldots,\sqC_r$: given a logarithmic map with splitting nodes $q_1,\ldots,q_r$ as specified by the shape of $\Delta$, we glue together the $r$ vertices of the tropicalisation which carry the splitting nodes, and declare the resulting special vertex to be the root of the resulting tree. We then impose a central alignment condition, entirely analogous to the genus one construction, where the special root takes on the role previously occupied by the core.

It remains to deal with the (in)equalities relating distances of vertices obtained by degenerating $\sqC_0$. These are not affected at all in the limit $e_1=\ldots=e_r=\infty$ and so the effect is to replace the factor of $\Ecal$ attached to the vertex $\sqC_0$
\begin{equation*} \ol\Mcal^{\leftrightarrow}_{\Gamma_{\tiny{\sqC_0}}}(P|H_0+H_\infty) \end{equation*}
with the radially aligned rubber space:
\begin{equation*} \VZ^{\leftrightarrow}_{\Gamma_{\tiny{\sqC_0}}}(P|H_0+H_\infty).\end{equation*}
The factorisation condition here is accounting for the ``fibre-direction'' factorisation; the ``base-direction'' factorisation has not entered the picture yet, and will appear when we come to describe $\Dcal$.

In this way, we obtain a logarithmic modification $\tilde\Ecal \to \Ecal$. By definition the map $\tilde\Dcal \to \Ecal$ is a compactification of a torus bundle over $\Ecal$, whose strata can be read off from the tropicalisation of $\tilde\Dcal$.

[LUCA SAYS MORE HERE, ABOUT HOW IT'S GENERICALLY A PROJECTIVE BUNDLE.]

\subsubsection{$\Dcal$ from $\widetilde\Dcal$} On $\widetilde\Dcal$ there is a natural vector bundle arising from the modular desription of $\widetilde\VZ_{1,\alpha}(\PP^N|H,d)$, namely:
\begin{equation*} F = \OO(\delta)^{\oplus r}. \end{equation*}
We claim that the fibre of $F$ over a moduli point can be naturally identified with
\begin{equation*} \left( \bigoplus_{i=1}^r \TT_{p_i} C_i^\prime \right) \otimes \TT_{C_0} \end{equation*}
where $C_i^\prime$ is a component of $C_i$ which lies on the radius, $p_i$ is the adjacent node which points towards the core, and $\TT_{C_0}$ is the universal tangent line bundle on the radially aligned rubber space.

For let us choose a component $C_i^\prime$ of the curve which lies on the radius. Then $\OO(\delta)$ can be expressed as a tensor product of tangent line bundles along the path connecting $C_i^\prime$ to the core. However, this lage product of line bundles is telescoping: any vertex which appears in the middle of the path will have an incoming and outgoing node, and since the vertex corresponds to a smooth rational curve these tangent spaces are naturally dual to each other \cite{VZ}. Thus these terms cancel, and product collapses into a contribution $\TT_{p_i}C_i^\prime$ from the left-hand side and a contribution from the right-hand side. Note that the final term on the right hand side is not given by $\TT_{q_i} C_0$ (where $q_i$ is the appropriate splitting node) but rather the tangent space to the core (which could be further away if $C_0$ has degenerated). Thus we obtain the universal tangent line bundle $\TT_{C_0}$ as the contribution to the right-hand side.

[EXAMPLE]

Note that $C_i^\prime$ is \emph{a} component of $C_i$ which lies on the radius. There may be more than one, but if this is the case then their tangent spaces are naturally identified by the logarithmic structure. Thus we obtain a natural morphism
\begin{equation*} \sum \operatorname{d}\!f \colon F(-\TT_{C_0}) \to \ev_q^\st \TT_H \end{equation*}
given by summing the derivatives at \emph{all} the inward-pointing nodes on the radius. Now, over $\widetilde\Dcal$ there is a natural section of the projective bundle $\PP(F) \to \widetilde\Dcal$ (obtained by projectivising the diagonal embedding). If we therefore lift the above morphism to the projectivisation, we obtain a composition: 
\begin{equation*} \OO_{\PP(F)}(-1) \to \pi^\st F \to \pi^\st(\ev_q^\st \TT_H (\TT_{C_0})).\end{equation*}
The vanishing locus of this section is given by an Euler class calculation on $\PP(F)$. In order to obtain $\Dcal$ we must then intersect this vanishing locus with $\OO_{\PP(F)}(1)$ (which represents the natural section $\widetilde\Dcal \to \PP(F)$.

Finally, we note that the class of $F$ is computable in terms of data pulled back from the base. To be more precise, we may consider the following bundle on $\Ecal$:
\begin{equation*} E = \bigoplus_{i=1}^r \TT_{q_i}C_i \otimes \TT_{q_i}C_0. \end{equation*}
We may then identify $F$ with the pullback of $E$ twisted by the exceptional loci. (This agrees with the more classical-minded description of $F$ as a bundle which resolves an otherwise non-transverse section of $E$.) We illustrate this in an example, which brings out the beauty of the interplay between logarithmic moduli and blow-ups.

[EXAMPLE]



\section{Recursion for general $(X,Y)$}\label{section recursion for general pair}

\section{Quantum Lefschetz algorithm}\label{section recursion algorithm}
Consider a smooth pair $(X,Y)$ with $Y$ very ample, and let $P$ be the projective bundle $P=\PP_Y(\operatorname{N}_{Y|X} \oplus\OO_Y)$. We assume that the genus zero and reduced genus one Gromov--Witten theories of $X$ are known. From this starting data, we will apply our recursion formula to compute:
\begin{enumerate}
\item the genus one \textbf{reduced restricted absolute Gromov--Witten theory} of $Y$;
\item the genus one \textbf{reduced relative Gromov--Witten theory} of the pair $(X,Y)$;
\item the genus one \textbf{reduced rubber theory} of $P$.
\end{enumerate}

\subsection{Reduced absolute, relative and rubber invariants} To be precise: by a reduced invariant of $Y$ we mean an integral over $\VZ_{1,n}(Y,\beta)$ of products of pullbacks of evaluation and psi classes along morphisms which forget a subset $S$ of the marked points (taking $S=\emptyset$ gives the ordinary evaluation and psi classes). Here the evaluation maps are viewed as mapping into $X$ (hence the adjective ``restricted''). Reduced relative invariants of $(X,Y)$ are defined in the same way, except now the forgetful morphism maps into a space of absolute maps:
\begin{equation*} \fgt_S \colon \VZ_{1,\alpha}(X|Y,\beta) \to \VZ_{1,m-\#S}(X,\beta).\end{equation*}
In particular, all the psi classes which we consider are \emph{collapsed psi classes}, meaning that they are relative cotangent line classes for the corresponding collapsed stable map. Note that, unlike in the absolute case, in the relative case it may well be the case that the entire insertion is pulled back along a single forgetful map $\fgt_S$. The reduced rubber invariants of $P$ are defined similarly (again using collapsed psi classes).

The systems of invariants defined above are equivalent to the classical systems of invariants (which do not use forgetful morphisms) by well-known topological recursion relations.

\subsection{Fictitious and true markings} The recursion procedure is rather delicate. Roughly speaking, we will induct on the degree (meaning $Y\cdot\beta$), number of marked points and total tangency. To get the correct notion of number of markings and total tangency in  the relative setting, we introduce the concept of \textbf{fictitious markings}. Consider a moduli space $\VZ_{1,\alpha}(X|Y,\beta)$ of reduced relative stable maps and a corresponding integrand $\gamma$. We let $F \subseteq \{1,\ldots,m\}$ be the maximal subset of marked points such that:
\begin{enumerate}
\item $\alpha_i = 1$ for all $i \in F$;
\item the entire integrand $\gamma$ is pulled back along $\fgt_F$.
\end{enumerate}
This subset is uniquely determined, and its elements are referred to as \textbf{fictitious markings}. Markings which are not fictitious are referred to as \textbf{true}. When inducting on relative invariants we will always be interested in the number of true markings (as opposed to the total number of markings) and the true tangency
\begin{equation*} \sum_{i \not\in F} \alpha_i \leq d=Y\cdot \beta \end{equation*}
as opposed to the total tangency, which is always $d$. This formalises the idea that relative invariants with non-maximal tangency $t<d$ can be obtained by adding $d-t$ fictitious markings of tangency $1$; see \cite[Lemma 1.15(i)]{Ga}.

\subsection{Structure of the recursion} Given the genus-zero Gromov--Witten theory of $X$, the arguments of \cite{Ga} give an effective algorithm to reconstruct the genus-zero theories of $Y$ and $(X,Y)$; moreover, the genus-zero rubber theory of $P$ is identical to the genus-zero theory of $Y$ \cite{GathmannThesis}. Thus we may assume that all genus-zero data is known. We assume in addition that we know the genus one reduced theory of $X$. The structure of the recursion is then as follows:

\begin{algorithm}
\DontPrintSemicolon
\For{$d \geq 0$}{
\For{$n \geq 0$}{
\For{$t \geq 0$\medskip}{
\textbf{\, Step 1: } Compute forgetful relative invariants of $(X,Y)$ (degree~$d$, $n+1$ true markings, true tangency $t$); see below.
}
\medskip \textbf{Step 2: } Compute absolute invariants of $Y$ (degree $d$, $n$ markings).\;
\For{$t \geq 0$\medskip}{
\textbf{\, Step 3: } Compute relative invariants of $(X,Y)$ (degree $d$, $n$ true markings, true tangency $t$).\;
}
}
\For{$n \geq 0$\medskip}{
\For{$m \geq 0$\medskip}{
\textbf{\, Step 4: } Compute rubber invariants of $P$ (degree $d$, $n$ relative markings, $m$ non-relative markings).
}
}
}
\end{algorithm}
\noindent Although the loops involving $d, n$ and $m$ have infinite length, in order to compute any single invariant it is only necessarily to iterate the preceding loops for a finite amount of time. A \textbf{forgetful relative invariant} of $(X,Y)$ is by definition a reduced relative invariant with a marked point $x_0$ such that all of the insertions are pulled back along $\fgt_{x_0}$. In our recursion, we first deal with this special class of relative invariants (with $n+1$ true markings), before computing the absolute invariants (with $n$ markings) and then returning to compute all of the relative invariants (with $n$ true markings). This need to treat separately a particular subclass of the relative invariants is an inescapable feature of the genus one recursion.

The base terms of the recursion all have $d=0$ and as such are easy to compute: the relative invariants of $(X,Y)$ are nothing but absolute invariants of $X$, the absolute invariants of $Y$ are given by obstruction bundle integrals over Deligne--Mumford spaces, and the  rubber invariants of $P$ are also given by integrals over Deligne--Mumford spaces, using the formula for the double ramification cycle \cite{Hain,JPPZ} in terms of tautological classes. We will now explain how to perform each of the four inductive steps outlined above.

\begin{notation}Given tuples $\mathbf{a}=(a_1,\ldots,a_n)$ and $\mathbf{b}=(b_1,\ldots,b_n)$, we say that $\mathbf{a}<\mathbf{b}$ if there exists an $i \in \{1,\ldots,n\}$ such that $a_j = b_j$ for $j < i$ and $a_i < b_i$.\end{notation}

\subsection*{Step 1} Suppose we are given a forgetful relative invariant to compute. That is, we have a relative space $\VZ_{1,\alpha}(X|Y,\beta)$ of degree $d$, with $n+1$ true markings and true tangency $t$, and a marking $x_0$ such that the insertion $\gamma$ is pulled back along $\fgt_{x_0}$. We assume inductively that every absolute, relative and rubber invariant with $(d^\prime,n^\prime) < (d,n)$ is known, and also that every forgetful relative invariant with $(d^\prime,n^\prime,t^\prime) < (d,n+1,t)$ is known. Choose a true marking $x_1$ with $\alpha_1 \geq 1$ and consider the space:
\begin{equation*} \VZ_{1,(\alpha-e_1) \cup (1)}(X|Y,\beta). \end{equation*}
Denote the newly-introduced marking by $y$ and consider the integrand $\tilde\gamma$ obtained from $\gamma$ by introducing $\fgt_y^\st$ everywhere. Applying our recursion formula to $x_1$, we obtain:
\begin{equation}\label{step 1 recursion}\left( (\alpha_1-1)\psi_1 + \ev_1^\st Y\right) \tilde\gamma \cap [\VZ_{1,(\alpha-e_1) \cup (1)}(X|Y,\beta)] = \tilde\gamma \cap [\Dcal(1)].\end{equation}
Let us first examine the left-hand side. The class $\psi_1$ differs from $\fgt_y^\st \psi_1$ by the locus where $x_1$ and $y$ are contained on a contracted rational bubble. This locus consists of reduced relative stable maps of the form \medskip

\begin{center}
\begin{tikzpicture}[scale=1]
%edge
\draw (0,0) to (2,0);
\draw [color=blue] (1,0) node[above]{\tiny$\alpha_1$};

%C_1
\draw [fill=white] (0,0) circle[radius=3pt];
\draw (0,-0.1) node[below]{\small$d$};
\draw (0,0) node[above]{\small$\sqC_1$};

%C_0
\draw [fill=black] (2,0) circle[radius=3pt];
\draw (2,-0.1) node[below]{\small$0$};
\draw (2,0) node[above]{\small$\sqC_0$};

%x_1
\draw [->] (2,0.1) -- (4,0.1);
\draw (3.9,0.1) node[above]{\small$x_1$};
\draw [color=blue](3,0.1) node[above]{\tiny$\alpha_1-1$};

%y
\draw [->] (2,-0.1) -- (4,-0.1);
\draw (3.9,-0.1) node[below]{\small$y$};
\draw [color=blue] (3,-0.1) node[below]{\tiny$1$};

%down arrow
\draw [color=blue,->] (1,-0.3) -- (1,-0.7);

%target
\draw [color=blue,->] (0,-1) to (2,-1);
\draw [fill=blue,color=blue] (0,-1) circle[radius=3pt];
\draw [color=blue] (2,-1) node[right]{\small$\Sigma(X|Y)$};
\end{tikzpicture}
\end{center}
% Algebro-geometric picture; replaced by tropical picture.
%%%%%%%%%%%%%%%%%%%%%%%%%%%%%%%%%%%%%%%%%%%%%%%%%%%%%%%%%%%
\begin{comment}
\begin{center}
\begin{tikzpicture}[scale=1.5]
%Draw target
\draw (0,0) -- (3,0) -- (3,2) -- (0,2) -- (0,0);
\draw (3,0) -- (6,0) -- (6,2) -- (3,2) -- (3,0);

%Draw genus one curve on left-hand side
\draw [blue] (3,0.5) to [out=90,in=270] (2.7,1) to [out=90,in=270] (3,1.5) to [out=90,in=0] (1,1.7) to [out=180,in=90] (0.5,1) to [out=270,in=180] (1,0.3) to [out=0,in=270] (3,0.5);
\draw [blue] (1.1,0.9) to [out=40, in=180] (1.4,1.1) to [out=0,in=140] (1.7,0.9);
\draw [blue] (1.2,1) to [out=300, in=180] (1.4,0.9) to [out=0,in=240] (1.6,1);

%Draw nodal blobs
\draw [fill=blue,color=blue] (3,0.5) circle[radius=1pt];
\draw [color=blue] (3,0.5) node[left]{\small$\alpha_2$};
\draw [fill=blue,color=blue] (3,1.5) circle[radius=1pt];
\draw [color=blue] (3,1.5) node[left]{\small$\alpha_1$};
%Draw rational bubble with x_1 and y
\draw [blue] (3,1.5) to [out=70,in=190] (5.5,1.9) to [out=0,in=90] (6,1.8) to [out=270,in=90] (5.75,1.5) to [out=270,in=90] (6,1.2) to [out=270,in=10] (5.5,1.1) to [out=180,in=290] (3,1.5);

%Draw right-hand markings
\draw [fill=blue,color=blue] (6,1.8) circle[radius=1pt];
\draw [color=blue] (6,1.8) node[left]{\small{$\alpha_1-1$}};
\draw (6,1.8) node[right]{$x_1$};
\draw [fill=blue,color=blue] (6,1.2) circle[radius=1pt];
\draw [color=blue] (6,1.2) node[left]{\small{$1$}};
\draw (6,1.2) node[right]{$y$};

%Draw ldots
\draw [color=blue] (4,0.5) node[right]{$\ldots$};
\end{tikzpicture}
\end{center}
\end{comment}
%%%%%%%%%%%%%%%%%%%%%%%%%%%%%%%%%%%%%%%%%%%%%%%%%%%%%%%%%
with all other marked points contained on $\sqC_1$. This is isomorphic to $\VZ_{1,\alpha}(X|Y,\beta)$ and when we restrict $\tilde\gamma$ to this locus we obtain precisely the class $\gamma$ which we started with. Thus the left-hand side of \eqref{step 1 recursion} may be written as
\begin{equation*} (\alpha_1-1) I + \fgt_y^\st \left( (\alpha_1-1)\psi_1 + \ev_1^\st Y\right)  \tilde\gamma \cap [\VZ_{1,(\alpha-e_1)\cup(1)}(X|Y,\beta)]\end{equation*}
where $I$ is the invariant we are trying to compute. The second term is a forgetful relative invariant with the same degree and number of true markings (since $y$ is fictitious), and strictly smaller true tangency; hence it is recursively known. We now examine the right-hand side of \eqref{step 1 recursion}. Recall that all of the genus zero data has already been computed, so we only need to focus on the genus one pieces. First consider the type $A$ loci. The genus one piece has strictly smaller degree (and hence is recursively known) exept in the following situation (with some stable distribution of the remaining markings):
\begin{center}
\begin{tikzpicture}[scale=1]
\draw (0,0) to (2,0);
\draw [fill=white] (0,0) circle[radius=3pt];
\draw (0,-0.1) node[below]{\small$d$};
\draw (0,0) node[above]{\small$\sqC_1$};
\draw [fill=black] (2,0) circle[radius=3pt];
\draw (2,-0.1) node[below]{\small$0$};
\draw (2,0) node[above]{\small$\sqC_0$};
\draw [->] (2,0) -- (3,0);
\draw (2.9,0) node[right]{$x_1$};

\draw [color=blue,->] (1,-0.3) -- (1,-0.7);

\draw [color=blue,->] (0,-1) to (2,-1);
\draw [fill=blue,color=blue] (0,-1) circle[radius=3pt];
\draw [color=blue] (2,-1) node[right]{\small$\Sigma(X|Y)$};
\end{tikzpicture}
\end{center}
In this situation, $C_1$ contains at most $n+1$ true markings. If it has $n-1$ or fewer, than it is known recursively. If it has exactly $n$ this means that $C_0$ contains exactly one true marking (besides $x_1$, which may be true or fictitious). We claim that in this situation we must have $y \in C_1$, since otherwise we would have a moduli space for $C_0$ given by $\ol\Mcal_{0,k}$ with $k \geq 4$, and  applying the projection formula with $\fgt_y$ we would conclude that the contribution is zero. Thus we have $y \in C_1$, and so the genus one contribution is a forgetful relative invariant with $n$ true markings, and hence is recursively known.

Finally, if $C_1$ contains exactly $n+1$ true markings, then the only additional markings on $C_0$ are fictitious, and by the same argument as in the previous paragraph there can only be one. Thus for each fictitious marked point we obtain a locus isomorphic to $\VZ_{1,\alpha}(X|Y,\beta)$ and $\tilde\gamma$ restricts to $\gamma$ here (from the point of view of computing invariants, the fictitious marked points are indistinguishable, meaning that the contributions are all the same). Thus for each fictitious marked point (of which there is at least one, namely $y$) we get a contribution of $\alpha_1 I$ to the right-hand side of \eqref{step 1 recursion}.

The contributions of the type $B$ loci only involve genus zero data and hence are known. The contributions of the type $C^0$ loci are determined by genus zero data and tautological integrals on Deligne--Mumford space, hence are also known. It remains to consider type $C^+$ loci. If the degree of the genus one piece is less than $d$ then we have a rubber invariant of strictly smaller degree. The only other possibility is that the entire curve is mapped into the divisor. In this case we may apply the projection formula with $\fgt_y$ to identify this with an integral over $\VZ_{1,m}(Y,\beta)$ for some (possibly large) number $m$ of marked points. But by assumption there is another marked point $x_0$ with all of the insertions pulled back along $\fgt_{x_0}$, so a further application of the projection formula shows that this contribution vanishes. To conclude, we may rearrange \eqref{step 1 recursion} to obtain an expression of the form
\begin{equation*} \lambda I = \text{recursively known terms} \end{equation*}
where $\lambda$ is an explicit scalar which is always nonzero; we have thus determined $I$, which completes Step 1.

\subsection*{Step 2} Consider now the absolute space $\VZ_{1,n}(Y,\beta)$ with an insertion $\gamma$, and suppose inductively that we have computed all forgetful relative invariants with $(d^\prime,n^\prime) \leq (d,n+1)$, all relative invariants and absolute invariants with $(d^\prime,n^\prime) < (d,n)$, and all special rubber invariants with $d^\prime < d$. Consider the following moduli space with $n+1$ markings:
\begin{equation*} \VZ_{1,(d,0,\ldots,0)}(X|Y,\beta). \end{equation*}
Let $x_0$ denote the relative marking and consider the integrand $\tilde\gamma$ obtained from $\gamma$ by introducing $\fgt^\st_{x_0}$ everywhere. Now recurse at $x_1$ to obtain:
\begin{equation}\label{step 2 recursion} \ev_1^\st Y \cdot \tilde\gamma \cap [\VZ_{1,(d,0,\ldots,0)}(X|Y,\beta)] = \tilde\gamma\cap[\Dcal(1)].\end{equation}
The left-hand side is a forgetful relative invariant of degree $d$ and $\leq n+1$ true markings, and so has already been computed. For the right-hand side, let us begin with loci of type $A$. The genus one contributions from each loci have $(d^\prime,n^\prime) < (d,n)$ except in the following case
\begin{center}
\begin{tikzpicture}[scale=1]
%edge
\draw (0,0) to (2,0);
\draw [color=blue] (1,0) node[above]{\tiny$d$};

%C_1
\draw [fill=white] (0,0) circle[radius=3pt];
\draw (0,-0.1) node[below]{\small$d$};
\draw (0,0) node[above]{\small$\sqC_1$};

%C_0
\draw [fill=black] (2,0) circle[radius=3pt];
\draw (2,-0.1) node[below]{\small$0$};
\draw (2,0) node[above]{\small$\sqC_0$};

%x_1
\draw [->] (2,0.1) -- (4,0.1);
\draw (3.9,0.1) node[above]{\small$x_1$};
\draw [color=blue](3,0.1) node[above]{\tiny$0$};

%y
\draw [->] (2,-0.1) -- (4,-0.1);
\draw (3.9,-0.1) node[below]{\small$x_0$};
\draw [color=blue] (3,-0.1) node[below]{\tiny$d$};

%down arrow
\draw [color=blue,->] (1,-0.3) -- (1,-0.7);

%target
\draw [color=blue,->] (0,-1) to (2,-1);
\draw [fill=blue,color=blue] (0,-1) circle[radius=3pt];
\draw [color=blue] (2,-1) node[right]{\small$\Sigma(X|Y)$};
\end{tikzpicture}
\end{center}
which gives a contribution of
\begin{equation*} d\cdot \gamma \cap [\VZ_{1,(d,\underbrace{0,\ldots,0}_{n-1})}(X|Y,\beta)] \footnote{(Navid) Fix formatting}\end{equation*}
to the right-hand side of \eqref{step 2 recursion}. Here $\gamma$ is the insertion we started with; the difference is that we are now considering relative maps to $(X,Y)$ with maximal tangency at $x_1$, rather than absolute maps to $Y$. The type $B$ and $C^0$ loci are recursively determined as in Step~1, and similarly the type $C^+$ loci are recursively determined except for the locus where the entire curve is mapped into the divisor. On this locus we may apply $\fgt_{x_0}$ and identify the contribution with
\begin{equation*} d^2 \cdot \gamma \cap [\VZ_{1,n}(Y,\beta)] = d^2 \cdot I \end{equation*}
where $I$ is the invariant we are trying to compute. Putting this all together, we obtain
\begin{equation}\label{step 2 recursion 2} I = (-\gamma/d) \cap [\VZ_{1,(d,\underbrace{0,\ldots,0}_{n-1})}(X|Y,\beta)] + \text{recursively known terms} \end{equation}
where on the right-hand side there are $n-1$ non-relative markings $x_2,\ldots,x_n$, and a relative marking $x_1$. We now apply the recursion again to the right-hand side, by considering the space
\begin{equation*} \VZ_{1,(d-1,1,0,\ldots,0)}(X|Y,\beta) \end{equation*}
where $x_1$ now has tangency $d-1$ and we have introduced a new marking $y$ with tangency $1$. We consider a new insertion, denoted $\tilde\gamma$ as usual, by introducing $\fgt_y^\st$ everywhere. Recursing at $x_1$ we obtain:
\begin{equation}\label{step 2 recursion 2} \left( (d-1)\psi_1 + \ev_1^\st Y \right)\cdot(-\tilde\gamma/d) \cap [\VZ_{1,(d-1,1,0,\ldots,0)}(X|Y,\beta)] = (-\tilde\gamma/d) \cap [\Dcal(1)].\end{equation}
The difference between $\psi_1$ and $\fgt_y^\st \psi_1$ is given by the locus where $x_1$ and $y$ belong to a collapsed rational bubble:
\begin{center}
\begin{tikzpicture}[scale=1]
%edge
\draw (0,0) to (2,0);
\draw [color=blue] (1,0) node[above]{\tiny$d$};

%C_1
\draw [fill=white] (0,0) circle[radius=3pt];
\draw (0,-0.1) node[below]{\small$d$};
\draw (0,0) node[above]{\small$\sqC_1$};

%C_0
\draw [fill=black] (2,0) circle[radius=3pt];
\draw (2,-0.1) node[below]{\small$0$};
\draw (2,0) node[above]{\small$\sqC_0$};

%x_1
\draw [->] (2,0.1) -- (4,0.1);
\draw (3.9,0.1) node[above]{\small$y$};
\draw [color=blue](3,0.1) node[above]{\tiny$1$};

%y
\draw [->] (2,-0.1) -- (4,-0.1);
\draw (3.9,-0.1) node[below]{\small$x_1$};
\draw [color=blue] (3,-0.1) node[below]{\tiny$d-1$};

%down arrow
\draw [color=blue,->] (1,-0.3) -- (1,-0.7);

%target
\draw [color=blue,->] (0,-1) to (2,-1);
\draw [fill=blue,color=blue] (0,-1) circle[radius=3pt];
\draw [color=blue] (2,-1) node[right]{\small$\Sigma(X|Y)$};
\end{tikzpicture}
\end{center}
The contribution of this locus to the left-hand side of \eqref{step 2 recursion 2} is:
\begin{equation*} (d-1)\cdot(-\gamma/d) \cap [\VZ_{1,(d,\underbrace{0,\ldots,0}_{n-1})}(X|Y,\beta)].\end{equation*}
What remains on the left-hand side is a forgetful relative invariant of degree $d$ and $\leq n+1$ true markings ($y$ being the ``forgetful'' marking), hence is recursively known. On the right-hand side, the type $A$ loci are recursively known except possibly in the following special cases (with some stable distribution of the remaining non-relative markings):
\begin{center}
\begin{minipage}{0.4\textwidth}
\begin{tikzpicture}[scale=1]
%edge
\draw (0,-0.1) to (2,-0.1);
\draw [color=blue] (1,-0.1) node[below]{\tiny$d-1$};

%C_1
\draw [fill=white] (0,0) circle[radius=3pt];
\draw (0,-0.1) node[below]{\small$d$};
\draw (0,0) node[above]{\small$\sqC_1$};

%C_0
\draw [fill=black] (2,0) circle[radius=3pt];
\draw (2,-0.1) node[below]{\small$0$};
\draw (2,0) node[above]{\small$\sqC_0$};

%x_1
\draw [->] (0,0.1) -- (1.5,0.1);
\draw (1.5,0.1) node[above]{\small$y$};
\draw [color=blue](0.75,0.1) node[above]{\tiny$1$};

%y
\draw [->] (2,0) -- (4,0);
\draw (3.9,0) node[below]{\small$x_1$};
\draw [color=blue] (3,0) node[below]{\tiny$d-1$};

%down arrow
\draw [color=blue,->] (1,-0.5) -- (1,-0.9);

%target
\draw [color=blue,->] (0,-1) to (2,-1);
\draw [fill=blue,color=blue] (0,-1) circle[radius=3pt];
\draw [color=blue] (2,-1) node[right]{\small$\Sigma(X|Y)$};

\draw (2,-1.3) node[below]{\small{Case 1}};
\end{tikzpicture}
\end{minipage}
\begin{minipage}{0.4\textwidth}
\begin{tikzpicture}[scale=1]
%edge
\draw (0,0) to (2,0);
\draw [color=blue] (1,0) node[above]{\tiny$d$};

%C_1
\draw [fill=white] (0,0) circle[radius=3pt];
\draw (0,-0.1) node[below]{\small$d$};
\draw (0,0) node[above]{\small$\sqC_1$};

%C_0
\draw [fill=black] (2,0) circle[radius=3pt];
\draw (2,-0.1) node[below]{\small$0$};
\draw (2,0) node[above]{\small$\sqC_0$};

%x_1
\draw [->] (2,0.1) -- (4,0.1);
\draw (3.9,0.1) node[above]{\small$y$};
\draw [color=blue](3,0.1) node[above]{\tiny$1$};

%y
\draw [->] (2,-0.1) -- (4,-0.1);
\draw (3.9,-0.1) node[below]{\small$x_1$};
\draw [color=blue] (3,-0.1) node[below]{\tiny$d-1$};

%down arrow
\draw [color=blue,->] (1,-0.3) -- (1,-0.7);

%target
\draw [color=blue,->] (0,-1) to (2,-1);
\draw [fill=blue,color=blue] (0,-1) circle[radius=3pt];
\draw [color=blue] (2,-1) node[right]{\small$\Sigma(X|Y)$};

\draw (2,-1.3) node[below]{\small{Case 2}};
\end{tikzpicture}
\end{minipage}
\end{center}
In Case 1 the contribution from $C_1$ is a forgetful relative invariant with $\leq n$ true markings, hence is recursively known. In Case 2, we first note that there cannot be any more markings on $C_0$ (since otherwise we could apply $\fgt_y$ and conclude that the contribution vanishes). Thus we obtain a single locus, which contributes precisely
\begin{equation*} d\cdot(-\gamma/d) \cap [\VZ_{1,(d,\underbrace{0,\ldots,0}_{n-1})}(X|Y,\beta)] = -\gamma\cap[\VZ_{1,(d,\underbrace{0,\ldots,0}_{n-1})}(X|Y,\beta)]\end{equation*}
which gives us the first term on the right-hand side of \eqref{step 2 recursion 2}. As usual the type $B$ and $C^0$ contributions are known recursively, and the only type $C^+$ contribution not known recursively occurs when the entire curve is mapped into the divisor, in which case we apply $\fgt_y$ to calculate the contribution as:
\begin{equation*} (-\gamma/d) \cap [\VZ_{1,n}(Y,\beta)] = -I/d.\end{equation*}
Substituting into \eqref{step 2 recursion 2} we end up with
\begin{equation*} I(1-d^{-1}) = \text{recursively known terms} \end{equation*}
which completes the recursion step as long as $d \neq 1$. But since $\VZ_{1,n}(H,1)=\emptyset$ it follows that $\VZ_{1,n}(Y,\beta)=\emptyset$ if $d=Y\cdot\beta=1$, so we may always assume $d \geq 2$ in the recursion.

\subsection*{Step 3} Now suppose we are given a relative space $\VZ_{1,\alpha}(X|Y,\beta)$ with an insertion $\gamma$, and suppose inductively that we have computed all forgetful relative invariants with $(d^\prime,n^\prime) \leq (d,n+1)$, all absolute invariants with $(d^\prime,n^\prime) \leq (d,n)$, all relative invariants with $(d^\prime,n^\prime,t^\prime)<(d,n,t)$ and all rubber invariants with $d^\prime < d$. Choose a true marking $x_1$ with $\alpha_1 \geq 1$ and consider the moduli space
\begin{equation*} \VZ_{1,(\alpha-e_1)\cup(1)}(X|Y,\beta) \end{equation*}
where $y$ is the newly-introduced marking. As usual consider the insertion $\tilde\gamma$ obtained from $\gamma$ by introducing $\fgt_y$ everywhere. Recursing at $x_1$ we obtain:
\begin{equation*} \left( (\alpha_1-1)\psi_1 + \ev_1^\st H\right) \tilde\gamma \cap [\VZ_{1,(\alpha-e_1)\cup(1)}(X|Y,\beta)] = \tilde\gamma \cap [\Dcal(1)].\end{equation*}
The left-hand side is a relative invariant with the same degree and number of true markings, but smaller true tangency: hence it is recursively known. On the right-hand side, the type $A$ contributions are recursively known except for those of the following form

\begin{center}
\begin{tikzpicture}[scale=1]
\draw (0,0) to (2,0);
\draw [color=blue] (1,-0.05) node[above]{\tiny$\alpha_1$};
\draw [fill=white] (0,0) circle[radius=3pt];
\draw (0,-0.1) node[below]{\small$d$};
\draw (0,0) node[above]{\small$\sqC_1$};
\draw [fill=black] (2,0) circle[radius=3pt];
\draw (2,-0.1) node[below]{\small$0$};
\draw (2,0) node[above]{\small$\sqC_0$};
\draw [->] (2,0) -- (4,0);
\draw (3.9,0) node[right]{$x_1$};
\draw [color=blue] (3,-0.05) node[above]{\tiny$\alpha_1-1$};

\draw [color=blue,->] (1,-0.3) -- (1,-0.7);

\draw [color=blue,->] (0,-1) to (2,-1);
\draw [fill=blue,color=blue] (0,-1) circle[radius=3pt];
\draw [color=blue] (2,-1) node[right]{\small$\Sigma(X|Y)$};
\end{tikzpicture}
\end{center}
where $C_0$ contains a single fictitious marking (if it had multiple fictitious markings then the contribution would vanish by projection formula) and all the other markings are on $C_1$. Thus for each fictitious marking we get a contribution of $\alpha_1 I$ where $I$ is the invariant we are trying to compute. Note that $\alpha_1 \neq 0$, and that this term appears at least once since $y$ is a fictitious marking; so we get a nonzero multiple of $I$.

As usual the type $B$ and $C^0$ contributions are recursively known. The type $C^+$ contributions are determined by lower-degree rubber invariants, except for when the whole curve is mapped into $Y$; however in this case we may apply $\fgt_y$ to identify the contribution with an absolute invariant of $Y$ with degree $d$ and $n$ markings, which is also recursively known. Thus we have determined the invariant $I$.

\subsection*{Step 4} Finally, consider a rubber space $\VZ_{1,\alpha}(P|Y_0+Y_\infty,\beta)_{\Gm}$ with insertion $\gamma$. Suppose inductively that we have computed all absolute and relative invariants with $d^\prime \leq d$ and all rubber invariants with $(d^\prime,n^\prime,m^\prime) < (d,n,m)$.

We first note the following important reduction: if there exists a relative marking $x_k$ such that all insertions are pulled back along $\fgt_{k}$, then we may apply the projection formula, together with the fact that
\begin{equation*} (\fgt_{k})_\st [\VZ_{1,\alpha}(P|Y_0+Y_\infty,\beta)_{\Gm}] = \alpha_k^2 \cdot [\VZ_{1,n+m-1}(Y,\beta)] \end{equation*}
to identify the rubber invariant with a multiple of a reduced invariant of $Y$, which has the same degree and hence is known recursively.

We will deal with the general case by reducing to the one above. Consider the moduli space
\begin{equation*}\label{step 4 recursion space} \VZ_{1,\alpha\cup (0)}(P|Y_0+Y_\infty,\beta)_{\Gm}\end{equation*}
obtained by introducing a marked point $y$ with no tangency. Let $x_1$ be a positive-tangency marking (such a marking always exists since $d \geq 2$) and let $\tilde\gamma$ be the insertion obtained from $\gamma$ by replacing $\ev_1$ and $\psi_1$ by $\ev_y$ and $\psi_y$, and then introducing $\fgt_{1}^\st$ everywhere.

We will make use of a recursion formula for rubber spaces analogous to the recursion formula for relative spaces used in Steps 1--3. Following \cite{EKatz}, there is a line bundle $L_y^{\not\in \operatorname{bot}}$ on $\VZ_{1,\alpha\cup (0)}(P|Y_0+Y_\infty,\beta)_{\Gm}$, together with a section $s_y^{\not\in\operatorname{bot}}$ whose vanishing locus consists of the locus $\Dcal(y)$ of rubber maps where $y$ is not mapped into the bottom level of the expanded target. As in \S \ref{}, we can give a logarithmic interpretation of this: it corresponds to the piecewise-linear function on the tropical moduli space which associates, to every rubber tropical map, the distance between $\varphi(\sqC_y)$ and the leftmost vertex of the tropical target, where $\sqC_y$ is the vertex of the source curve containing the flag corresponding to $y$. Using this tropical description, we may easily calculate the vanshing orders of $s_y^{\not\in\operatorname{bot}}$ along the various components of $\Dcal(y)$, and show that $\cchern_1(L_y^{\not\in\operatorname{bot}}) = \Psi_0 - \ev_y^\st Y$ (see also \cite{EKatzLB}, where similar results in the non-reduced setting are obtained, using different methods). From this, we obtain a rubber recursion formula
\begin{equation}\label{step 4 recursion} (\Psi_0 - \ev_y^\st Y) \tilde\gamma \cap [\VZ_{1,\alpha\cup(0)}(P|Y_0+Y_\infty,\beta)_{\Gm}] = \tilde\gamma \cap [\Dcal(y)]\end{equation}
where the fundamental class $[\Dcal(y)]$ is weighted by vanishing orders on the components. We will first show that the left-hand side is recursively known. By construction the class $\tilde\gamma$ is pulled back via $\fgt_{1}^\st$, and the same is true for $\ev_y^\st Y$. It remains to examine $\Psi_0$. If there exists a negative-tangency marking $x_2$, then we have \cite[Construction 5.1.17]{GathmannThesis}
\begin{equation*}\label{Psi0 formula} \Psi_0 = -\alpha_2 \hat\psi_2 - \ev_2^\st Y \end{equation*}
where $\hat\psi_2$ is a \emph{non-collapsed} psi class. (If there are no negative-tangency markings, then the construction given in \cite[\S 1.5.2]{MaulikPandharipande} shows that $\Psi_0=0$.) Now, $\hat\psi_2 - \psi_2$ is given by the loci where $x_2$ belongs to a trivial bubble. This entails a splitting of the curve into pieces, each of which contributes a rubber integral. Typically each of these pieces will have $(g^\prime,d^\prime,n^\prime,m^\prime) < (1,d,n,m)$ and hence be recursively known. The one exception is when all of the genus and degree is concentrated on the top level of the expansion, with the bottom level containing only a single non-relative marking in addition to all the negative-tangency markings. But in this case the contribution is a rubber invariant where all of the insertions are pulled back via $\fgt_1^\st$, and hence we may apply the projection formula to identify this with an absolute invariant of $Y$ which is recursively known. We conclude that, up to recursively knwon terms, we may replace $\hat\psi_2$ by $\psi_2$ in the left-hand side of \eqref{step 4 recursion}. If we now compare $\psi_2$ with $\fgt_1^\st \psi_2$ we see that the difference is given by the locus where $x_1$ and $x_2$ belong to a collapsed rational piece. The contribution of this locus consists of rubber invariant with strictly fewer relative markings, and hence is recursively known. So up to recursively-known terms \eqref{step 4 recursion} becomes
\begin{equation*} \fgt_1^\st (-\alpha_2\psi_2 - \ev_y^\st Y)\tilde\gamma \cap [\VZ_{1,\alpha\cup(0)}(P|Y_0+Y_\infty,\beta)_{\Gm}] = \tilde\gamma\cap[\Dcal(y)]\end{equation*}
and now the left-hand side is also recursively known, by the projection formula. Let us now examine the right-hand side. The components of $\Dcal(y)$ are indexed by splittings of the curve, and certainly the contributions are recursively known unless there is a piece of the curve carrying all of the genus and degree, so we may restrict to examining these contributions.

Let us denote the piece of the curve carrying all of the genus and degree by $C^\prime\subseteq C$. If $C^\prime$ is mapped to top level, then either it contains $x_1$, in which case we apply $\fgt_1$ to compute the contribution, or it does not contains $x_1$, in which case it has fewer than $n$ relative markings and is known recursively.  If on the other hand $C^\prime$ is not mapped to top level, then generically it is mapped to the bottom level (since generically the core is not contracted on this locus, so the desingularisation process does nothing). One possible contribution is given by the following locus
\begin{center}
\begin{tikzpicture}[scale=1]
%edge
\draw (0,0) to (2,0);
\draw [color=blue] (1,0) node[below]{\tiny$\alpha_1$};

%C_1
\draw [fill=white] (0,0) circle[radius=3pt];
\draw (0,-0.1) node[below]{\small$d$};

%y
\draw [->] (2,0) -- (2,1);
\draw (2,0.9) node[left]{$y$};
\draw [color=blue] (2,0.5) node[right]{\tiny$0$};

%x_1
\draw [->] (2,0) -- (3,0);
\draw (3,0) node[right]{$x_1$};
\draw [color=blue] (2.5,0) node[below]{\tiny$\alpha_1$};

%C_0
\draw [fill=black] (2,0) circle[radius=3pt];
\draw (2,-0.1) node[below]{\small$0$};

%down arrow
\draw [color=blue,->] (1,-0.5) -- (1,-0.9);

%target
\draw [color=blue] (0,-1) to (2,-1);
\draw [fill=blue,color=blue] (0,-1) circle[radius=3pt];
\draw [fill=blue,color=blue] (2,-1) circle[radius=3pt];
\draw [color=blue,->] (2,-1) -- (3,-1);
\draw [color=blue,->] (0,-1) -- (-1,-1);
\end{tikzpicture}
\end{center}
which contributes a nonzero multiple of the invariant $I$ we are trying to compute. For the other possibilities, first note that, unless every component at the top level contains a single positive-tangency marking and a single node (together with possibly some tangency-zero markings), then $C^\prime$ has fewer than $n$ relative markings and hence the contribution is known recursively. On the other hand, if any non-relative marking other than $y$ is mapped to the top level, $C^\prime$ has $\leq n$ relative markings and less than $m$ non-relative markings, and hence again the contribution is known recursively. The only remaining possibilities are when $x_1$ is replaced by another positive-tangency marking $x_k$ in the above picture; but then the contribution can be calculated by applying the projection formula to $\fgt_{1}$. We therefore conclude that the only contribution to the right-hand side of \eqref{step 4 recursion} which is not known recursively is a nonzero multiple of the invariant we were trying to compute. This completes the recursion step.

\bibliographystyle{alpha}
\bibliography{Bibliography}

\bigskip\bigskip

\noindent Luca Battistella\\
Max-Planck-Institut f\"ur Mathematik, Bonn \\
\href{mailto:battistella@mpim-bonn.mpg.de}{battistella@mpim-bonn.mpg.de}\\

\noindent Navid Nabijou \\
School of Mathematics and Statistics, University of Glasgow \\
\href{mailto:Navid.Nabijou@glasgow.ac.uk}{navid.nabijou@glasgow.ac.uk}\\

\noindent Dhruv Ranganathan \\
Department of Pure Mathematics and Mathematical Statistics, University of Cambridge \\
\href{mailto:dr508@cam.ac.uk}{dr508@cam.ac.uk}

\end{document}