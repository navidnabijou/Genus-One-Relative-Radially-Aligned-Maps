 \documentclass[11pt]{amsart}

\usepackage{accents}
\usepackage[all]{xy}
\usepackage{amsfonts}
\usepackage{amsmath}
\usepackage{amssymb}
\usepackage{amsthm}
\usepackage{cite}
\usepackage{color}
\usepackage{epstopdf}
\usepackage{fancyhdr}
\usepackage{float}
\usepackage{graphicx}
\usepackage{hyperref}
\usepackage{latexsym}
\usepackage{mathrsfs}
\usepackage{natbib}
\usepackage{sseq}
\usepackage{tikz-cd}
\usepackage{url}
\usepackage{verbatim}

%\usepackage{makeidx}
%\makeindex

\usepackage[full]{textcomp}
\usepackage[sups]{Baskervaldx}
\usepackage{cabin}
\usepackage[varqu,varl]{inconsolata}
\usepackage[baskervaldx,bigdelims,vvarbb]{newtxmath}
\usepackage[cal=cm]{mathalfa}

\usepackage[margin=0.8in]{geometry}

%\usepackage{moreverb}
%\usepackage{mathtools}
%\usepackage{marginnote}
%\usepackage{pifont}
%\usepackage{pictexwd,dcpic}
%\usepackage{setspace}
%\usepackage{lastpage}

\theoremstyle{plain}
\newtheorem{theorem}{Theorem}[section]

\newtheorem{claim}[theorem]{Claim}
\newtheorem{conjecture}[theorem]{Conjecture}
\newtheorem{corollary}[theorem]{Corollary}
\newtheorem{lemma}[theorem]{Lemma}
\newtheorem{proposition}[theorem]{Proposition}
%\newtheorem{question}[theorem]{Question}

\theoremstyle{remark}
\newtheorem{remark}[theorem]{Remark}

\theoremstyle{definition}
\newtheorem{aside}[theorem]{Aside}
\newtheorem*{aside*}{Aside}
\newtheorem*{question*}{Question}
\newtheorem{condition}[theorem]{Condition}
\newtheorem{construction}[theorem]{Construction}
\newtheorem{convention}[theorem]{Convention}
\newtheorem{definition}[theorem]{Definition}
\newtheorem{example}[theorem]{Example}
\newtheorem{exercise}{Exercise}
\newtheorem{notation}[theorem]{Notation}
\newtheorem{proposition-definition}[theorem]{Proposition-Definition}
\newtheorem{question}[theorem]{Question}
\newtheorem{setting}[theorem]{Setting}

%\numberwithin{equation}{section}

\DeclareMathOperator{\ev}{ev}
\DeclareMathOperator{\vdim}{vdim}
\DeclareMathOperator{\Hom}{Hom}
\DeclareMathOperator{\mult}{mult}
\DeclareMathOperator{\id}{id}
\DeclareMathOperator{\Bl}{Bl}
\DeclareMathOperator{\Pic}{Pic}
\DeclareMathOperator{\frPic}{\mathfrak{Pic}}
\DeclareMathOperator{\GW}{GW}
\DeclareMathOperator{\Achow}{A}
\DeclareMathOperator{\pt}{pt}

\newcommand{\virt}[1]{[#1]^{\operatorname{virt}}}
\newcommand{\M}[4]{\overline{\mathcal{M}}_{#1,#2}(#3,#4)}
\newcommand{\Q}[4]{\overline{\mathcal{Q}}_{#1,#2}(#3,#4)}
\newcommand{\C}{\mathbb{C}}
\newcommand{\T}{\mathbb{T}}
\newcommand{\G}{\mathbb{G}}
\newcommand{\PP}{\mathbb{P}}
\newcommand{\OO}{\mathcal{O}}
\newcommand{\N}{\mathbb{N}}
\newcommand{\Z}{\mathbb{Z}}
\newcommand{\QQ}{\mathbb{Q}}
\newcommand{\A}{\mathbb{A}}
\newcommand{\R}{\mathbb{R}}
\newcommand{\CP}{\mathbb{CP}}
\newcommand{\PD}{\mathrm{PD}}
\newcommand{\HH}{\operatorname{H}}

\newcommand{\mg}{\mathcal{M}_g}
\newcommand{\mgbar}{\bar{\mathcal{M}}_g}
\newcommand{\mgnbar}{\bar{\mathcal{M}}_{g,n}}
\newcommand{\monbar}{\bar{\mathcal{M}}_{0,n}}

\newcommand{\FF}{\mathcal{F}}
\newcommand{\calChat}{\hat{\cC}}
\newcommand{\calC}{\mathcal{C}}
\newcommand{\calO}{\mathcal{O}}
\newcommand{\calQ}{\mathcal{Q}}
\newcommand{\calL}{\mathcal{L}}
\newcommand{\calLbar}{\underline{\mathcal{L}}}
\newcommand{\frC}{\mathfrak{C}}
\newcommand{\frCbar}{\underline{\mathfrak{C}}}
\newcommand{\frD}{\mathfrak{D}}
\newcommand{\calM}{\mathcal{M}}
\newcommand{\calI}{\mathcal{I}}
\newcommand{\calE}{\mathcal{E}}
\newcommand{\frM}{\mathfrak{M}}
\newcommand{\frN}{\mathfrak{N}}
\newcommand{\frNbar}{\underline{\mathfrak{N}}}
\newcommand{\frX}{\mathfrak{X}}
\newcommand{\frXbar}{\underline{\mathfrak{X}}}
\newcommand{\Cstar}{\C^\times}
\newcommand{\pibar}{\underline{\pi}}
\newcommand{\ubar}{\underline{u}}
\newcommand{\frA}{\mathfrak{A}}
\newcommand{\frAbar}{\underline{\mathfrak{A}}}
\newcommand{\frd}{\mathfrak{d}}
\newcommand{\frF}{\mathfrak{F}}
\newcommand{\frFbar}{\underline{\mathfrak{F}}}
\newcommand{\sbar}{\underline{s}}
\newcommand{\frG}{\mathfrak{G}}
\newcommand{\frL}{\mathfrak{L}}
\newcommand{\frP}{\mathfrak{P}}
\newcommand{\frLbar}{\underline{\mathfrak{L}}}
\newcommand{\vhat}{\overline{v}}
\newcommand{\frE}{\mathfrak{E}}
\newcommand{\Cbar}{\underline{C}}
\newcommand{\Lbar}{\underline{L}}
\newcommand{\vbar}{\overline{v}}

\newcommand{\red}{\text{red}}

\newcommand{\im}{\text{im }}
\newcommand{\tick}{\ding{52}}
\newcommand{\quo}[1]{#1/\mathord{\sim}}
\newcommand{\sltwo}{\operatorname{SL_2}(\mathbb{Z})}
\newcommand{\sltwor}{\operatorname{SL_2}(\mathbb{R})}
\newcommand{\dd}{\text{d}}
\newcommand{\grad}{\text{grad}}
\newcommand{\ind}[1]{\text{Ind}(#1)}

\newcommand{\Newt}[1]{\text{Newt}(#1)}
\newcommand{\conv}{\text{conv}}
\newcommand{\val}{\text{val}}
\newcommand{\rank}[1]{\text{rank} \ #1}

\newcommand{\MnX}{\overline{\mathcal{M}}_{g,n}(X,\beta)}
\newcommand{\MnXD}{\overline{\mathcal{M}}_{g,n}(X/D,\beta,\underline{\alpha})}
\newcommand{\Spec}{\text{Spec}}
\newcommand{\acts}{\curvearrowright}

\def\bibfont{\footnotesize}

\begin{document}
 
\title{Response to referee report: AG 608 (Version 1)}
%\author{}
\maketitle

\noindent We thank the referee for their careful reading of our paper, and their many valuable suggestions which have greatly helped in improving the quality of the exposition.

We have made many improvements to the paper which directly address the referee's comments. Following their suggestion, we have also gone through and made additional changes, in order to improve the readability and appeal to a wider audience. Major such changes include:
\begin{enumerate}
	\item An improved introduction (see \S\S 1.1--1.5), explicating the connection between logarithmic structures and moduli of Gorenstein singularities which is a key geometric ingredient in the paper.
	\item More details in the proof of Theorem 1.9 (logarithmic smoothness of the moduli space), including a careful analysis of the various relative deformation/obstruction theories. 
\end{enumerate}




\subsection*{Response to specific comments}
\begin{enumerate}
\item[(25)] At the start of Section 5.1 we have added formulae to clarify the definition of the invariants.
\item[(26)] In Section 5.2 we define ``true tangency'' and ``true markings''; we have including more motivation for considering true and fictitious markings. In Section 5.3 we have added (following the table) a more detailed explanation of the recursion algorithm, which clarifies why all the base cases have degree zero. We have moreover discussed these cases in more depth (see Remark 5.4), clarifying the role played by ``tangency'' in this degenerate setting.
\item[(27)] We have clarified the meaning of $\alpha-e_1$. Here true markings can have $\alpha_1=1$, as long as all insertions are not pulled back along forgetting $x_1$; we have added a remark to this effect in Section 5.2. The term ``reduced relative'' is potentially confusing and has been removed.
\item[(28)] Throughout Section 5 we have replaced the ambiguous phrase ``is recursively known'' by the more precise ``has been computed earlier in the recursion'' or ``can be expressed as a polynomial in previously computed invariants''.
\item[(29)] Throughout Section 5 we have replaced this phrase with an explicit formula.
\item[(30)]
\item[(31)] We point out that this uses the same arguments as in Step~1. In the Step~1 arguments, we add explicit references to Gathmann's genus zero results and the necessary recursions on genus one Deligne--Mumford space.
\item[(32)] We have broken up Step 2 and Step 4 into smaller pieces. At several points we have made use of numbered lists for case analysis, which clarifies the logical structure. We have also provided summaries within steps as well as at the end of each step, to help the reader follow which parts of the formula the present arguments pertain to.
\end{enumerate}


\subsection*{Additional minor changes}
\begin{enumerate}
\item The term ``special rubber'' used at the start of Step 2 is an anachronism from a draft version of the paper, and has been removed.
\end{enumerate}



\end{document}