\documentclass[11pt]{amsart}
%\usepackage[english]{babel}
\usepackage{appendix}
\usepackage{amsmath}
\usepackage{amsfonts}
\usepackage{amssymb}
%\usepackage{showlabels}
\usepackage{hyperref}
\usepackage{amsthm}
\usepackage{marginnote}
\usepackage{stmaryrd}
\usepackage{enumitem}
\usepackage[english]{babel}
\usepackage{yfonts}
\usepackage[T1]{fontenc}
\usepackage[utf8x]{inputenc}
%\usepackage{enumerate}
\usepackage{verbatim}
\usepackage{graphicx}
\usepackage{verbatim}
\usepackage{faktor}
\usepackage{xcolor}
\usepackage{xfrac}
\usepackage{tikz,tikz-cd}
\usetikzlibrary{decorations.pathmorphing,decorations.pathreplacing,patterns}
\usepackage[all]{xy}
\usepackage{bbm}
\usepackage{tabularx}
\usepackage{longtable}
\usepackage{tabu}
\usepackage{booktabs}
\usepackage{mathtools}

\newcommand{\plC}{\scalebox{0.8}[1.3]{$\sqsubset$}}

\newcommand{\TT}{\operatorname{T}}
\newcommand{\oM}{\overline{\mathcal{M}}}
\newcommand{\M}[4]{\overline{\mathcal{M}}_{#1,#2}(#3,#4)}
\newcommand{\Q}[4]{\mathcal{Q}_{#1,#2}(#3,#4)}
\newcommand{\Qe}[4]{\mathcal{Q}^{\epsilon}_{#1,#2}(#3,#4)}
\newcommand{\Qt}[4]{\widetilde{\mathcal Q}_{#1,#2}(#3,#4)}
\newcommand{\QG}[4]{\mathcal{Q}G_{#1,#2}(#3,#4)}
\newcommand{\QGe}[4]{\mathcal{Q}G^{\epsilon}_{#1,#2}(#3,#4)}
\newcommand{\D}[3]{\mathcal{D^Q}(#1,#2,#3)}
\newcommand{\E}[3]{\mathcal{E^Q}(#1,#2,#3)}
\newcommand{\PP}{\mathbb P}
\newcommand{\Z}{\mathbb{Z}}
\newcommand{\tVZc}{\widetilde{\mathcal{V\!Z}}^{\rm{ctr}}_{1,n}}
\newcommand{\N}{\mathbb{N}}
\newcommand{\OO}{\mathcal{O}}
\renewcommand{\to}{\rightarrow}
\newcommand{\A}{\mathcal A}
\newcommand{\B}{\mathcal B}
\newcommand{\C}{\mathfrak C}
\newcommand{\cC}{\mathcal C}
\newcommand{\EE}{\mathbf{E}}
\renewcommand{\L}{\mathcal L}
\newcommand{\LL}{\mathbf{L}}
\newcommand{\MM}{\mathfrak M}
\newcommand{\Aaff}{\mathbb{A}}
\newcommand{\kfield}{\Bbbk}
\newcommand{\comp}{\chi}
\newcommand{\sst}{\sigma^{\operatorname{ss}}}
\newcommand{\Pic}{\operatorname{Pic}}
\newcommand{\Def}{\operatorname{Def}}
\newcommand{\Spec}{\operatorname{Spec}}
\newcommand{\Proj}{\operatorname{Proj}}
\newcommand{\Hom}{\operatorname{Hom}}
\newcommand{\Ext}{\operatorname{Ext}}
\newcommand{\Gm}{\mathbb{G}_{\text{m}}}
\newcommand{\virt}[1]{[#1]^{\operatorname{virt}}}
\newcommand{\vip}[1]{[#1]^{\operatorname{prod}}}
\newcommand{\Id}{\operatorname{Id}}
\newcommand{\CC}{\mathbb{C}}
\newcommand{\QQ}{\mathbb{Q}}
\newcommand{\HH}{\operatorname{H}}
\newcommand{\Achow}{\operatorname{A}}
\newcommand{\pt}{\operatorname{pt}}
\newcommand{\bq}{\begin{equation}}
\newcommand{\eq}{\end{equation}}
\newcommand{\ba}{\begin{aligned}}
\newcommand{\ea}{\end{aligned}}
\newcommand{\be}{\begin{enumerate}}
\newcommand{\ee}{\end{enumerate}}
\newcommand{\bsm}{\left(\begin{smallmatrix}}
\newcommand{\esm}{\end{smallmatrix}\right)}                   
\newcommand{\bpm}{\begin{pmatrix}}
\newcommand{\epm}{\end{pmatrix}}
\newcommand{\barr}{\begin{displaymath}\begin{array}{cccc}}
\newcommand{\earr}{\end{array}\end{displaymath}}
\newcommand{\barrl}{\begin{displaymath}\begin{array}{lcl}}
\newcommand{\earrl}{\end{array}\end{displaymath}}
\newcommand{\barl}{\begin{displaymath}\begin{array}{l}}
\newcommand{\earl}{\end{array}\end{displaymath}}
\newcommand{\bxym}{ \begin{displaymath}\xymatrix }
\newcommand{\exym}{\end{displaymath}}
\newcommand{\bcd}{\begin{center}\begin{tikzcd}}
\newcommand{\ecd}{\end{tikzcd}\end{center}}
\newcommand{\R}{\operatorname{R}^{\bullet}}
%\newcommand{\sslash}{\mathbin{/\mkern-6mu/}}
\newcommand{\tr}{{\rm tr}}
\newcommand{\Isom}{\text{Isom}}
\newcommand{\pr}{\operatorname{pr}}
\newcommand{\ev}{\operatorname{ev}}
\newcommand{\codim}{\operatorname{codim}}
\newcommand{\vdim}{\operatorname{vdim}}
\newcommand{\ildef}[1]{\emph{#1}}
\newcommand{\om}[1]{\mathcal{#1}}
\newcommand{\h}{\operatorname{h}}
\newcommand{\Aut}{\operatorname{Aut}}
\newcommand{\RR}{\textbf{R}}
\newcommand{\NN}{\operatorname{N}}


\theoremstyle{definition}
\newtheorem{thm}{Theorem}[section]
\newtheorem{lem}[thm]{Lemma}
\newtheorem{lemma}[thm]{Lemma}
\newtheorem{prop}[thm]{Proposition}
\newtheorem{cor}[thm]{Corollary}
\newtheorem*{teo*}{Theorem}
\newtheorem{ipotesi}{ipotesi}
\newtheorem*{nota}{Nota}
\newtheorem{claim}{Claim}
\newtheorem{question}[thm]{Question}
\newtheorem{conj}[thm]{Conjecture}

\newtheorem{innercustomthm}{Theorem}
\newenvironment{customthm}[1]
  {\renewcommand\theinnercustomthm{#1}\innercustomthm}
  {\endinnercustomthm}

\theoremstyle{definition}
\newtheorem{example}[thm]{Example}
\newtheorem{ex}[thm]{Example}
\newtheorem{dfn}[thm]{Definition}
\newtheorem{definition}[thm]{Definition}
\newtheorem{aside}[thm]{Aside}
\newtheorem{remark}[thm]{Remark}
\newtheorem{com}[thm]{Comment}
\newtheorem{num}{Number}
\newtheorem*{sketch}{Sketch}
\newtheorem*{rem}{Remark}
\newtheorem*{aside*}{Aside}
\newtheorem*{acknowledgements}{Acknowledgements}

\newcommand{\ilemph}[1]{\emph{#1}}

\setcounter{tocdepth}{1}

\newcommand{\todo}[1]{\vspace{5mm}\par \noindent
\framebox{\begin{minipage}[c]{0.95 \textwidth} \tt #1\end{minipage}} \vspace{5mm} \par}

\def\ti{-\allowhyphens}
\newcommand{\thismonth}{\ifcase\month % case 0 --- impossible!
  \or January\or February\or March\or April\or May\or June%
  \or July\or August\or September\or October\or November%
  \or December\fi}
\newcommand{\thismonthyear}{{\thismonth} {\number\year}}
\newcommand{\thisdaymonthyear}{{\number\day} {\thismonth} {\number\year}}

\usepackage[T1]{fontenc}
\usepackage{newpxtext,newpxmath}

\title[Genus One Radial Maps]{Relative Stable Maps in Genus One via Radial Alignments}
\author{Luca Battistella and Navid Nabijou}
\begin{document}

%\begin{abstract} \end{abstract}

\maketitle
%\appendixtitletocoff
%\tableofcontents

Recall \cite[Proposition 4.6.2.2]{RSPW}:
\begin{prop}
The morphism $\MM^{\rm{ctr}}_{1,n}\to\MM^{\dagger}_{1,n}$ is a log-modification.
\end{prop}

Explanation: this is a local statement so I can probably reduce to an atomic neighbourhood $S$ of a point $p$. $S$ and the curve over it are endowed with the minimal log structure; let $P=\oM_p$ determine a chart for this log structure. Observe that the subcurve $\plC_0$ of the tropicalisation $\plC$ of $C_p$ determines a subset $\rm{MinPos}$ of the set of vertices, namely those adjacent to $\plC_0$. Perform the following log-blowups: consider the set of primitive values of the function $\lambda\colon \plC\to P$, and blow up the ideal that they generate; now locally the set of values of $\lambda$ is principal with generator $p$: blow up the ideal generated by $\{\lambda(v)-p\}\setminus\{-p\}$. Keep going until $\lambda(v_i)$ is reached for some $v_i\in\rm{MinPos}$; at this point stop and declare the contraction radius $\delta:=\lambda(v_i)$. Finish by adjoining $\lambda(v)-\delta$ for all the vertices untouched to this stage. This shows that the choice of $\delta$ is not an extra degree of freedom.\marginpar{do I sound like a physicist?}



\section{Relative space equals closure of the nice locus}
Recall the definition of maps from centrally aligned curves via the cartesian diagram:
\bcd
\tVZc(X,\beta)\ar[d]\ar[r]\ar[dr,phantom,"\Box"] & \M{1}{n}{X}{\beta}\ar[d] \\
\MM_{1,n}^{\rm{ctr}}\ar[r] & \MM_{1,n}^{\dagger}
\ecd

\begin{dfn}
The centrally aligned relative space is defined within $\tVZc(Y,\beta)$ by the following:
\begin{itemize}
\item \emph{factorisation condition}: the map $f\colon C\to X$ factors through the associated contraction to a Smyth's singularity:
\bcd
\widetilde C\ar[r]\ar[d] & \overline C\ar[d,"\bar f"] \\
C\ar[r,"f"] & X
\ecd

\item \emph{Gathmann's relative condition}: for every connected component $Z$ of $f^{-1}(Y)$, either $Z$ is a point and, if it is marked $Z=x_i$, the multiplicity of $f$ at $x_i$ along $Y$ is at least $\alpha_i$; or $Z$ is a curve and $f^*\OO_X(Y)-\sum_{x_i\in Z}\alpha_ix_i$ is effective. Notice that $Z$ is at most genus $1$ and every line bundle of positive degree on a Gorenstein irreducible genus $1$ curve is effective, hence this condition can be rephrased as: the numerical condition
\[f_*[Z]\cdot Y+\sum_{j=1}^r m^{(j)}\geq \sum_{x_i\in Z}\alpha_i,\]
and, in case the numerical equality holds, the following equality of line bundles:
\[f_{\lvert Z}^*\OO_X(Y)=\OO_Z\left(\sum_{x_i\in Z}\alpha_ix_i - \sum_{j=1}^r m^{(j)}y_j\right)\in\Pic(Z).\]

\item \emph{novel condition}: there are no two vertices of $\rm{MinPos}$ closest to the circuit (i.e. $v_1\neq v_2\in \rm{MinPos}$ with $\delta=\lambda(v_1)=\lambda(v_2)$) unless the corresponding $m^{(j_1)}$ and $m^{(j_2)}$ are equal.
\end{itemize}
\end{dfn}

Here is one example where we show by a dimensional computation that the novel condition must be included if we hope to determine the closure of the nice locus.
\begin{example}
Consider $\M{1}{(3)}{\PP^1|H}{3}$. The virtual dimension is $7-3=4$. Here is a parametrisation of the nice locus: choose an element $(E,p)$ of $\oM_{1,1}\setminus\partial\oM_{1,1}$ (which has dimension $1$), let then $s_0\colon\OO_E\hookrightarrow\OO_E(3p)$ and $s_1$ be any section of $\OO_E(3p)$ not vanishing at $p$ (notice that $h^0(E,\OO_E(3p))=3$). Consider now the following weighted graph for a map in the boundary:
\begin{center}
\begin{tikzpicture}
\draw[color=brown] (0,0) node[left]{$E$} -- (4,0);
\draw[fill=black] (1,0) circle[radius=2pt] node[above]{$x_1$};
\draw[fill=black] (2,0) circle[radius=2pt] node[below right]{$y_1$};
\draw[fill=black] (3,0) circle[radius=2pt] node[below right]{$y_2$};
\draw (2,1.5) node[left]{$R_1$} (3,1.5) node[left]{$R_2$};
\draw (2,-1) -- (2,2) node[above]{$2$} (3,-1) -- (3,2) node[above]{$1$};
\draw[->] (5,0.5) -- (8,0.5);
\draw (9,-1) -- (9,2) node[above]{$\PP^1$};
\draw[fill=black] (9,0) circle[radius=2pt] node[right]{$H$};
\end{tikzpicture}
\end{center}
where the brown line represents a contracted genus $1$ curve. Now $(E,x_1,y_1,y_2)$ is a point of $\oM_{1,3}$ subject to the divisorial condition $3x_1-2y_1-y_2=0\in A_0(E)$; furthermore we have to choose the second ramification value of the $2\colon 1$ map from $R_1$ to $\PP^1$. This already makes up for a $3$-dimensional moduli space of degenerate relative maps corresponding to such a graph. The minimal log structure for this curve has a chart from $\-mathbb N^2$, with generators $e_1$ and $e_2$ corresponding to the smoothing parameters of the two nodes. If we allowed $e_1$ and $e_2$ to be identified in the characteristic sheaf, then we would get an extra $\Gm$ of choices for the log structure, so in total a $4$-dimensional moduli space. There is no chance this could live in the closure of the nice locus.
\end{example}

Thanks to the compatibility of $\delta$ (the contraction radius) with the stable map, namely the subcurve $C_0$ where $\lambda<\delta$ is the maximal connected subcurve of genus $1$ contracted by $f$, when the source curve is irreducible we have $\delta=0$, which is already ordered in every path to infinity. Hence the log structures only come into play when the source curve is reducible. We see then the nice locus in the radially aligned setting is the same as the nice locus in the ordinary setting. In particular, it is irreducible.

\begin{lem}
The nice locus is irreducible.
\end{lem}
\begin{proof}
A parametrisation can be given from the vector bundle:
\[ \operatorname{Vb}\left(\pi_*\OO_{\mathcal E}(\sum_{j=n+1}^{n+\delta}\sigma_j)\oplus\pi_*\OO_{\mathcal E}(\sum_{j=1}^{n+\delta}\sigma_j)^{\oplus r}\right) \quad \text{on} \quad \mathcal{M}_{1,n+\delta}\]
where $\pi\colon\mathcal E\to\mathcal M$ is the universal curve and $\delta=d-\sum\alpha_i$.
\end{proof}

We now want to show that the relative space in the radially aligned setting is equal to the closure of the nice locus; irreducibility follows immediately.

\begin{lem}
The closure of the nice locus is contained in the relative space.
\end{lem}
\begin{proof}
We address the relative conditions one at a time. Notice that they are all obviuously satisfied on the nice locus.
\begin{itemize}
\item The factorisation property is closed: see \cite[Theorem 4.3]{RSPW}.
\item Gathmann's relative condition is closed: see \cite[Proposition 4.9]{Vre}.
\item The novel condition is satisfied on the closure of the nice locus. Here is no proof but some heuristics. Consider for example a map similar to the one above:
\begin{center}
\begin{tikzpicture}
\draw[color=brown] (0,0) node[left]{$E$} -- (4,0);
\draw[fill=black] (1,0) circle[radius=2pt] node[above]{$x_1$};
\draw[fill=black] (2,0) circle[radius=2pt] node[below right]{$y_1$};
\draw[fill=black] (3,0) circle[radius=2pt] node[below right]{$y_2$};
\draw (2,1.5) node[left]{$R_1$} (3,1.5) node[left]{$R_2$};
\draw (2,-1) -- (2,2) node[above]{$3$} (3,-1) -- (3,2) node[above]{$2$};
\draw[->] (5,0.5) -- (8,0.5);
\draw (9,-1) -- (9,2) node[above]{$\PP^1$};
\draw[fill=black] (9,0) circle[radius=2pt] node[right]{$H$};
\draw[fill=black] (9,1.5) circle[radius=2pt] node[right]{$H'$};
\end{tikzpicture}
\end{center}
and assume that it is in the closure of the nice locus; I want to argue that the two smoothing parameters cannot be identified. By the previous points I may assume that the factorisation property and Gathmann's relative conditions hold. I have a diagram:
\bcd
\cC^{\rm{ss}}\ar[r]\ar[dr] & \widetilde \cC\ar[r]\ar[d] & \overline \cC\ar[d,"\bar f"] \\
& \cC\ar[r,"f"] & X
\ecd
where I have included the semistable model of $\cC$ (and $\widetilde \cC$), thought of as a curve marked with $f^{-1}(H')$ as well, where $H\neq H'\in\PP^1$.

Notice that $\cC$ is a normal surface with at worst singular points at the nodes of $C=\cC_0$ (a variety is smooth at any smooth point of a Cartier divisor) and the singularities are of type $A_{n_i},\ i=1,2$ (from the deformation theory of nodal curves).

I assume maximal multiplicity $\sum\alpha=d$, i.e. I am looking at the moduli space $\M{1}{(5)}{\PP^1|H}{5}$. Hence the line bundle and $s_0$ are determined as $f^*\OO_{\PP^1}(1)=\OO_{\cC}(5x_1)\otimes\OO_{\cC}(\beta E)$ for some (positive rational) $\beta$ and $s_0$ is the natural inclusion of $\OO_{\cC}$ (up to $\Gm$). 

The map is totally ramified at $y_i$ by Gathmann's condition. We conclude:
\[ \frac{\beta}{n_i+1}=(\beta E)\cdot R_i=m^{(i)}\]
Hence, having fixed the multiplicities, the two possible singularities of $\cC$ determine each other. In our example we may pick $n_1=1,\ n_2=2,\ \beta=6$. But knowing the singularity determines the semistable model: in our case $y_1$ is replaced by a $(-2)$-curve and $y_2$ by a chain of two $(-2)$-curves. Now we know from the work of Smyth \cite[Proposition 2.12]{SMY1} that the exceptional locus of $\cC^{\rm{ss}}\to\overline\cC$ is balanced, therefore we may deduce that $\bar f$ is constant on the branch of the genus $1$ singularity to which $R_2$ is joined, so $\overline C$ looks like this:

\begin{center}
\begin{tikzpicture}
\draw (-1,0) -- (1,0) node[right]{$\overline R_1$};
\draw[color=gray!50] (-1,-1) -- (1,1);
\draw[fill=black] (-.75,-.75) circle[radius=2pt] node[left]{$x_1$};
\draw[color=gray!50] (0,-1) -- (0,1);
\draw (-1,1.5) node[above left]{$\overline R_2$} -- (.25,.5);
\end{tikzpicture}
\end{center}
where a gray line is contracted by the map. On the other hand if $\delta=\lambda(v_1)=\lambda(v_2)$ then the prescription of \cite[Proposition 3.6.1]{RSPW} implies that $\tilde C\to \overline C$ looks like:
\begin{center}
\begin{tikzpicture}
\draw[color=brown] (0,0) node[left]{$E$} -- (4,0);
\draw[color=gray!50] (1,-1) -- (1,2);
\draw[fill=black] (1,1) circle[radius=2pt] node[left]{$x_1$};
\draw (2,1.5) node[left]{$R_1$} (3,1.5) node[left]{$R_2$};
\draw (2,-1) -- (2,2) node[above]{$3$} (3,-1) -- (3,2) node[above]{$2$};
\draw[->] (5,0.5) -- (7,0.5);
\draw (8,0.5) -- (10,0.5) node[right]{$\overline R_1$};
\draw[color=gray!50] (8,-.5) -- (10,1.5);
\draw[fill=black] (8.25,-.25) circle[radius=2pt] node[left]{$x_1$};
\draw (9,-.5) -- (9,1.5) node[above left]{$\overline R_2$};
\end{tikzpicture}
\end{center}
Contradiction!
\end{itemize}
\end{proof}

It thus remains to show that, given a relative radially aligned map, we can smooth it to one in the nice locus. \textcolor{red}{This is done by considering different cases locally, then gluing. \marginpar{what do we mean?}}

\subsection*{Case 1: non-contracted genus one internal component} Assume that the curve takes the form
\begin{equation*} C = C_0 \cup C_1 \cup \ldots \cup C_k \end{equation*}
where all the $C_i$ are smooth, $C_0$ has genus one, all the other $C_i$ have genus zero, and for $i \in \{1,\ldots,k\}$, $C_i$ intersects $C_0$ at a single node (denoted $q_i$) and does not intersect any other components.

Suppose furthermore that $C_0$ is a non-contracted \emph{internal component}, meaning that it is mapped into $H$ via $f$, and that $C_1$,\ldots,$C_k$ are \emph{external components}, meaning that they are not mapped into $H$ via $f$. The picture is:

[FIGURE]

Suppose that this is a relative stable map. This means that [BLAH]. We claim that it can be smoothed to a relative stable map in the nice locus. The construction depends on choosing an appropriate smoothing of the curve $C$, so that the map also smooths.

We start with $W = C_0 \times \Aaff^1_t$ (where $t$ denotes a fixed co-ordinate on the affine line). This is a smooth surface, fibred over $\Aaff^1_t$, with fibre equal to the elliptic curve $C_0$. Consider the points $q_1, \ldots, q_k$ on $C_0$. We will perform a series of weighted blow-ups at the points $(q_i,0) \in W$, in order to obtain a surface whose general fibre is smooth (in fact, isomorphic to $C_0$) and whose central fibre is isomorphic to $C$.

Fix $i \in \{1,\ldots,k\}$ and let $m_i$ be the multiplicity of $f$ with $H$ at $q_i \in C_i$. We define:
\begin{equation*} l = \operatorname{lcm}(m_1,\ldots,m_k) \qquad r_i = l/m_i \end{equation*}
We now blow-up the surface $W$ at the points $(q_i,0)$ with weight $r_i$ in the horizontal direction and weight $1$ in the vertical direction: if $x_i$ is a local co-ordinate for the fibre around $q_i$, this means that we blow-up in the ideal $(x_i,t^{r_i})$.

The result is a fibred surface $W^\prime \to \Aaff^1_t$ with general fibre equal to $C_0$ and central fibre $W^\prime_0 \cong C$. The total space of $W^\prime$ is no longer smooth (its singular points are [BLAH]), but this is not a problem since the projection to $\Aaff^1_t$ is still flat. The central fibre is a linearly trivial Cartier divisor:
\begin{equation*} W^\prime_0 = C_0 + C_1 + \ldots + C_k = 0 \in \Pic W^\prime \end{equation*}
For $i \in \{1,\ldots,k\}$ we have that $r_i C_i$ is Cartier, although the same is not necessarily true of $C_i$. Furthermore, since
\begin{equation*} l C_0 = - \sum_{i=1}^k l C_i = - \sum_{i=1}^k m_i (r_i C_i) \end{equation*}
in $A_1(W^\prime)$, it follows that $lC_0$ is Cartier. Finally, a local computation shows that
\begin{equation*} r_i C_i \cdot C_0 = 1 \end{equation*}
for $i \in \{1,\ldots,k\}$. Now, let $x_1,\ldots,x_n$ denote the marked points of $C$. These are smooth points of the central fibre $W^\prime_0$, and hence can be extended to Cartier divisors $\tilde{x}_1,\ldots,\tilde{x}_n$ on $W^\prime$. Consider the line bundle:
\begin{equation*} \tilde{L} = \OO_{W^\prime}(l C_0 + \Sigma_{j=1}^n \alpha_j \tilde{x}_j) \end{equation*}
on $W^\prime$. We claim that this gives a smoothing of the line bundle $L=f^*\OO(1)$ on $C$, i.e. that $\tilde{L}|_{W^\prime_0} = L$. We show this by first restricting $\tilde{L}$ to each of the components $C_i$ of $W^\prime_0 \cong C$. For $i \in \{1,\ldots,k\}$, we have
\begin{align*} \tilde{L}|_{C_i} & = \OO_{C_i} \left( (l C_0 \cdot C_i) q_i + \sum_{x_j \in C_i} \alpha_j x_j \right) = \OO_{C_i} \left( (l/r_i) q_i + \sum_{x_j \in C_i} \alpha_j x_j \right)\\
& = \OO_{C_i} \left( m_i q_i + \sum_{x_j \in C_i} \alpha_j x_j \right) = L|_{C_i} \end{align*}
while for $i=0$ we have:
\begin{align*} \tilde{L}|_{C_0} & = \OO_{C_0} \left( - \sum_{i=1}^k (l C_i \cdot C_0) q_i + \sum_{x_j \in C_0} \alpha_j x_j \right) = \OO_{C_0} \left( - \sum_{i=1}^k m_i q_i + \sum_{x_j \in C_0} \alpha_j x_j \right)  = L|_{C_0} \end{align*}
Finally the fact that $\tilde{L}|_{W^\prime_0} = L$ follows from the fact that the dual intersection graph of $C$ has genus zero.

Now, $\tilde{L}$ comes with a unique section whose restriction to $W^\prime_0 \cong C$ is $s_0$. After we extend the sections $s_1,\ldots,s_N$, it is clear that the resulting stable map is in the nice locus (i.e. that it is not mapped into $H$).

In order to extend the sections $s_1,\ldots,s_N$, we simply check that they are unobstructed. The space containing the obstructions to extending the sections is:
\begin{equation*} \HH^1(C,L) \end{equation*}
\marginpar{Is this true even when $C$ is reducible}By taking the normalisation exact sequence for $C$, tensoring with $L$ and passing to cohomology, we obtain an exact sequence:
\begin{align*} 0 \to & \HH^0(C,L) \to \bigoplus_{i=0}^k \HH^0(C_i,L) \xrightarrow{\theta} \bigoplus_{i=1}^k L_{q_i} \to \\
\to & \HH^1(C,L) \to \bigoplus_{i=0}^k \HH^1(C_i,L) \to 0 \end{align*}
Now, each of $C_1,\ldots,C_k$ is isomorphic to $\PP^1$ and $L|_{C_i}$ has non-negative degree; hence the map $\theta$ is surjective. Thus the map
\begin{equation*} \HH^1(C,L) \to \bigoplus_{i=0}^k \HH^1(C_i,L) \end{equation*}
is an isomorphism. But $\HH^1(C_i,L)=0$ for $i \in \{1,\ldots,k\}$ since $C_i \cong \PP^1$ and $L|_{C_i}$ has non-negative degree; also we have by Serre duality
\begin{equation*} \HH^1(C_0,L) \cong \HH^0(C_0, L^\vee \otimes \omega_{C_0}) = \HH^0(C_0,L^\vee) = 0 \end{equation*}
where the penultimate equality holds because $\operatorname{g}(C_0)=1$ and the last equality holds because $L|_{C_0}$ has \emph{strictly} positive degree (here we are using the fact that $f|_{C_0}$ is non-constant).

To conclude, we have a family $\tilde{C} = W^\prime$ of nodal curves and a map from this family to $\PP^N$
\bcd
\tilde{C} \ar[r,"\tilde{f}"] \ar[d,"\pi"] & \PP^N \\
\Aaff^1_t & \,
\ecd
such that when we restrict to $0 \in \Aaff^1_t$ we recover the map $f\colon C \to \PP^N$ and such that the general fibre is an element of the nice locus.

\newpage
\bibliographystyle{alpha}
\bibliography{relqm}

\bigskip\bigskip

\noindent Luca Battistella\\
Department of Mathematics, Imperial College London \\
\texttt{l.battistella14@imperial.ac.uk}\\

\noindent Navid Nabijou \\
Department of Mathematics, Imperial College London \\
\texttt{navid.nabijou09@imperial.ac.uk}



\end{document}