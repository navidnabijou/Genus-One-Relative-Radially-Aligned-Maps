\documentclass[11pt]{amsart}
\linespread{1.25}
\usepackage[margin=3.5cm]{geometry}

\usepackage[english]{babel}
\usepackage{appendix}
\usepackage{amsmath}
\usepackage{amsfonts}
\usepackage{amssymb}
%\usepackage{showlabels}
\usepackage{hyperref}
\usepackage{amsthm}
\usepackage{marginnote}

\usepackage{stmaryrd}
\usepackage{enumitem}
\usepackage[english]{babel}
\usepackage{yfonts}
\usepackage[T1]{fontenc}
\usepackage[utf8x]{inputenc}

\usepackage{calrsfs}
\DeclareMathAlphabet{\pazocal}{OMS}{zplm}{m}{n}

\usepackage{verbatim}
\usepackage{graphicx}
\usepackage{verbatim}
\usepackage{faktor}
\usepackage{xcolor}
\usepackage{xfrac}
\usepackage{tikz,tikz-cd}
\usetikzlibrary{decorations.pathmorphing,decorations.pathreplacing,patterns}

\usepackage[all]{xy}
\usepackage{bbm}
\usepackage{tabularx}
\usepackage{longtable}
\usepackage{tabu}
\usepackage{booktabs}
\usepackage{mathtools}

\usepackage[]{textcomp}
\usepackage[sups]{Baskervaldx}
\usepackage{cabin}
\usepackage[varqu,varl]{inconsolata}
\usepackage[baskervaldx,bigdelims,vvarbb]{newtxmath}
\usepackage[cal=cm]{mathalfa}


\newcommand{\plC}{\scalebox{0.8}[1.3]{$\sqsubset$}}
\newcommand{\sidenote}[1]{\marginpar{\textbf{\color{red}#1}}}

\theoremstyle{definition}
\newtheorem{thm}{Theorem}[section]
\newtheorem{lem}[thm]{Lemma}
\newtheorem{lemma}[thm]{Lemma}
\newtheorem{prop}[thm]{Proposition}
\newtheorem{cor}[thm]{Corollary}
\newtheorem*{teo*}{Theorem}
\newtheorem{ipotesi}{ipotesi}
\newtheorem*{nota}{Nota}
\newtheorem{claim}{Claim}
\newtheorem{question}[thm]{Question}
\newtheorem{conj}[thm]{Conjecture}

\newtheorem{innercustomthm}{Theorem}
\newenvironment{customthm}[1]
  {\renewcommand\theinnercustomthm{#1}\innercustomthm}
  {\endinnercustomthm}

\theoremstyle{definition}
\newtheorem{example}[thm]{Example}
\newtheorem{ex}[thm]{Example}
\newtheorem{dfn}[thm]{Definition}
\newtheorem{definition}[thm]{Definition}
\newtheorem{aside}[thm]{Aside}
\newtheorem{remark}[thm]{Remark}
\newtheorem{com}[thm]{Comment}
\newtheorem{num}{Number}
\newtheorem*{sketch}{Sketch}
\newtheorem*{rem}{Remark}
\newtheorem*{aside*}{Aside}
\newtheorem*{acknowledgements}{Acknowledgements}

\title{Gromov-Witten theory of hypersurfaces in genus one}
\author{Dhruv Ranganathan}
\date{February 2019}

\begin{document}

\maketitle

\begin{abstract}
    This is a section of the working draft of Battistella--Nabijou--Ranganathan. We construct reduced Gromov--Witten theory or projective in genus one, relative to a hyperplane, its rubber variant, establish its unobstructedness, and describe the stratification by boundary strata.
\end{abstract}

\section{The moduli space and deformation theory}

Let $\mathfrak M_{1,n}$ be the logarithmic algebraic stack of genus $1$ $n$-pointed prestable curves. Let $\plC$ be a $n$-pointed genus $1$ tropical curve. Let $C$ be a logarithmic curve over $S$, let $\plC$ denote its tropicalization, and let $\lambda$ be the section of $\overline{M}_{C/S}$ giving the distance from the circuit. 

We begin by recalling the logarithmic moduli spaces of genus $1$ curves constructed in~\cite[Sections 2 \& 4]{RSPW}. Given a family of tropical curves over a base $\sigma$, a central alignment is a piecewise-linearly varying choice of radius $\delta_s$ for $s\in\sigma$, together with a consistent ordering of the vertices that lie inside the circle of radius $s$ around the circuit components. Precisely, let $T$ be a geometric point with logarithmic structure.

\begin{definition}
A \textbf{central alignment} of $C/T$ is an element $\delta\in\overline{M}_S$ such that
\begin{enumerate}
    \item the section $\delta$ is comparable to $\lambda(v)$ for all vertices $v$ of $\plC$,
    \item for any pair of vertices $v$ and $w$ at distance less than $\delta$, the sections $\lambda(v)$ and $\lambda(w)$ are comparable.
\end{enumerate}
\end{definition}

The moduli stack $\mathfrak M_{1,n}^{\mathrm{cen}}$ is a logarithmic algebraic stack in the smooth topology. The main construction of~\cite{RSPW} canonically associates to any radially aligned family of curves $\mathscr C_S$ a partial destabilization $\widetilde{\mathscr C_S}$ and a contraction
\[
\widetilde{\mathscr C_S}\to \overline{\mathscr C_S},
\]
where $\overline{\mathscr C_S}$ is a Gorenstein elliptic singular curve. The number of branches is equal to the number of excident edges at the circle of radius $\delta$. This data uniquely determines the singularity.

The space of stable maps $\overline{\mathcal M}^{\mathrm{cen}}_{1,n}(\mathbb P^N,d)$ from the universal nodal curve $\mathscr C$ over the stack $\mathfrak M_{1,n}^{\mathrm{cen}}$ is a proper and algebraic, with projective coarse moduli. The \textbf{factorization condition} that the map $\mathscr C\to \mathbb P^N$ factors through the contraction $\widetilde{\mathscr C}\to \overline{\mathscr C}$ is a closed condition. 
\begin{thm}[{\cite[Theorem B]{RSPW}}]
The substack $\mathcal{VZ}_{1,n}(\mathbb P^N,d)$ of $\overline{\mathcal M}^{\mathrm{cen}}_{1,n}(\mathbb P^N,d)$ parameterizing maps to $\mathbb P^N$ that satisfy the factorization property is smooth and proper of the expected dimension.
\end{thm}

\subsection{Relative geometry: compactification} Fix $H\subset \mathbb P^N$ be a hyperplane. Let $\alpha$ be a partition of the degree $d$. Consider the moduli space $\mathcal M_{1,\alpha}^\circ(\mathbb P^N|H)$ of maps from smooth elliptic curves $C\to \mathbb P^N$ that meet $H$ at finitely many marked points with vanishing orders given by the partition $\alpha$. This is a smooth non-compact Deligne--Mumford stack. 

We will first compactify the space described above, and then desingularize it. For the compactification, we begin with Abramovich--Chen--Gross--Siebert's space of logarithmic stable maps, though we will typically work with various subcategories and variants. 

The moduli space $\overline{\mathcal M}^{\mathrm{log}}_{1,\alpha}(\PP^N|H)$ is a fibered category over logarithmic schemes, whose fiber over $(S,M_S)$ is the groupoid of logarithmic curves of genus $1$ over (S,M_S)$ equipped with a map to $\mathbb P^N$ of degree $d$ and contact order $\alpha$. It is a fundamental fact in the subject that this category is representable by a proper algebraic stack with logarithmic structure. 

There is a representable finite logarithmic morphism to the Kontsevich space, forgetting the logarithmic structure on the target:
$$
\overline{\mathcal M}^{\mathrm{log}}_{1,\alpha}(\mathbb P^N|H) \to \overline{\mathcal M}_{1,n}(\mathbb P^N,d).
$$
The space of centrally aligned maps is a logarithmic modification
$$
\overline{\mathcal M}^{\mathrm{cen}}_{1,n}(\mathbb P^N,d)\to\overline{ \mathcal M}_{1,n}(\mathbb P^N,d)
$$ 
The fiber product leads to a fourth moduli space of \textbf{centrally aligned logarithmic maps to $(\mathbb P^n,H)$}, which we denote $\widetilde{\mathcal{VZ}}_{1,\alpha}(\mathbb P^N|H)$.


\subsection{Relative geometry: expansions} Our next task is to pick out a non-singular principal component in $\widetilde{\mathcal{VZ}}_{1,\alpha}(\mathbb P^N|H)$. The principal component of this space, consisting of the closure of the space of maps from nonsingular curves, maps into the principal component of $\overline{\mathcal M}^{\mathrm{cen}}_{1,n}(\mathbb P^N,d)$. Indeed, smoothable logarithmic maps are, in particular, smoothable as ordinary maps. However, an additional condition is required to isolate the principal component of the space of logarithmic maps. 

In order make the obstruction theory more geometric, we expand the target. To elucidate the connection with the static target, consider a logarithmic stable map $[C\to (\mathbb P|H)]$ over $\mathrm{Spec} \ (\mathbb N\to \mathbb C)$. At the level of tropicalizations, we have a map of fans
\[
\plC\to \mathbb R_{\geq 0}.
\]
Choose a subdivision of $\mathbb R_{\geq 0}$ whose vertices consist of the images of vertices of $\plC$. Pull this subdivision back to $\plC$ by marking all preimages of the vertices of $\mathbb R_{\geq 0}$. Denote the resulting map $\widetilde \plC \to \widetilde{\mathbb R}_{\geq 0}$. 

These subdivisions induce logarithmic modifications
\[
\widetilde C\to \mathbb P^N[s],
\]
see~\cite{AW}. Here the latter is the $s$-times iterated deformation to the normal cone of $H$ in $\mathbb P^N$. The number of components is equal to the number of vertices in $\widetilde{\mathbb R}_{\geq 0}$. The curve is modified by adding rational components corresponding to the newly introduced vertices.

The result is a logarithmic stable map to an expanded target $\mathbb P^N[s]$, together with a contraction to the main target component $\mathbb P^N[s]\to\mathbb P^N$. More generally, this construction is easily modified for any family of logarithmic maps, such that the vertices of $\plC$ are totally ordered in the family, see~\cite{} for details.

Globally, Kim constructs a moduli space of logarithmic stable maps to expanded degenerations, which on logarithmic points, reduces to the above construction~\cite{KimLog}. Indeed, by above description, Kim's space is identified with a subcategory of the Abramovich--Chen--Gross--Siebert space, and its minimal objects are identified with a logarithmic modification of the unexpanded space.

\subsection{Relative geometry: factorization} Let $[C\to \mathbb P^N[s]\to \mathbb P^N]$ be a logarithmic map from a centrally aligned curve to an expansion. Recall that $\mathbb P^N[s]$ consists of a union of $\mathbb P^N$ with the projective bundle $\mathbb P(\mathcal O\oplus \mathcal O(1))$. We refer to these latter components as \textbf{the higher levels}. Thus, we will say that subcurve $D\subset C$ maps to \textbf{higher level} if the collapsed map
\[
D\to \mathbb P^N
\]
is contained in $H\subset \mathbb P^N$. 

Let $D_1\subset C$ be the maximal genus $1$ subcurve that is mapped to higher level contracted by the map $C\to \mathbb P^N[s]$ in the fiber direction. Let $\delta_1$ be the associated radius from the circuit to the nearest non-contract component. Let $D_2$ be the maximal genus $1$ subcurve that is contracted by the collapsed map to $\mathbb P^N$, and let $\delta_2$ be the associated radius. This coincides with the radius of the underlying map to $\mathbb P^N$. Of course, $\delta_1\leq \delta_2$.

The datum $(\delta_1,\delta_2)$ determines a destabilization $\widetilde C$ of $C$ together with successive contractions $\widetilde C\to \overline C_1\to\overline C_2$. Keeping this definition, we come to the key definition.

\begin{definition}
The map $[C\to \mathbb P^N[s]\to \mathbb P^N]$ \textbf{factorizes completely} if
\begin{itemize}
\item the map $C\to \mathbb P^N[s]$ factors through $\overline C_1$ such that at least one branch of $\overline C_1$ has positive degree in the fiber direction. 
\item the collapsed map to $\mathbb P^N$ factorizes through $\overline C_2$ such that at least one branch of $\overline C_2$ has positive degree.
\end{itemize}
\end{definition}

In particular, if $[C\to \mathbb P^N[s]\to \mathbb P^N]$ is a family of centrally aligned maps over $S$ that factorizes completely, there is a forgetful moduli map $S\to \mathcal{VZ}_{1,n}(\mathbb P^N,d)$, to the principal component of the space of absolute maps.

Let $\mathcal{VZ}_{1,\alpha}(\mathbb P^N|H)$ be stack of maps from centrally aligned curves to expansions of $\mathbb P^N$ that factorize completely.

\begin{thm}\label{thm: log-smoothness}
The stack $\mathcal{VZ}_{1,\alpha}(\mathbb P^N|H)$ is proper and logarithmically non-singular over $\mathrm{Spec} \ \mathbb C$. 
\end{thm}

\begin{proof}
We prove the results via the forgetful morphism
$$
\nu: \mathcal{VZ}_{1,\alpha}(\mathbb P^N|H)\to \mathcal{VZ}_{1,n}(\mathbb P^N,d),
$$
which remembers only the stabilization of the collapsed map. The map is certainly centrally aligned, and we have already argued above that it satisfies the factorization property for $\mathbb P^N$ after composition. The morphism $\nu$ is in fact proper: the verification for the valuative criterion is the essential content of Vakil's Lemma~\cite[Lemma~5.9]{Vak}, and a detailed proof may be found in~\cite[Theorem~4.3]{RSPW}.

We come to logarithmic smoothness. Assume first that the elliptic curve maps into higher level. Note that the relative logarithmic tangent bundle of the expansion, over the base $\mathbb P^N$ is a trivial line bundle of rank $1$, as the fibers are toric. Consider an $S$-family of maps. After replacing the source $C$ by a destabilization, we have maps
\[
C\to \overline{C}_1\to\overline{C}_2\to \mathbb P^N[s]\to \mathbb P^N
\]
factorizing completely. Let $f_B$ and $f_T$ be the maps from $C$ to the base and total space respectively. Examining the morphism $\nu$, we see that there is a map
\[
\mathrm{Def}(C,f_B)\to \mathrm{Obs}(f_T) = H^1(\overline{C}_1,\mathcal O_{\overline{C}_1}),
\]
where the former is the space of deformations of the curve and map to the base, as a factorized centrally aligned map. The latter is the space of obstructions to lifting a map to the base into the total space. The cokernel of this map are the absolute obstructions, which we will show vanishes, proving logarithmic smoothness.

We recall how the group $H^1(\overline{C}_1,\mathcal O_{\overline{C}_1})$ functions as the obstruction group for the lifting. It suffices to work near the minimal grnus $1$ subcurve $D_1$ of $C_1$, since the rest of the curve is rational. This lifting is given by a rational function with prescribed orders of poles given by the slopes of the tropicalization map. That is, if $\alpha$ be the piecewise linear function giving this tropical map, then the lifting is described by a section of the associated bundle $\mathcal O_{C_1}(-\alpha)$.

Given a strict square-zero extension $S'$ of $S$, the piecewise linear function $\alpha$ extends uniquely by strictness to any deformation of the curve. The resulting deformation of $\mathcal O_{C_1}(-\alpha)$ produces an infinitesimal deformation of the trivial bundle in the Picard group, whose class in $H^1(\overline{C}_1,\mathcal O_{\overline{C}_1})$, is the obstruction to deformation. We will show that it is possible to choose a deformation of $(C,f_B)$ that cancels out the obstruction to lifting to $f_T$, and thus the map to the obstruction space above is surjective. 

The line bundle $\mathcal O_{C_1}(-\alpha)$ is equivalent to the divisor $\sum a_i x_i$ where $x_i$ are the points connecting $D_1$ to the rest of the curve, and $a_i$ are the slopes of $\alpha$ along the edges corresponding to the $x_i$. Infinitesimally moving the point $x_i$ is a deformation of the curve that is unobstructed by the map $f_B$, since the map is constant on the interiors of the circuits. By deforming the $x_i$, at least one of which is nonzero, we produce a one-dimensional space of obstructions. The absolute obstructions therefore vanish, since they are the cokernel of a surjective map, so the result follows.  
%Consider an $S$-family of expanded maps $[F: C\to \mathbb P^N[s]$. Choose $N$ generic hyperplanes in and let $\Delta = \{H\}\cup\{H_1,\ldots,H_N}$ be this set of hyperplanes. There is a morphism of logarithmic schemes $(\mathbb P^N,\Delta)\to (\mathbb P^N,H)$. This induces a new expansion $\mathbb P^N[\Delta,s]$. Moreover, by the genericity of the $H_i$, they each intersect the image of $C$ in finitely many reduced points. Pulling back the logarithmic structure, we obtain a new logarithmic map 
%\[
%F': C'\to \mathbb P^N[\Delta,s].
%\]
%We make two observations. First, since the logarithmic structure on $C'$ is strict away from $H$, unobstructedness of deformations for $F'$ is equivalent to unobstructedness for $F$. Second, the target degeneration is now a toric degeneration, and the logarithmic tangent bundle is trivial.

\end{proof}

The above theorem guarantees that the singularities of the space of totally factorized maps to expansions is logarithmically smooth. In fact, one can say more in this case.

\begin{cor}
The logarithmically smooth stack $\mathcal{VZ}_{1,\alpha}(\mathbb P^N|H)$ has at worst orbifold singularities, and consequently admits a non-representable cover by a smooth Deligne--Mumford stack.
\end{cor}

\begin{proof}
Since $\mathcal{VZ}_{1,\alpha}(\mathbb P^N|H)$ is logarithmically smooth, it will suffice to show that the cones of its tropicalization are simplicial. To see this, consider a logarithmic stable map to an expansion $C\to \mathbb P^N[s]$ without a central alignment. The tropical moduli cone obtained as the dual of the minimal base monoid can be identified with $\mathbb R_{\geq 0}^{s}$, see for instance~\cite[Section~2.2]{ChenDegeneration}. The alignment procedure is an iterated barycentric subdivision at the level of tropical moduli spaces, as explained in~\cite[Section 4.6]{RSPW}, and such subdivisions preserve the property of being simplicial. We conclude from this that the blowup $\widetilde{\mathcal{VZ}}^{\mathrm{exp}}_{1,\alpha}(\mathbb P^N|H)$ obtained by centrally aligning Kim's spaces has simplicial cones. Finally, the morphism $\mathcal{VZ}_{1,\alpha}(\mathbb P^N|H)\to \widetilde{\mathcal{VZ}}^{\mathrm{exp}}_{1,\alpha}(\mathbb P^N|H)$ is strict, so the cones of $\mathcal{VZ}_{1,\alpha}(\mathbb P^N|H)$ are simplicial, as claimed.
\end{proof}

\subsection{Stratification} The logarithmic smoothness established in Theorem~\ref{thm: log-smoothness} allows us to index the strata of 

\subsection{Rigidified rubber and variants}


\bibliographystyle{siam}
\bibliography{Bibliography.bib}

\end{document}
