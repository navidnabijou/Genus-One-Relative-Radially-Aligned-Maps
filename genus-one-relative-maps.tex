\documentclass[11pt]{amsart}
%\usepackage[english]{babel}
\usepackage{appendix}
\usepackage{amsmath}
\usepackage{amsfonts}
\usepackage{amssymb}
%\usepackage{showlabels}
\usepackage{hyperref}
\usepackage{amsthm}
\usepackage{marginnote}
\usepackage{stmaryrd}
\usepackage{enumitem}
\usepackage[english]{babel}
\usepackage{yfonts}
\usepackage[T1]{fontenc}
\usepackage[utf8x]{inputenc}
%\usepackage{enumerate}
\usepackage{verbatim}
\usepackage{graphicx}
\usepackage{verbatim}
\usepackage{faktor}
\usepackage{xcolor}
\usepackage{xfrac}
\usepackage{tikz,tikz-cd}
\usetikzlibrary{decorations.pathmorphing,decorations.pathreplacing,patterns}
\usepackage[all]{xy}
\usepackage{bbm}
\usepackage{tabularx}
\usepackage{longtable}
\usepackage{tabu}
\usepackage{booktabs}
\usepackage{mathtools}

\newcommand{\TT}{\operatorname{T}}
\newcommand{\M}[4]{\overline{\mathcal{M}}_{#1,#2}(#3,#4)}
\newcommand{\Q}[4]{\mathcal{Q}_{#1,#2}(#3,#4)}
\newcommand{\Qe}[4]{\mathcal{Q}^{\epsilon}_{#1,#2}(#3,#4)}
\newcommand{\Qt}[4]{\widetilde{\mathcal Q}_{#1,#2}(#3,#4)}
\newcommand{\QG}[4]{\mathcal{Q}G_{#1,#2}(#3,#4)}
\newcommand{\QGe}[4]{\mathcal{Q}G^{\epsilon}_{#1,#2}(#3,#4)}
\newcommand{\D}[3]{\mathcal{D^Q}(#1,#2,#3)}
\newcommand{\E}[3]{\mathcal{E^Q}(#1,#2,#3)}
\newcommand{\PP}{\mathbb P}
\newcommand{\Z}{\mathbb{Z}}
\newcommand{\N}{\mathbb{N}}
\newcommand{\OO}{\mathcal{O}}
\renewcommand{\to}{\rightarrow}
\newcommand{\A}{\mathcal A}
\newcommand{\B}{\mathcal B}
\newcommand{\C}{\mathfrak C}
\newcommand{\EE}{\mathbf{E}}
\renewcommand{\L}{\mathcal L}
\newcommand{\LL}{\mathbf{L}}
\newcommand{\MM}{\mathfrak M}
\newcommand{\Aaff}{\mathbb{A}}
\newcommand{\kfield}{\Bbbk}
\newcommand{\comp}{\chi}
\newcommand{\sst}{\sigma^{\operatorname{ss}}}
\newcommand{\Pic}{\operatorname{Pic}}
\newcommand{\Def}{\operatorname{Def}}
\newcommand{\Spec}{\operatorname{Spec}}
\newcommand{\Proj}{\operatorname{Proj}}
\newcommand{\Hom}{\operatorname{Hom}}
\newcommand{\Ext}{\operatorname{Ext}}
\newcommand{\Gm}{\mathbb{G}_{\text{m}}}
\newcommand{\virt}[1]{[#1]^{\operatorname{virt}}}
\newcommand{\vip}[1]{[#1]^{\operatorname{prod}}}
\newcommand{\Id}{\operatorname{Id}}
\newcommand{\CC}{\mathbb{C}}
\newcommand{\QQ}{\mathbb{Q}}
\newcommand{\HH}{\operatorname{H}}
\newcommand{\Achow}{\operatorname{A}}
\newcommand{\pt}{\operatorname{pt}}
\newcommand{\bq}{\begin{equation}}
\newcommand{\eq}{\end{equation}}
\newcommand{\ba}{\begin{aligned}}
\newcommand{\ea}{\end{aligned}}
\newcommand{\be}{\begin{enumerate}}
\newcommand{\ee}{\end{enumerate}}
\newcommand{\bsm}{\left(\begin{smallmatrix}}
\newcommand{\esm}{\end{smallmatrix}\right)}                   
\newcommand{\bpm}{\begin{pmatrix}}
\newcommand{\epm}{\end{pmatrix}}
\newcommand{\barr}{\begin{displaymath}\begin{array}{cccc}}
\newcommand{\earr}{\end{array}\end{displaymath}}
\newcommand{\barrl}{\begin{displaymath}\begin{array}{lcl}}
\newcommand{\earrl}{\end{array}\end{displaymath}}
\newcommand{\barl}{\begin{displaymath}\begin{array}{l}}
\newcommand{\earl}{\end{array}\end{displaymath}}
\newcommand{\bxym}{ \begin{displaymath}\xymatrix }
\newcommand{\exym}{\end{displaymath}}
\newcommand{\bcd}{\begin{center}\begin{tikzcd}}
\newcommand{\ecd}{\end{tikzcd}\end{center}}
\newcommand{\R}{\operatorname{R}^{\bullet}}
%\newcommand{\sslash}{\mathbin{/\mkern-6mu/}}
\newcommand{\tr}{{\rm tr}}
\newcommand{\Isom}{\text{Isom}}
\newcommand{\pr}{\operatorname{pr}}
\newcommand{\ev}{\operatorname{ev}}
\newcommand{\codim}{\operatorname{codim}}
\newcommand{\vdim}{\operatorname{vdim}}
\newcommand{\ildef}[1]{\emph{#1}}
\newcommand{\om}[1]{\mathcal{#1}}
\newcommand{\h}{\operatorname{h}}
\newcommand{\Aut}{\operatorname{Aut}}
\newcommand{\RR}{\textbf{R}}
\newcommand{\NN}{\operatorname{N}}


\theoremstyle{definition}
\newtheorem{thm}{Theorem}[section]
\newtheorem{lem}[thm]{Lemma}
\newtheorem{lemma}[thm]{Lemma}
\newtheorem{prop}[thm]{Proposition}
\newtheorem{cor}[thm]{Corollary}
\newtheorem*{teo*}{Theorem}
\newtheorem{ipotesi}{ipotesi}
\newtheorem*{nota}{Nota}
\newtheorem{claim}{Claim}
\newtheorem{question}[thm]{Question}
\newtheorem{conj}[thm]{Conjecture}

\newtheorem{innercustomthm}{Theorem}
\newenvironment{customthm}[1]
  {\renewcommand\theinnercustomthm{#1}\innercustomthm}
  {\endinnercustomthm}

\theoremstyle{definition}
\newtheorem{example}[thm]{Example}
\newtheorem{ex}[thm]{Example}
\newtheorem{dfn}[thm]{Definition}
\newtheorem{definition}[thm]{Definition}
\newtheorem{aside}[thm]{Aside}
\newtheorem{remark}[thm]{Remark}
\newtheorem{com}[thm]{Comment}
\newtheorem{num}{Number}
\newtheorem*{sketch}{Sketch}
\newtheorem*{rem}{Remark}
\newtheorem*{aside*}{Aside}
\newtheorem*{acknowledgements}{Acknowledgements}

\newcommand{\ilemph}[1]{\emph{#1}}

\setcounter{tocdepth}{1}

\newcommand{\todo}[1]{\vspace{5mm}\par \noindent
\framebox{\begin{minipage}[c]{0.95 \textwidth} \tt #1\end{minipage}} \vspace{5mm} \par}

\def\ti{-\allowhyphens}
\newcommand{\thismonth}{\ifcase\month % case 0 --- impossible!
  \or January\or February\or March\or April\or May\or June%
  \or July\or August\or September\or October\or November%
  \or December\fi}
\newcommand{\thismonthyear}{{\thismonth} {\number\year}}
\newcommand{\thisdaymonthyear}{{\number\day} {\thismonth} {\number\year}}

\usepackage[T1]{fontenc}
\usepackage{newpxtext,newpxmath}

\title[Genus One Radial Maps]{Relative Stable Maps in Genus One via Radial Alignments}
\author{Luca Battistella and Navid Nabijou}
\begin{document}

%\begin{abstract} \end{abstract}

\maketitle
%\appendixtitletocoff
%\tableofcontents

\section{Relative space equals closure of the nice locus}
Since the log structures only come into play when the source curve is reducible, it follows that the nice locus in the radially aligned setting is the same as the nice locus in the ordinary setting. In particular, it is irreducible.

We now want to show that the relative space in the radially aligned setting is equal to the closure of the nice locus; irreducibility follows immediately. One direction is clear: [WHY?]

It thus remains to show that, given a relative radially aligned map, we can smooth it to one in the nice locus. This is done by considering different cases locally, then gluing.

\subsection*{Case 1: non-contracted genus one internal component} Assume that the curve takes the form
\begin{equation*} C = C_0 \cup C_1 \cup \ldots \cup C_k \end{equation*}
where all the $C_i$ are smooth, $C_0$ has genus one, all the other $C_i$ have genus zero, and for $i \in \{1,\ldots,k\}$, $C_i$ intersects $C_0$ at a single node (denoted $q_i$) and does not intersect any other components.

Suppose furthermore that $C_0$ is a non-contracted \emph{internal component}, meaning that it is mapped into $H$ via $f$, and that $C_1$,\ldots,$C_k$ are \emph{external components}, meaning that they are not mapped into $H$ via $f$. The picture is:

[FIGURE]

Suppose that this is a relative stable map. This means that [BLAH]. We claim that it can be smoothed to a relative stable map in the nice locus. The construction depends on choosing an appropriate smoothing of the curve $C$, so that the map also smooths.

We start with $W = C_0 \times \Aaff^1_t$ (where $t$ denotes a fixed co-ordinate on the affine line). This is a smooth surface, fibred over $\Aaff^1_t$, with fibre equal to the elliptic curve $C_0$. Consider the points $q_1, \ldots, q_k$ on $C_0$. We will perform a series of weighted blow-ups at the points $(q_i,0) \in W$, in order to obtain a surface whose general fibre is smooth (in fact, isomorphic to $C_0$) and whose central fibre is isomorphic to $C$.

Fix $i \in \{1,\ldots,k\}$ and let $m_i$ be the multiplicity of $f$ with $H$ at $q_i \in C_i$. We define:
\begin{equation*} l = \operatorname{lcm}(m_1,\ldots,m_k) \qquad r_i = l/m_i \end{equation*}
We now blow-up the surface $W$ at the points $(q_i,0)$ with weight $r_i$ in the horizontal direction and weight $1$ in the vertical direction: if $x_i$ is a local co-ordinate for the fibre around $q_i$, this means that we blow-up in the ideal $(x_i,t^{r_i})$.

The result is a fibred surface $W^\prime \to \Aaff^1_t$ with general fibre isomorphic to $C_0$ and central fibre isomorphic to $C$. The total space of $W^\prime$ is no longer smooth, but this is fine; the projection to $\Aaff^1_t$ is still flat. The central fibre is a linearly trivial Cartier divisor:
\begin{equation*} W^\prime_0 = C_0 + C_1 + \ldots + C_k = 0 \in \Pic W^\prime \end{equation*}
Fix $i \in \{1,\ldots,k\}$; then $r_i C_i$ is Cartier, although $C_i$ may not necessarily be. Furthermore, since
\begin{equation*} l C_0 = - \sum_{i=1}^k l C_i = - \sum_{i=1}^k m_i (r_i C_i) \end{equation*}
in $A_1(W^\prime)$, it follows that $lC_0$ is Cartier. Finally, a local computation shows that
\begin{equation*} r_i C_i \cdot C_0 = 1 \end{equation*}
for $i \in \{1,\ldots,k\}$. Now, let $x_1,\ldots,x_n$ denote the marked points of $C$. These are smooth points of the central fibre $W^\prime_0$, and hence can be extended to Cartier divisors $\tilde{x}_1,\ldots,\tilde{x}_n$ on $W^\prime$. Consider the line bundle:
\begin{equation*} \tilde{L} = \OO_{W^\prime}(l C_0 + \Sigma_{j=1}^n \alpha_j \tilde{x}_j) \end{equation*}
on $W^\prime$. We claim that this gives a smoothing of the line bundle $L=f^*\OO(1)$ on $C$, i.e. that $\tilde{L}|_{W^\prime_0} = L$. We show this by first restricting $\tilde{L}$ to each of the components $C_i$ of $W^\prime_0 \cong C$. For $i \in \{1,\ldots,k\}$, we have
\begin{align*} \tilde{L}|_{C_i} & = \OO_{C_i} \left( (l C_0 \cdot C_i) q_i + \sum_{x_j \in C_i} \alpha_j x_j \right) = \OO_{C_i} \left( (l/r_i) q_i + \sum_{x_j \in C_i} \alpha_j x_j \right)\\
& = \OO_{C_i} \left( m_i q_i + \sum_{x_j \in C_i} \alpha_j x_j \right) = L|_{C_i} \end{align*}
while for $i=0$ we have:
\begin{align*} \tilde{L}|_{C_0} & = \OO_{C_0} \left( - \sum_{i=1}^k (l C_i \cdot C_0) q_i + \sum_{x_j \in C_0} \alpha_j x_j \right) = \OO_{C_0} \left( - \sum_{i=1}^k m_i q_i + \sum_{x_j \in C_0} \alpha_j x_j \right)  = L|_{C_0} \end{align*}
Finally the fact that $\tilde{L}|_{W^\prime_0} = L$ follows from the fact that the dual intersection graph of $C$ has genus zero.

Now, $\tilde{L}$ comes with a unique section whose restriction to $W^\prime_0 \cong C$ is $s_0$. After we extend the sections $s_1,\ldots,s_N$, it is clear that the resulting stable map is in the nice locus (i.e. that it is not mapped into $H$).

Now, to extend the sections $s_1,\ldots,s_N$, we simply check that they are unobstructed. The space containing the obstructions is:
\begin{equation*} \HH^1(C,L) \end{equation*} \marginpar{Is this true even when $C$ is reducible}
By taking the normalisation exact sequence for $C$, tensoring with $L$ and passing to cohomology, we obtain:

\newpage
\bibliographystyle{alpha}
\bibliography{relqm}

\bigskip\bigskip

\noindent Luca Battistella\\
Department of Mathematics, Imperial College London \\
\texttt{l.battistella14@imperial.ac.uk}\\

\noindent Navid Nabijou \\
Department of Mathematics, Imperial College London \\
\texttt{navid.nabijou09@imperial.ac.uk}



\end{document}