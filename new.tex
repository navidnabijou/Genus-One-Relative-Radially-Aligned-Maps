\documentclass[11pt]{amsart}

\usepackage[english]{babel}
\usepackage{appendix}
\usepackage{amsmath}
\usepackage{amsfonts}
\usepackage{amssymb}
%\usepackage{showlabels}
\usepackage{hyperref}
\usepackage{amsthm}
\usepackage{marginnote}
\usepackage{stmaryrd}
\usepackage{enumitem}
\usepackage[english]{babel}
\usepackage{yfonts}
\usepackage[T1]{fontenc}
\usepackage[utf8x]{inputenc}

\usepackage{calrsfs}
\DeclareMathAlphabet{\pazocal}{OMS}{zplm}{m}{n}

\usepackage{verbatim}
\usepackage{graphicx}
\usepackage{verbatim}
\usepackage{faktor}
\usepackage{xcolor}
\usepackage{xfrac}
\usepackage{tikz,tikz-cd}
\usetikzlibrary{decorations.pathmorphing,decorations.pathreplacing,patterns}

\usepackage[all]{xy}
\usepackage{bbm}
\usepackage{tabularx}
\usepackage{longtable}
\usepackage{tabu}
\usepackage{booktabs}
\usepackage{mathtools}

\usepackage[]{textcomp}
\usepackage[sups]{Baskervaldx}
\usepackage{cabin}
\usepackage[varqu,varl]{inconsolata}
\usepackage[baskervaldx,bigdelims,vvarbb]{newtxmath}
\usepackage[cal=cm]{mathalfa}


\newcommand{\plC}{\scalebox{0.8}[1.3]{$\sqsubset$}}
\newcommand{\sidenote}[1]{\marginpar{\textbf{\color{red}#1}}}

\def\Yagraph{\tikz[baseline=-3pt,scale=.8]{
\draw (2,0) -- (0,1) (2,0) -- (0,.5) (2,0) -- (0,-1);
\draw (2,0) circle(2pt)[fill=black];
\draw (0,1) circle(2pt)[fill=white];
\draw (0,.5) circle(2pt)[fill=black];
\draw (0,-.5) node{$\ldots$};
\draw (0,-1) circle(2pt)[fill=black];
\draw (-1,1.5) -- (1,1.5) -- (1,-1.5) -- (-1,-1.5) -- (-1,1.5);
\draw (1,1.5) -- (3,2) -- (3,-1) -- (1,-1.5);
}}

\def\Ybgraph{\tikz[baseline=-3pt,scale=.8]{
\draw (2,0) to[out=120,in=0] (0,1) (2,0) -- (0,1) (2,0) -- (0,.5) (2,0) -- (0,-1);
\draw (2,0) circle(2pt)[fill=black];
\draw (0,1) circle(2pt)[fill=black];
\draw (0,.5) circle(2pt)[fill=black];
\draw (0,-.5) node{$\ldots$};
\draw (0,-1) circle(2pt)[fill=black];
\draw (-1,1.5) -- (1,1.5) -- (1,-1.5) -- (-1,-1.5) -- (-1,1.5);
\draw (1,1.5) -- (3,2) -- (3,-1) -- (1,-1.5);
}}

\def\Ycgraph{\tikz[baseline=-3pt,scale=.8]{
\draw (2,0) -- (0,1) (2,0) -- (0,.5) (2,0) -- (0,-1);
\draw (2,0) circle(2pt)[fill=white];
\draw (0,1) circle(2pt)[fill=black];
\draw (0,.5) circle(2pt)[fill=black];
\draw (0,-.5) node{$\ldots$};
\draw (0,-1) circle(2pt)[fill=black];
\draw (-1,1.5) -- (1,1.5) -- (1,-1.5) -- (-1,-1.5) -- (-1,1.5);
\draw (1,1.5) -- (3,2) -- (3,-1) -- (1,-1.5);
}}

\newcommand{\pM}{\pazocal{M}}
\newcommand{\TT}{\operatorname{T}}
\newcommand{\oM}{\overline{\mathcal{M}}}
\newcommand{\M}[4]{\overline{\mathcal{M}}_{#1,#2}(#3,#4)}
\newcommand{\Q}[4]{\mathcal{Q}_{#1,#2}(#3,#4)}
\newcommand{\Qe}[4]{\mathcal{Q}^{\epsilon}_{#1,#2}(#3,#4)}
\newcommand{\Qt}[4]{\widetilde{\mathcal Q}_{#1,#2}(#3,#4)}
\newcommand{\QG}[4]{\mathcal{Q}G_{#1,#2}(#3,#4)}
\newcommand{\QGe}[4]{\mathcal{Q}G^{\epsilon}_{#1,#2}(#3,#4)}
\newcommand{\D}[3]{\mathcal{D^Q}(#1,#2,#3)}
\newcommand{\E}[3]{\mathcal{E^Q}(#1,#2,#3)}
\newcommand{\PP}{\mathbb P}
\newcommand{\Z}{\mathbb{Z}}
\newcommand{\VZ}{\pazocal{V\!Z}}
\newcommand{\tVZc}[4]{\widetilde{\mathcal{V\!Z}}^{\rm{ctr}}_{#1,#2}(#3,#4)}
\newcommand{\VZc}[4]{\mathcal{V\!Z}^{\rm{ctr}}_{#1,#2}(#3,#4)}
\newcommand{\VZcLi}[4]{\mathcal{V\!Z}^{\rm{ctr, Li}}_{#1,#2}(#3,#4)}
\newcommand{\VZrel}[4]{\mathcal{V\!Z}^{\rm{rel}}_{#1,#2}(#3,#4)}
\newcommand{\stab}{\rm{stab}}
\newcommand{\stC}{C'}
\newcommand{\stf}{f'}
\newcommand{\stpi}{\pi'}
\newcommand{\sts}{s'}
\newcommand{\N}{\mathbb{N}}
\newcommand{\OO}{\mathcal{O}}
\renewcommand{\to}{\rightarrow}
\newcommand{\A}{\mathcal A}
\newcommand{\B}{\mathcal B}
\newcommand{\C}{\mathfrak C}
\newcommand{\cC}{\mathcal C}
\newcommand{\EE}{\mathbf{E}}
\renewcommand{\L}{\mathcal L}
\newcommand{\LL}{\mathbf{L}}
\newcommand{\MM}{\mathfrak M}
\newcommand{\Aaff}{\mathbb{A}}
\newcommand{\kfield}{\Bbbk}
\newcommand{\comp}{\chi}
\newcommand{\sst}{\sigma^{\operatorname{ss}}}
\newcommand{\Pic}{\operatorname{Pic}}
\newcommand{\Def}{\operatorname{Def}}
\newcommand{\Spec}{\operatorname{Spec}}
\newcommand{\Proj}{\operatorname{Proj}}
\newcommand{\Hom}{\operatorname{Hom}}
\newcommand{\Ext}{\operatorname{Ext}}
\newcommand{\Gm}{\mathbb{G}_{\text{m}}}
\newcommand{\virt}[1]{[#1]^{\operatorname{virt}}}
\newcommand{\vip}[1]{[#1]^{\operatorname{prod}}}
\newcommand{\Id}{\operatorname{Id}}
\newcommand{\CC}{\mathbb{C}}
\newcommand{\QQ}{\mathbb{Q}}
\newcommand{\HH}{\operatorname{H}}
\newcommand{\Achow}{\operatorname{A}}
\newcommand{\pt}{\operatorname{pt}}
\newcommand{\bq}{\begin{equation}}
\newcommand{\eq}{\end{equation}}
\newcommand{\ba}{\begin{aligned}}
\newcommand{\ea}{\end{aligned}}
\newcommand{\be}{\begin{enumerate}}
\newcommand{\ee}{\end{enumerate}}
\newcommand{\bsm}{\left(\begin{smallmatrix}}
\newcommand{\esm}{\end{smallmatrix}\right)}                   
\newcommand{\bpm}{\begin{pmatrix}}
\newcommand{\epm}{\end{pmatrix}}
\newcommand{\barr}{\begin{displaymath}\begin{array}{cccc}}
\newcommand{\earr}{\end{array}\end{displaymath}}
\newcommand{\barrl}{\begin{displaymath}\begin{array}{lcl}}
\newcommand{\earrl}{\end{array}\end{displaymath}}
\newcommand{\barl}{\begin{displaymath}\begin{array}{l}}
\newcommand{\earl}{\end{array}\end{displaymath}}
\newcommand{\bxym}{ \begin{displaymath}\xymatrix }
\newcommand{\exym}{\end{displaymath}}
\newcommand{\bcd}{\begin{center}\begin{tikzcd}}
\newcommand{\ecd}{\end{tikzcd}\end{center}}
\newcommand{\R}{\operatorname{R}^{\bullet}}
\newcommand{\dvr}{\Delta}
%\newcommand{\sslash}{\mathbin{/\mkern-6mu/}}
\newcommand{\tr}{{\rm tr}}
\newcommand{\Isom}{\text{Isom}}
\newcommand{\pr}{\operatorname{pr}}
\newcommand{\ev}{\operatorname{ev}}
\newcommand{\codim}{\operatorname{codim}}
\newcommand{\vdim}{\operatorname{vdim}}
\newcommand{\ildef}[1]{\emph{#1}}
\newcommand{\om}[1]{\mathcal{#1}}
\newcommand{\h}{\operatorname{h}}
\newcommand{\Aut}{\operatorname{Aut}}
\newcommand{\RR}{\textbf{R}}
\newcommand{\NN}{\mathbb{N}}
\newcommand{\ovm}[1]{\overline{\mathcal{#1}}}
\newcommand{\ovt}[1]{\widetilde{\mathcal{#1}}}
\newcommand{\ov}[1]{\overline{#1}}

\theoremstyle{definition}
\newtheorem{thm}{Theorem}[section]
\newtheorem{lem}[thm]{Lemma}
\newtheorem{lemma}[thm]{Lemma}
\newtheorem{prop}[thm]{Proposition}
\newtheorem{cor}[thm]{Corollary}
\newtheorem*{teo*}{Theorem}
\newtheorem{ipotesi}{ipotesi}
\newtheorem*{nota}{Nota}
\newtheorem{claim}{Claim}
\newtheorem{question}[thm]{Question}
\newtheorem{conj}[thm]{Conjecture}

\newtheorem{innercustomthm}{Theorem}
\newenvironment{customthm}[1]
  {\renewcommand\theinnercustomthm{#1}\innercustomthm}
  {\endinnercustomthm}

\theoremstyle{definition}
\newtheorem{example}[thm]{Example}
\newtheorem{ex}[thm]{Example}
\newtheorem{dfn}[thm]{Definition}
\newtheorem{definition}[thm]{Definition}
\newtheorem{aside}[thm]{Aside}
\newtheorem{remark}[thm]{Remark}
\newtheorem{com}[thm]{Comment}
\newtheorem{num}{Number}
\newtheorem*{sketch}{Sketch}
\newtheorem*{rem}{Remark}
\newtheorem*{aside*}{Aside}
\newtheorem*{acknowledgements}{Acknowledgements}

\newcommand{\ilemph}[1]{\emph{#1}}

\setcounter{tocdepth}{1}

\newcommand{\todo}[1]{\vspace{5mm}\par \noindent
\framebox{\begin{minipage}[c]{0.95 \textwidth} \tt #1\end{minipage}} \vspace{5mm} \par}

\def\ti{-\allowhyphens}
\newcommand{\thismonth}{\ifcase\month % case 0 --- impossible!
  \or January\or February\or March\or April\or May\or June%
  \or July\or August\or September\or October\or November%
  \or December\fi}
\newcommand{\thismonthyear}{{\thismonth} {\number\year}}
\newcommand{\thisdaymonthyear}{{\number\day} {\thismonth} {\number\year}}

\title[Genus One Reduced Relative Invariants]{Relative Stable Maps in Genus One via Central Alignments}
\author{Luca Battistella, Navid Nabijou and Dhruv Ranganathan}
\date{\thismonthyear}

\begin{document}


\begin{abstract} For a smooth projective variety $X$ and a smooth very ample hypersurface $Y \subseteq X$, we define moduli spaces of relative stable maps to $(X,Y)$ in genus one, as closed substacks of the moduli space of maps from centrally aligned curves, constructed in \cite{RSPW}. We construct virtual classes for these moduli spaces, which we use to define \emph{reduced relative Gromov--Witten invariants} in genus one.

[GOALS: We prove a recursion formula which allows us to completely determine these invariants in terms of the reduced Gromov--Witten invariants, as defined in [REF]. We also prove a relative version of the Li--Zinger formula, relating our invariants to the usual relative Gromov--Witten invariants. Also say something about quasimaps.]
\end{abstract}

\maketitle

\appendixtitletocoff
\tableofcontents

\section{Introduction}

\textbf{Statement of the problem.} Contrary to the genus zero case, the moduli space of genus one maps to projective space - with or without markings - is far from smooth; indeed it has various boundary components of different dimensions, representing maps that contract a genus one curve and have all the degree supported on a number of rational tails. The many incarnations of relative moduli spaces also suffer of the same undesirable feature.

Since the work of Vakil--Zinger and Ranganathan--Santos-Parker--Wise, it has been clear that it is possible to identify a desingularisation of the main component by adding the extra data of a contraction of the source curve $\nu\colon C\to \bar{C}$ - where the latter is allowed to acquire a Smyth singularity - and requiring the stable map $f\colon C\to \PP^N$ to factor through $\nu$.

\textbf{Choice of relative space and desingularisation.} We focus on the space of logarithmic stable maps to $(\PP^N|H)$, following ACGS. We perform a log modification of this space as detailed below. For a log curve $C\to S=\Spec(k=\bar k)$, modify the dual graph of $C$ by replacing the minimal genus one subcurve (in case it is a circle of $\PP^1$) by a single vertex of genus one, called the \emph{core} and denoted by $\circ$, and define a piecewise linear function with values in $\overline{\pazocal M}_S$ on such a graph by setting \[\lambda(v)=\sum_{q\in[\circ,v]}\rho_q,\]
where the $\rho_q$ are the smoothing parameters of the nodes $q$ separating $v$ from the core. Such a function is related to the log canonical bundle of $C\to S$. When the map contracts a subcurve of genus one, we endow it with the extra data of a radius $\delta\in\overline{\pazocal M}_S$ subject to the following compatibility condition:

(*) \emph{the circle of radius $\delta$ around $\circ$ passes through $\geq1$ vertex of positive $f$-degree}.

\noindent Furthermore, we require all the values of $\lambda$ to be comparable with $\delta$, and among themselves whenever they are $\leq \delta$. This is called a \emph{centrally aligned} log structure and carries enough information to define a contraction $\nu\colon C\to \bar{C}$, possibly after a semistabilisation of $(C,f)$ - in fact even more. The space thus obtained, $\widetilde \VZ_{1,\alpha}(\PP^N|H,d)$, is a log modification of $\M{1}{\alpha}{\PP^N|H}{d}$.

The main component $\VZ_{1,\alpha}(\PP^N|H,d)$ is then identified by a double factorisation condition:
\begin{enumerate}
 \item If $f$ contracts a genus one subcurve, then $f$ is required to factor through the Smyth singularity $\nu\colon C\to \bar C$ determined by the contraction radius $\delta$ as above.
 \item If furthermore the core is contracted by the associated tropical map $\phi$, let $\delta_2$ be the minimal distance from $\circ$ to a vertex supporting a flag that escapes $\phi^{-1}(\phi(\circ))$; we require $f$ to factor through $\nu_2\colon C\to\bar C_2$.
\end{enumerate}
The main result is that
\begin{thm}
 $\VZ_{1,\alpha}(\PP^N|H,d)$ is (log) smooth.
\end{thm}

\textbf{Gathmann-type recursion.} There is a forgetful morphism \[\VZ_{1,\alpha}(\PP^N|H,d)\to \VZ_{1,n}(\PP^N,d),\] hitting Gathmann's relative space. We may therefore pullback Gathmann's line bundle and section, cutting out the locus where the $k$-th marking is tangent to $H$ to order $\alpha_k+1$. Because $\alpha$ was maximal ($\sum\alpha=d$) by assumption, this means that the curve has to break, and $x_k$ has to lie on an internal component - one which is entirely mapped into $H$. We identify the zero locus of Gathmann's section explicitly. Here is an interesting remark: the combinatorics of such boundary loci is governed by tropical geometry, and it is otherwise very hard to districate the interaction between the relative condition and the exceptional loci of the Vakil--Zinger blow-up.
\begin{cor}
 $\VZ_{1,\alpha}(\PP^N|H,d)$ is smooth over its Artin fan, in particular codimension one logarithmic strata can be read off from the latter.\marginpar{why are we only interested in log strata?}
\end{cor}
The Artin fan is a tropical gadget. Its local structure is given by subdividing the ACGS minimal monoid according to the alignment. We are only interested in picking its rays. The upshot is that the combinatorics is slightly more involved than in the genus zero case: the alignment may force some teeth of the comb to break.
\begin{thm}[Gathmann-type formula, maximal tangency, $(\PP^N|H)$ case]
 \[(\alpha_k\psi_k+\ev_k^*H)[\VZ_{1,\alpha}(\PP^N|H,d)]=[D_{1,\alpha;k}(\PP^N|H,d)],\]
 the latter being a sum of broken comb loci indexed by rays of the tropical fan.
\end{thm}
Importantly, the broken comb loci admit a very explicit description in terms of tautological integrals on the underlying boundary of Gathmann's relative space.
\begin{thm}
 Up to a finite cover of the underlying boundary stratum - which is a combinatorially-determined fiber product of moduli spaces of genus zero and one, absolute and relative maps with lower numerical invariants - every component of $D_{\alpha,k}(\PP^N|H,d)$ can be described as the transverse intersection of two loci in a projective bundle, where:
 \begin{itemize}
  \item the latter parametrises the possible line bundle isomorphisms imposed by the alignment of the log structure;
  \item the first locus is a subbundle representing the residual isomorphisms after fixing the ones virtually imposed by tropical continuity;
  \item the second locus is determined by the factorisation conditions.
 \end{itemize}
\end{thm}
The upshot is that we may then push the formula down to the Gathmann's space, so as to obtain multiplicities and tautological classes.
\begin{cor}
 \[(\alpha_k\psi_k+\ev_k^*H)[\VZ^G_{1,\alpha}(\PP^N|H,d)]=[D^G_{1,\alpha;k}(\PP^N|H,d)],\]
 the latter being expressible as a weighted sum of tautological classes on Gathmann's comb loci.
\end{cor}
Once we have this formula, the following extensions are classical:
\begin{itemize}
 \item A similar formula for raising the tangency holds in the non-maximal tangency case. It can be proven by adding auxiliary markings of contact order $1$; forgetting them is then a $(d-\sum\alpha)!:1$ cover because the nice locus is dense inside Gathmann's relative spaces.
 \item The formula holds more generally for any smooth projective target $X$ relative to a generic hyperplane section $Y=X\cap H\subseteq \PP^N$. This follows via virtual pullback.
\end{itemize}
Finally the recursive structure of the boundary allows us to prove the following
\begin{thm}[In-principle quantum Lefschetz]
 The restricted reduced genus one invariants of $Y$ can be inductively deducted from the full descendant genus zero and one (reduced) Gromov-Witten theory of $X$.
\end{thm}
The proof is more delicate than its genus zero analogue because invariants with the same numerical data appear intertwined in the last steps of the recursion.

\section{A desingularisation of the log space}
The ultimate goal of the paper is to apply Gathmann's techniques to the Vakil-Zinger desingularisation $\VZ_{1,n}(\PP^N,d)$ and to obtain a quantum Lefschetz result for reduced invariants under some positivity assumption. The key step is to study the unobstructed case $(\PP^N|H)$. We approach the problem by lifting it to the ACGS space of log stable maps. This allows us to exploit the tools developed in \cite{RSPW,RSPW2}. We are in an intermediate situation between those two papers, and indeed we get an intermediate answer.

\subsection{The ACGS minimal monoid and central alignments}

\begin{prop}
 The map $\oM_{1,\alpha}^{\mathrm{cen}}(\PP^N|H,d)\to\oM_{1,\alpha}(\PP^N|H,d)$ is a log modification. In particular $\oM_{1,\alpha}^{\mathrm{cen}}(\PP^N|H,d)$ is a log algebraic stack.
\end{prop}

{\color{gray} possibly start pointing out that the log structure is already partially aligned by the map to $(\PP^N|H)$}

\subsection{Factorisation conditions}

\begin{prop}
 Factoring through the Smyth curve is a closed condition. In particular $\VZ_{1,\alpha}(\PP^N|H,d)\subseteq_{\mathrm{cl}}\oM_{1,\alpha}^{\mathrm{cen}}(\PP^N|H,d)$ is a log algebraic stack.
\end{prop}

\subsection{Log smoothness}
\begin{thm}
 $\VZ_{1,\alpha}(\PP^N|H,d)$ is log smooth.
\end{thm}
\begin{proof}
 We reduce to the situation dealt with in \cite{RSPW2} by adding generic extra hyperplanes $H_1,\ldots, H_N$.
 
 First, note that, for divisors $D_1\subseteq D_2$ in $X$, there is a morphism of log schemes $(X,\pazocal M_{D_2})\to (X,\pazocal M_{D_1})$, or equivalently a morphism of log structures $\pazocal M_{D_1}\to\pazocal M_{D_2}$ over $i\!d_X$, because functions invertible off $D_1$ are in particular invertible off $D_2$ as well, and divisorial log structures are subsheaves of the structure sheaf $\pazocal M_{D}\subseteq\OO_X$.
 
 Now fix a point $[(C,f)]\in\VZ_{1,\alpha}(\PP^N|H,d)$. Choose hyperplanes $H_1,\ldots, H_N$ in such a way that they intersect the image of $f$ transversally, namely $f^{-1}(H_1\cup\ldots\cup H_N)$ is a reduced collection of points $\{q^i_j\}_{\substack{i=1,\ldots,N \\ j=1,\ldots,d}}$ in the smooth locus of $C$. This condition will then hold in an open neighbourhood of $[(C,f)]$. Mark $C$ at these points, end endow it with the pullback along $f$ of the divisorial log structure $(\PP^N,\Delta)$, where $\Delta=H+\sum_{i=1}^N H_i$. Then \[f\colon (C,\{p_k\}_{k=1,\ldots,n}\{q^i_j\}_{\substack{i=1,\ldots,N \\ j=1,\ldots,d}})\to(\PP^N,\Delta)\]
 is a lift of $[(C,f)]$ to $\oM_{1,\alpha}(\PP^N|\Delta,d)$ (under the forgetful morphism discussed in the previous paragraph).
 
 Looking at the associated tropical map $\phi$, observe that:
 \begin{itemize}
  \item new flags have been attached only to vertices of positive degree, and these already have a flag escaping $\phi^{-1}\phi(v)$, because the sum of the incoming slopes is not zero (by modified balancing);
  \item the image of the new tropical map $\tilde\phi$ is entirely contained in the ray of the tropicalisation of $(\PP^N,\Delta)$ corresponding to $H$, with new flags going off to infinity in all the new ray directions from every vertex of positive degree.
 \end{itemize}
 In particular, for every quotient $N'$ of the lattice $N$, the associated tropical map $\tilde\phi'$ will either
 \begin{enumerate}
  \item have image contained in the ray corresponding to $H$, isomorphically to the original $\phi$, so the contraction radius can be seen to coincide with $\delta_2$, or
  \item collapse the entire curve to the zero-cell of the fan, in which case we argue from the previous remarks that the contraction radius is $\delta$.
 \end{enumerate}

 Hence the lift of $[(C,f)]$ is centrally aligned and satisfies the factorisation property for every subtorus $H<T$, therefore it is well-spaced (see \cite[Definition 3.4.2]{RSPW2}) and it belongs to $\VZ_{1,\tilde\alpha}(\PP^N|\Delta,d)$. Note that the deformation spaces of $(C,f)$ and its lift are isomorphic by construction, as can be checked by the infinitesimal criterion - an infinitesimal deformation of $(C,f)$ brings along a unique deformation of the $\{q^i_j\}$ compatible with the map to $(\PP^N,\Delta)$. At the logarithmic level, observe that the ACGS minimal monoid is the same, because no component of $C$ is entirely mapped into any of the newly added hyperplanes; since the global contraction radius $\delta$ is the same, the subdivisions corresponding to the alignment procedure do coincide as well. This shows that the forgetful morphism is (log) \'etale in a neighbourhood of the lift of $[(C,f)]$, hence we may conclude by appealing to \cite[Theorem 3.5.1]{RSPW2}.
\end{proof}

\begin{cor}\label{cor:log_smooth}
 $\VZ_{1,\alpha}(\PP^N|H,d)$ is smooth over its Artin fan.
\end{cor}

{\color{gray} describe the latter as explicitly as possible; comment on the cones that are (possibly?) not there because of the compatibility of the alignment with the log map (this is probably awkward, useless, and superceded by saying ``we align subdivide the ACGS minimal dual monoid'') and because of smoothability/factorisation (this is probably related to tropical well-spacedness)}

\section{Description of the boundary}

Note that $f$ has contact order with $H$ exactly equal to $\alpha_k$ at the marking $p_k$, or else the irreducible component of $C$ on which $x_k$ lies is mapped entirely inside $H$. By pulling back along $f$ the equation defining $H$, and taking its $\alpha_k+1$-st derivative at $x_k$ (which makes sense because all the lower order derivatives do vanish by assumption) we single out the latter locus. By staring at the exact sequence of jet bundles
\[0\to p_k^*\Omega_{C}^{\otimes\alpha_k}\otimes\ev_k^*\OO_{\PP^N}(H)\to p_k^*\mathcal J^{\alpha_k}(f^*\OO_{\PP^N}(H))\to p_k^*\mathcal J^{\alpha_k-1}(f^*\OO_{\PP^N}(H))\to 0\]
we realise that there is on $\VZ_{1,\alpha}(\PP^N|H,d)$ a line bundle (the leftmost term in the exact sequence) with a natural section cutting the locus where the curve breaks, and the piece containing $p_k$ is mapped inside $H$.
\begin{lem}
 \[(\alpha_k\psi_k+\ev_k^*H)[\VZ_{1,\alpha}(\PP^N|H,d)]=[D_{1,\alpha;k}(\PP^N|H,d)],\]
\end{lem}
In what follows we shall give an explicit (recursive) description of the right-hand side, in terms of tautological classes on spaces of maps with lower numerical invariants.
\begin{remark}
 The divisors we are after are a union of logarithmic strata, because the locus where the log structure is trivial coincides precisely with the nice locus.
\end{remark}
By Corollary \ref{cor:log_smooth}, the divisorial logarithmic strata correspond to rays of the Artin fan.

\subsection{Combinatorial description}
As long as the minimal subcurve of genus one has positive degree, the picture is classical: the dual graph of the curve is bipartite - vertices mapped inside $H$ are distinguished from the others. In fact there is only one such vertex, because, if there were more, the relative position of their image in $\mathbb R_{\geq 0}$ under the tropical map would be unconstrained, therefore spanning a locus larger than a ray in the tropical moduli space. We follow in R. Vakil's footsteps in setting the following notation:
\begin{figure}[h]
    \centering
    \begin{minipage}{0.3\textwidth}
        \centering
        \Yagraph
        \caption{$\mathcal Y_a$}
    \end{minipage}\hfill
    \begin{minipage}{0.3\textwidth}
        \centering
        \Ybgraph
        \caption{$\mathcal Y_b$}
    \end{minipage}\hfill
    \begin{minipage}{0.3\textwidth}
        \centering
        \Ycgraph
        \caption{$\mathcal Y_c^+$}
    \end{minipage}
\end{figure}

Notice that continuity of the tropical map determines the mutual relationship among the edge lengths.

On the other hand, when there is a contracted elliptic subcurve - and it will be contracted into the hyperplane, because otherwise it wouldn't be generic, by density of the nice locus in $\VZ_{1,\tilde\alpha}(\PP^N|H,\tilde d)$ - the picture becomes much more complicated due to the alignment. The combs may break multiple times and they may even point in opposite directions.
\begin{figure}
 \tikz{
 \draw (0,0) circle(2pt)[fill=black] (0,2) circle(2pt)[fill=black] (0,4) circle(2pt)[fill=black] (0,6) circle(2pt)[fill=black] (0,-2) circle(2pt)[fill=black] (2,0) circle(2pt)[fill=white] (1.1,1.78) circle(2pt)[fill=black] (3.42,1.42) circle(2pt)[fill=black] (.58,-1.42) circle(2pt)[fill=black];
 \draw (2,0) -- node[above]{4} (0,0) (2,0) --node[above]{2} (1.1,1.78) --node[above]{2} (0,2) (1.1,1.78) --node[above]{1} (0,4) (2,0) --node[above]{1} (3.42,1.42) (2,0) --node[above]{1} (.58,-1.42) (3.42,1.42) --node[above]{5} (0,6) (.58,-1.42) --node[above]{3} (0,-2);
 \draw[red] (2,0) circle(2);
 }
\end{figure}



\subsection{Boundary loci}
We follow in Vakil's footsteps in setting the following notation:
\begin{enumerate}[label=(\alph*)]
 \item \[\mathcal Y^a=\M{0}{\lvert\alpha^{(0)}\rvert+r}{H}{d_0}\times_{H^r}\left(\VZc{1}{\alpha^{(1)}\cup\{m^{(1)}\}}{\PP^N|H}{d_1}\times\prod_{i=2}^r\M{0}{\alpha^{(i)}\cup\{m^{(i)}\}}{\PP^N|H}{d_i}\right)\]
 \item \[\mathcal Y^b=\M{0}{\lvert\alpha^{(0)}\rvert+r}{H}{d_0}\times_{H^r}\left(\M{0}{\alpha^{(1)}\cup\{m^{(1)},m^{(2)}\}}{\PP^N|H}{d_1}\times\prod_{i=3}^r\M{0}{\alpha^{(i)}\cup\{m^{(i)}\}}{\PP^N|H}{d_i}\right)\]
 \item \[\mathcal Y^c\subseteq \VZc{1}{\lvert\alpha^{(0)}\rvert+r}{H}{d_0}\times_{H^r}\prod_{i=1}^r\M{0}{\alpha^{(i)}\cup\{m^{(i)}\}}{\PP^N|H}{d_i}\]
 $\mathcal Y^c$ is cut within the latter by Gathmann's condition on line bundles:
 \[f_{|E}^*\mathcal O_H(1)\cong\mathcal O_E\left(\sum_{x_j\in E}\alpha_jx_j-\sum_{i=1}^r m^{(i)}y_i\right)\]\sidenote{Express the line bundle condition in terms of tautological classes?}
 Notice that such condition depends on the choice of $m^{(i)}$ but is really just a condition on the first factor of the product, so $\mathcal Y^c$ itself can be expressed as a product.
\end{enumerate}













\section{The space of relative centrally aligned maps}
\noindent Recall \cite{RSPW} that the moduli space of maps from centrally aligned curves is obtained by considering the  Cartesian diagram
\bcd
\tVZc{1}{n}{X}{\beta}\ar[d]\ar[r]\ar[dr,phantom,"\Box"] & \M{1}{n}{X}{\beta}\ar[d] \\
\MM_{1,n}^{\rm{ctr}}\ar[r] & \MM_{1,n}^{\dagger}
\ecd
so that objects of $\tVZc{1}{n}{X}{\beta}$ consist of
\begin{enumerate}
\item a centrally aligned curve $(C,M_C,\delta)$;
\item a stable map $f\colon C \to X$;
\end{enumerate}
subject to the condition that the subcurve $C_0 \subseteq C$, consisting of those components $C_v$ for which $\lambda(v) < \delta$, coincides with the maximal connected genus one subcurve contracted by $f$. They then define
\begin{equation*} \VZc{1}{n}{X}{\beta} \subseteq \tVZc{1}{n}{X}{\beta} \end{equation*}
to be the closed substack consisting of maps satisfying the \emph{factorisation condition}, namely that the map $f\colon C\to X$ factors through the associated contraction to a Smyth curve, i.e. there exists a map $\bar{f}$ making the following square commute:
\bcd
\widetilde C\ar[r]\ar[d] & \overline C\ar[d,"\bar f" left,dotted] \\
C\ar[r,"f"] & X
\ecd
One should think of the factorisation condition as identifying the main component of the moduli space.

\section{Definition of centrally aligned relative space}
\section{Characterisation of the closed points of the relative space}
\section{Recursion formula}

\subsection{The case $\sum\alpha=d$}

\subsection{Definition of the boundary components} According to Vakil's article there should be three sorts of boundary components.
\begin{enumerate}[label=(\alph*)]
 \item \[\mathcal Y^a=\M{0}{\lvert\alpha^{(0)}\rvert+r}{H}{d_0}\times_{H^r}\left(\VZc{1}{\alpha^{(1)}\cup\{m^{(1)}\}}{\PP^N|H}{d_1}\times\prod_{i=2}^r\M{0}{\alpha^{(i)}\cup\{m^{(i)}\}}{\PP^N|H}{d_i}\right)\]
 \item \[\mathcal Y^b=\M{0}{\lvert\alpha^{(0)}\rvert+r}{H}{d_0}\times_{H^r}\left(\M{0}{\alpha^{(1)}\cup\{m^{(1)},m^{(2)}\}}{\PP^N|H}{d_1}\times\prod_{i=3}^r\M{0}{\alpha^{(i)}\cup\{m^{(i)}\}}{\PP^N|H}{d_i}\right)\]
 \item \[\mathcal Y^c\subseteq \VZc{1}{\lvert\alpha^{(0)}\rvert+r}{H}{d_0}\times_{H^r}\prod_{i=1}^r\M{0}{\alpha^{(i)}\cup\{m^{(i)}\}}{\PP^N|H}{d_i}\]
 $\mathcal Y^c$ is cut within the latter by Gathmann's condition on line bundles:
 \[f_{|E}^*\mathcal O_H(1)\cong\mathcal O_E\left(\sum_{x_j\in E}\alpha_jx_j-\sum_{i=1}^r m^{(i)}y_i\right)\]\sidenote{Express the line bundle condition in terms of tautological classes?}
 Notice that such condition depends on the choice of $m^{(i)}$ but is really just a condition on the first factor of the product, so $\mathcal Y^c$ itself can be expressed as a product.
\end{enumerate}


\bibliographystyle{alpha}
\bibliography{relqm}

\bigskip\bigskip

\noindent Luca Battistella\\
Department of Mathematics, Imperial College London \\
\texttt{l.battistella14@imperial.ac.uk}\\

\noindent Navid Nabijou \\
Department of Mathematics, Imperial College London \\
\texttt{navid.nabijou09@imperial.ac.uk}\\

\noindent Dhruv Ranganathan \\
Department of Mathematics, Massachusetts Institute of Technology \\
\texttt{dhruvr@mit.edu}


\end{document}